\begin{recomendations}\label{chapter:Recomendations}

Para continuar el estudio y argumentar las conclusiones de la tesis se propone experimentar con un mayor volumen de im\'agenes en ambos casos de estudio. Se propone además tener en cuenta diferentes configuraciones de parámetros que permitan analizar a fondo la influencia de cada uno en la calidad de la restauraci\'on. Tambi\'en se sugiere la realizaci\'on de la investigaci\'on de las aplicaciones propuestas en el epígrafe \ref{sec:suggestions}, como son:
\begin{itemize}
	\item La utilizaci\'on para la compresi\'on de im\'agenes.
	\item La aplicaci\'on al aumento de la resoluci\'on de im\'agenes.
	\item La aplciaci\'on a la eliminaci\'on de ruido en im\'agenes.
	\item Implementaci\'on de la estrategia adaptativa para im\'agenes de colposcop\'ia.
\end{itemize}
Finalmente, se propone analizar si los costos computacionales del algoritmo \ref{al:PRA} son mejorables, o en cambio buscar alternativas para resolver m\'as eficientemente el problema \TSP/ mediante otros algoritmos. 

\end{recomendations}
