\begin{conclusions}\label{chapter:conclusions}

\qquad 

En este trabajo, con el objetivo principal de lograr la restauraci\'on de im\'agenes mediante un esquema basado en la ordenaci\'on suave de los parches de la imagen, se llevaron a cabo varias tareas en pos de su cumplimiento. En primer lugar se realiz\'o un estudio de los aspectos te\'oricos m\'as importantes en relaci\'on al problema de la restauraci\'on de im\'agenes. El resultado de este estudio qued\'o plasmado en el Cap\'itulo \ref{chapter:background}, donde se encuentran definiciones y lemas de utilidad para cap\'itulos posteriores. De igual forma en el Cap\'itulo \ref{chapter:scheme} se realiz\'o un estudio completo de la restauraci\'on \SOP/, mostrando de forma te\'orica y con ejemplos su funcionamiento. Se mostr\'o y explic\'o el pseudoc\'odigo del algoritmo \ref{al:PRA}, el cual es el n\'ucleo del esquema en general. Adem\'as se analizaron aspectos referentes a los costos computacionales de dicho algoritmo. En el Cap\'itulo \ref{chapter:code} se trataron los esfuerzos realizados para lograr la implementaci\'on del esquema de la restauraci\'on. En este cap\'itulo se plasmaron de forma breve las tecnolog\'ias empleadas y se explic\'o la estructura del m\'odulo creado, a modo de documentaci\'on del mismo. Se ejemplific\'o tambi\'en, de forma simple, como usar dicho m\'odulo implementado. Finalmente, se propuso una estrategia de restauraciones consecutivas, cuya utilidad se puso a prueba en la experimentaci\'on.

En Cap\'itulo \ref{chapter:results}, haciendo uso de la implementaci\'on desarrollada en el cap\'itulo anterior, se llev\'o a cabo una experimentaci\'on sobre dos casos de estudios diferentes. Este ejercicio tan importante, sac\'o a relucir tanto las virtudes como los inconvenientes de la restauraci\'on \SOP/. Teniendo como restauraciones de referencias \TELEA/, \NS/ y la m\'etrica de calidad \PSNR/, el primer caso de estudio demostr\'o la eficacia de \SOP/ en cierto grupo de im\'agenes. En este caso de estudio se experiment\'o con im\'agenes que se conoce su versión original y que los p\'ixeles de desinformaci\'on est\'an distribuidos de forma uniforme sobre la misma. La restauraci\'on \SOP/ super\'o con creces a \TELEA/ y \NS/, y al mismo tiempo demostr\'o lo acertado que es realizar restauraciones consecutivas. El segundo caso de estudio se utilizaron im\'agenes colposc\'opicas, las cuales tienen zonas de brillo a eliminar, con lo cual se desconoce su versi\'on original. La restauraci\'on \SOP/ no fue efectiva en este caso, a diferencia de \TELEA/ y \NS/ que s\'i mostraron un buen desempeño. Se concluy\'o con estos resultados que los parches vac\'ios son un fen\'omeno a evitar cuando se usa \SOP/, y la \'unica forma de lograrlo es aumentando el tamaño de los parches. Aumentar el tamaño del parche a su vez conlleva al aumento de la memoria a usar y el tiempo de ejecuci\'on del esquema de restauraci\'on. Al mismo tiempo, se detect\'o que la complejidad temporal de \SOP/ es invariante con respecto al \'area total de las zonas a recuperar. Con la intenci\'on lidiar con las desventajas anteriormente mencionadas, se propuso una estrategia adaptativa que permite aplicar la restauraci\'on \SOP/ a las im\'agenes de colposcop\'ias.

En resumen, luego de realizar el estudio, la implementaci\'on y la experimentación con el esquema de restauraci\'on \SOP/ se obtuvieron las siguientes conclusiones. Este tipo de restauraci\'on de conjunto con la estrategia de restauraciones consecutivas es eficaz con im\'agenes donde los p\'ixeles a restaurar est\'an distribuidos de forma uniforme a lo largo de la imagen, a\'un incluso si estos representan la mayor\'ia de la imagen. Por lo tanto se sugiere su posible aplicaci\'on a la compresi\'on de im\'agenes, ampliaci\'on de im\'agenes y eliminaci\'on de ruido. Por otro lado, \SOP/ no muestra buenos resultados con im\'agenes con zonas compactas de desinformaci\'on, escenarios en los cuales se hace necesario aplicar una estrategia adaptativa. Esta estrategia permite dividir la restauraci\'on en subrestauraciones donde se hace m\'as f\'acil usar par\'ametros m\'as grandes del algoritmo \ref{al:PRA} que mejoran la calidad de la restauraci\'on.


\end{conclusions}
