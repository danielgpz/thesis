\begin{conclusions}\label{chapter:conclusions}

En este trabajo se realizaron una serie de tareas encaminadas al cumplimiento del objetivo general, a saber, la restauración de imágenes mediante un ordenamiento suave de sus parches. En primer lugar se realiz\'o un estudio de los aspectos te\'oricos m\'as importantes en relaci\'on al problema de la restauraci\'on de im\'agenes en general. El resultado de este estudio qued\'o plasmado en el Cap\'itulo \ref{chapter:background}, donde se encuentran definiciones y lemas de utilidad para la lectura y comprensión de los cap\'itulos posteriores. De igual forma en el Cap\'itulo \ref{chapter:scheme} se realiz\'o un estudio completo de la restauraci\'on \SOP/, ilustrando de forma te\'orica y con ejemplos su funcionamiento. Se mostr\'o y explic\'o el pseudoc\'odigo del algoritmo \ref{al:PRA}, el cual constituye el n\'ucleo del esquema general. Adem\'as, se analizaron aspectos referentes a los costos computacionales de dicho algoritmo. En el Cap\'itulo \ref{chapter:code} se describe en detalle la implementaci\'on del esquema de la restauraci\'on. Para explicar la implementación realizada se mencionan primeramente las tecnologías empleadas y la estructura del módulo creado  con el objetivo de documentar el mismo. La implementación que se presenta introduce una mejora relacionada con la paralelización del esquema general. Esto redundó en beneficio del tiempo de ejecución. Como resultado de la implementación se propuso la estrategia de restauración consecutiva, la cual se explica en detalle y se utiliza en la experimentación realizada en el Capítulo \ref{chapter:results}. En el capítulo se dedica también un espacio a explicar  y ejemplificar, de forma simple, el uso del módulo desarrollado.

En el Cap\'itulo \ref{chapter:results}, haciendo uso de la implementaci\'on desarrollada en el cap\'itulo anterior, se llev\'o a cabo una experimentaci\'on sobre dos casos de estudios diferentes. Este ejercicio tan importante, sac\'o a relucir tanto las virtudes como los inconvenientes de la restauraci\'on \SOP/. Teniendo como restauraciones de referencias \TELEA/, \NS/ y la m\'etrica de calidad \PSNR/, el primer caso de estudio demostr\'o la eficacia de \SOP/ en cierto grupo de im\'agenes. En este caso de estudio se experiment\'o con im\'agenes de las que se conoce su versión original y en las cuales los p\'ixeles de desinformaci\'on est\'an distribuidos de forma uniforme sobre la misma. La restauraci\'on \SOP/ super\'o con creces a \TELEA/ y \NS/, y al mismo tiempo demostr\'o lo acertado que es realizar restauraciones consecutivas. En el segundo caso de estudio se utilizaron im\'agenes colposc\'opicas, las cuales tienen zonas de brillo a eliminar, pero se desconoce su versi\'on sin brillo. La restauraci\'on \SOP/ no fue efectiva en la experimentación realizada, a diferencia de \TELEA/ y \NS/ que s\'i mostraron un buen desempeño. 

Un resultado importante de la experimentación lo constituye el hecho de la importancia de evitar parches vacíos. Al analizar esto, relacionado con el costo computacional, se explicó que si se aumenta el tamaño de los parches, como sugiere el experimento, el costo temporal aumenta y los requerimientos de memoria. Sin embargo, se explica además, el costo temporal es invariante con respecto al área total de las zonas a recuperar. Como resultado de este estudio se propuso una estrategia adaptativa que permite aplicar la restauraci\'on \SOP/ a las im\'agenes de colposcop\'ias.

En resumen, luego de realizar el estudio, la implementaci\'on y la experimentación con el esquema de restauraci\'on \SOP/ se concluye lo siguiente:
\begin{itemize}
	\item La restauración \SOP/ es eficaz en imágenes donde los píxeles faltantes están distribuidos uniformemente en la imagen.
	\item El esquema \SOP/ con restauración consecutiva pudiera aplicarse a la compresi\'on, la ampliación y eliminación del ruido en imágenes.
	\item Una estrategia adaptativa por zonas, que se propone en el trabajo, sería la vía de evitar la restauración de partes con parches vacíos e incluso determinar el tamaño adecuado de ellos en cada zona de la imagen. Esta estrategia permite dividir la restauraci\'on en subrestauraciones donde se hace m\'as f\'acil usar par\'ametros m\'as grandes del algoritmo \ref{al:PRA} que mejoran la calidad de la restauraci\'on.
\end{itemize}

\end{conclusions}
