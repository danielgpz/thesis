%===================================================================================
% Chapter: Introduction
%===================================================================================
\chapter*{Introducción}\label{chapter:introduction}
\addcontentsline{toc}{chapter}{Introducción}
%===================================================================================

\qquad

Los primeros pasos en la recuperaci\'on de imágenes con un enfoque científico datan incluso antes del siglo \textbf{XVIII}. Pietro Edwards\cite{itwiki:PE}, quien dirigía la restauraci\'on de las pinturas p\'ublicas de Venecia, Italia, ten\'ia intenci\'on era restaurar las obras de artes que atesoraba su ciudad lo m\'as fiel posible a su respectivo autor original.

Las obras de artes plasmadas sobre un lienzo, un mural, el techo de una catedral inevitablemente han de sufrir el paso del tiempo. A lo largo de los años se agrietan, pierden poco a poco el color o incluso pueden desprenderse alguna de sus partes. Es trabajo de un grupo de especialistas restaurar las mismas para que estas persistan a lo largo de la historia en un estado lo m\'as similar posible al resultado inmediato de su autor. Uno de los ejemplos m\'as famosos es el del mural pintado por Leonardo Da Vinci: \textit{La \'ultima cena} \cite{wiki:CR-TLS}, el cual ha pasado varios ciclos de conservaci\'on-restauraci\'on en manos de diferentes personas que no han estado exentos de errores y que ha llevado a que hoy en d\'ia dicha obra se encuentre en un estado muy delicado, aunque con la tecnología actual se logr\'o frenar su continuo deterioro existen medidas muy estrictas para su exhibición al p\'ublico.

Esas fotos que nos legaron nuestros abuelos y padres en no pocas ocasiones se dañan al doblarlas o por la humedad, en los casos m\'as cr\'iticos si se quieren recuperar se deben someter a un proceso de este tipo. Hoy en d\'ia con la llegada de la era digital y el establecimiento del Internet, las fotos que le dejaremos a nuestros hijos ser\'an en su mayor\'ia digitales. Las im\'agenes digitales no se deterioran físicamente, el \'unico daño al que est\'an expuestas es a que se borren o se corrompa el archivo digital, lo cual no constituye una amenaza pues al ser subidas al Internet se crean suficientes copias de seguridad para evitar su p\'erdida. Con lo cual todos nuestros recuerdos, obras de artes, etc se encuentran a salvo de forma digital en Internet y se conservar\'an en su estado original para la eternidad.

Debido a que una imagen digital se encuentra representada en forma de matriz ha permitido la aplicaci\'on de resultados matem\'aticos y de la computaci\'on que mejoran considerablemente la calidad de la imagen restaurada. Sin embargo con la era digital surgen nuevas y \'utiles aplicaciones de la restauraci\'on de im\'agenes. Cuando se quiere eliminar un objeto, texto o persona de una imagen, esta zona se considera como la parte faltante y se restaura la imagen. Otra aplicaci\'on es para la compresi\'on, si se desea enviar un volumen considerable de imágenes por la red y no se quiere cargar la misma, una solución es solo enviar por cada imagen un conjunto de p\'ixeles representativo, lo cual representaría un volumen de informaci\'on m\'as reducido; entonces el remitente de estos realiza un proceso de restauraci\'on con cada conjunto para obtener las imágenes originales. Otras aplicaciones giran entorno a defectos inherentes a la imagen original, defectos que surgen comúnmente en el momento de tomar la imagen, ya sea ruido en la imagen, objetos en movimiento, brillo intenso en algunas zonas, planos desenfocados, emborronamiento, entre otros factores que hacen que no se obtenga una imagen fiel al espacio físico real que se desea capturar. Hay que tener en cuenta que la mayor\'ia de las aplicaciones a im\'agenes son extensibles a los vídeos, pues estos no son mas que una sucesión de imágenes consecutivas.

La restauraci\'on de im\'agenes digitales pertenece a un campo m\'as general, conocido como procesamiento de im\'agenes en el cual est\'an presenten otras tareas como la de clasificaci\'on, extracci\'on de informaci\'on, reconocimiento de patrones, etc. En el mundo de la medicina resultan muy \'utiles los avances en este campo de investigaci\'on, pues es posible detectar fen\'omenos en las im\'agenes que resultan difícil encontrar con la percepción del ojo humano; lo cual en muchos casos facilita el diagnostico de patologías en una fase muy temprana lo que se traduce en muchas vidas salvadas. 

En el caso especifico de la recuperaci\'on puede ayudar con otras problemáticas de las im\'agenes m\'edicas, las cuales no est\'an exentas de los defectos inherentes al aparato con que se captura la im\'agen o causados por las condiciones en que se encuentra la zona del cuerpo que se desea diagnosticar; se ha de tener en cuenta que gran parte de estas fotograf\'ias se toman en el interior del organismo donde no llega la luz natural y por el cual fluyen l\'iquidos y fluidos varios. Los defectos m\'as comunes son zonas oscuras, brillo intenso que se refleja sobre algunos tejidos h\'umedos, desenfocado o emborronamiento de algunas zonas; es donde la restauraci\'on de esas partes afectadas juega un rol importante para que el especialista pueda evaluar mejor la situaci\'on del paciente.

\section*{Estado del Arte}

El problema de recuperaci\'on de im\'agenes (en ingl\'es \II) y el procesamiento de imágenes en general ha sido atacado de disimiles formas. Con Inteligencia Artificial, haciendo uso del aprendizaje profundo; utilizando transformadas convenientes, filtros en una, dos o tres dimensiones; interpolaci\'on, clustering.

En \II, uno de los principales enfoques es aquel que comprende los m\'etodos de difusión o variacionales, cuyo principio es utilizar la informaci\'on de los p\'ixeles de la frontera de cada zona incompleta para propagarla a medida que se rellena cada uno. En \cite{ballester2001variational} se hace uso de un modelo variacional para su uso en im\'agenes a color y en escala de grises; en \cite{chan2001nontexture} se usa un modelo de difusión; y en \cite{telea2004image} una t\'ecnica basada en el m\'etodo de \textit{fast marching}. También se encuentran los métodos basados en ecuaciones en derivadas parciales (PDE), unos de los m\'as utilizados; donde se destacan los desarrollados por Bertalmío \cite{bertalmio2001navier,bertalmio2000image}, que utilizan las ecuaciones de la dinámica de fluidos o la ecuación de difusión del calor para propagar la intensidad de los píxeles de la frontera hacia las regiones faltantes.
