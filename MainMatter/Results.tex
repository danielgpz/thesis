\chapter{Experimentaci\'on y resultados}\label{chapter:results}
[General sobre la experimentaci\'on]

\section{Caso de estudio 1: Restauraci\'on conociendo la imagen original}

Descripci\'on general del experimento:
\begin{itemize}
	\item Las im\'agenes a utilizar son tomadas de la conocida colecci\'on de \textit{Berkley}. Todas en escala de grises y con mediana o pequeña resoluci\'on.
	\item Se consideraron 26 im\'agenes, de ellas:
	\begin{itemize}
		\item Conforman el grupo (A): 12 im\'agenes de dimensi\'on $512 \times 512$.
		\item Conforman el grupo (B): 14 im\'agenes, 11 de dimensi\'on $418 \times 312$, y 3 de $312 \times 418$.
	\end{itemize}
	\item Se realizaron 3 tipos diferentes de restauraciones, de la siguiente manera:
	\begin{itemize}
		\item Restauraci\'on con el esquema final propuesto en este trabajo:
		\begin{table}[H]
			\centering
			\begin{tabular}{|c|cccc|}
				\hline
				Iteraci\'on & $K$ & $\sqrt{n}$ & $B$ & $\epsilon$ \\\hline
				1 & $10$ & $5$ & $6$ & $10^4$\\
				2 & $10$ & $4$ & $7$ & $10^6$\\
				3 & $10$ & $4$ & $8$ & $10^8$\\\hline
			\end{tabular}
			\caption{Par\'ametros para el grupo (A)}
		\end{table}
		\begin{table}[H]
			\centering
			\begin{tabular}{|c|cccc|}
				\hline
				Iteraci\'on & $K$ & $\sqrt{n}$ & $B$ & $\epsilon$ \\\hline
				1 & $10$ & $6$ & $6$ & $10^4$\\
				2 & $10$ & $5$ & $7$ & $10^6$\\
				3 & $10$ & $5$ & $8$ & $10^8$\\\hline
			\end{tabular}
			\caption{Par\'ametros para el grupo (B)}
		\end{table}
		\item Restauraci\'on usando \texttt{NS} con radio de vecindad $8$, para ambos grupos (A) y (B).
		\item Restauraci\'on usando \texttt{TELEA} con radio de vecindad $8$, para ambos grupos (A) y (B).
	\end{itemize}
	\item La m\'etrica a usar para compara los resultados de las diferentes restauraciones es la medida \textbf{PSNR} \cite{enwiki:psnr}.
\end{itemize}

\makeatletter\newcommand\prberkley[1][all]{\ifnum\pdfstrcmp{#1}{all}=0\def\prberkley@out{\{"cameraman.tif": \{"corrupted": \{"file": "cameraman\_corrupted.tif", "psnr": 6.58\}, "iterations": \{"1": \{"psnr": 26.97, "duration": "0:26:53"\}, "2": \{"psnr": 29.68, "duration": "0:32:09"\}, "3": \{"psnr": 30.63, "duration": "0:51:35"\}\}, "duration": "1:50:39"\}, "house.tif": \{"corrupted": \{"file": "house\_corrupted.tif", "psnr": 5.68\}, "iterations": \{"1": \{"psnr": 31.92, "duration": "0:26:42"\}, "2": \{"psnr": 35.63, "duration": "0:31:14"\}, "3": \{"psnr": 36.07, "duration": "0:50:59"\}\}, "duration": "1:48:56"\}, "jetplane.tif": \{"corrupted": \{"file": "jetplane\_corrupted.tif", "psnr": 3.81\}, "iterations": \{"1": \{"psnr": 26.24, "duration": "0:27:19"\}, "2": \{"psnr": 28.43, "duration": "0:31:01"\}, "3": \{"psnr": 29.2, "duration": "0:49:18"\}\}, "duration": "1:47:39"\}, "lake.tif": \{"corrupted": \{"file": "lake\_corrupted.tif", "psnr": 6.14\}, "iterations": \{"1": \{"psnr": 24.42, "duration": "0:26:09"\}, "2": \{"psnr": 25.99, "duration": "0:29:18"\}, "3": \{"psnr": 26.48, "duration": "0:50:33"\}\}, "duration": "1:46:01"\}, "lena.tif": \{"corrupted": \{"file": "lena\_corrupted.tif", "psnr": 6.62\}, "iterations": \{"1": \{"psnr": 28.19, "duration": "0:29:43"\}, "2": \{"psnr": 30.26, "duration": "0:31:28"\}, "3": \{"psnr": 31.01, "duration": "0:51:15"\}\}, "duration": "1:52:26"\}, "livingroom.tif": \{"corrupted": \{"file": "livingroom\_corrupted.tif", "psnr": 6.9\}, "iterations": \{"1": \{"psnr": 25.18, "duration": "0:27:54"\}, "2": \{"psnr": 26.8, "duration": "0:30:51"\}, "3": \{"psnr": 27.53, "duration": "0:50:22"\}\}, "duration": "1:49:08"\}, "mandril.tif": \{"corrupted": \{"file": "mandril\_corrupted.tif", "psnr": 6.52\}, "iterations": \{"1": \{"psnr": 23.38, "duration": "0:26:30"\}, "2": \{"psnr": 24.31, "duration": "0:30:40"\}, "3": \{"psnr": 24.53, "duration": "0:51:02"\}\}, "duration": "1:48:14"\}, "peppers.tif": \{"corrupted": \{"file": "peppers\_corrupted.tif", "psnr": 6.91\}, "iterations": \{"1": \{"psnr": 27.71, "duration": "0:31:38"\}, "2": \{"psnr": 29.67, "duration": "0:32:17"\}, "3": \{"psnr": 30.54, "duration": "0:48:16"\}\}, "duration": "1:52:13"\}, "pirate.tif": \{"corrupted": \{"file": "pirate\_corrupted.tif", "psnr": 7.42\}, "iterations": \{"1": \{"psnr": 26.14, "duration": "0:25:38"\}, "2": \{"psnr": 27.67, "duration": "0:29:11"\}, "3": \{"psnr": 28.32, "duration": "0:48:03"\}\}, "duration": "1:42:53"\}, "walkbridge.tif": \{"corrupted": \{"file": "walkbridge\_corrupted.tif", "psnr": 7.07\}, "iterations": \{"1": \{"psnr": 22.94, "duration": "0:25:37"\}, "2": \{"psnr": 23.88, "duration": "0:29:15"\}, "3": \{"psnr": 24.23, "duration": "0:48:02"\}\}, "duration": "1:42:56"\}, "woman\_blonde.tif": \{"corrupted": \{"file": "woman\_blonde\_corrupted.tif", "psnr": 6.08\}, "iterations": \{"1": \{"psnr": 26.49, "duration": "0:25:35"\}, "2": \{"psnr": 27.87, "duration": "0:29:05"\}, "3": \{"psnr": 28.42, "duration": "0:47:55"\}\}, "duration": "1:42:36"\}, "woman\_darkhair.tif": \{"corrupted": \{"file": "woman\_darkhair\_corrupted.tif", "psnr": 7.22\}, "iterations": \{"1": \{"psnr": 33.08, "duration": "0:25:32"\}, "2": \{"psnr": 35.36, "duration": "0:29:09"\}, "3": \{"psnr": 35.77, "duration": "0:48:10"\}\}, "duration": "1:42:52"\}, "im11.jpg": \{"corrupted": \{"file": "im11\_corrupted.jpg", "psnr": 9.83\}, "iterations": \{"1": \{"psnr": 25.34, "duration": "0:14:35"\}, "2": \{"psnr": 26.34, "duration": "0:19:09"\}, "3": \{"psnr": 26.92, "duration": "0:30:33"\}\}, "duration": "1:04:18"\}, "im15.jpg": \{"corrupted": \{"file": "im15\_corrupted.jpg", "psnr": 6.79\}, "iterations": \{"1": \{"psnr": 19.93, "duration": "0:17:52"\}, "2": \{"psnr": 20.25, "duration": "0:24:34"\}, "3": \{"psnr": 20.3, "duration": "0:40:22"\}\}, "duration": "1:22:49"\}, "im16.jpg": \{"corrupted": \{"file": "im16\_corrupted.jpg", "psnr": 5.06\}, "iterations": \{"1": \{"psnr": 18.49, "duration": "0:21:39"\}, "2": \{"psnr": 18.73, "duration": "0:25:55"\}, "3": \{"psnr": 18.71, "duration": "0:37:03"\}\}, "duration": "1:24:38"\}, "im18.jpg": \{"corrupted": \{"file": "im18\_corrupted.jpg", "psnr": 3.14\}, "iterations": \{"1": \{"psnr": 22.45, "duration": "0:15:03"\}, "2": \{"psnr": 22.93, "duration": "0:19:13"\}, "3": \{"psnr": 23.32, "duration": "0:32:04"\}\}, "duration": "1:06:22"\}, "im19.jpg": \{"corrupted": \{"file": "im19\_corrupted.jpg", "psnr": 6.71\}, "iterations": \{"1": \{"psnr": 24.22, "duration": "0:15:06"\}, "2": \{"psnr": 24.7, "duration": "0:21:06"\}, "3": \{"psnr": 24.56, "duration": "0:39:00"\}\}, "duration": "1:15:13"\}, "im24.jpg": \{"corrupted": \{"file": "im24\_corrupted.jpg", "psnr": 3.67\}, "iterations": \{"1": \{"psnr": 27.41, "duration": "0:18:44"\}, "2": \{"psnr": 29.09, "duration": "0:23:58"\}, "3": \{"psnr": 29.77, "duration": "0:32:30"\}\}, "duration": "1:15:13"\}, "im25.jpg": \{"corrupted": \{"file": "im25\_corrupted.jpg", "psnr": 8.68\}, "iterations": \{"1": \{"psnr": 25.74, "duration": "0:15:12"\}, "2": \{"psnr": 26.53, "duration": "0:18:24"\}, "3": \{"psnr": 26.97, "duration": "0:28:41"\}\}, "duration": "1:02:18"\}, "im26.jpg": \{"corrupted": \{"file": "im26\_corrupted.jpg", "psnr": 5.78\}, "iterations": \{"1": \{"psnr": 21.38, "duration": "0:14:22"\}, "2": \{"psnr": 21.87, "duration": "0:18:20"\}, "3": \{"psnr": 21.91, "duration": "0:38:17"\}\}, "duration": "1:11:00"\}, "im28.jpg": \{"corrupted": \{"file": "im28\_corrupted.jpg", "psnr": 7.31\}, "iterations": \{"1": \{"psnr": 20.82, "duration": "0:14:46"\}, "2": \{"psnr": 22.02, "duration": "0:18:42"\}, "3": \{"psnr": 22.37, "duration": "0:30:16"\}\}, "duration": "1:03:45"\}, "im29.jpg": \{"corrupted": \{"file": "im29\_corrupted.jpg", "psnr": 14.7\}, "iterations": \{"1": \{"psnr": 24.97, "duration": "0:15:14"\}, "2": \{"psnr": 26.53, "duration": "0:19:20"\}, "3": \{"psnr": 27.38, "duration": "0:29:06"\}\}, "duration": "1:03:41"\}, "im30.jpg": \{"corrupted": \{"file": "im30\_corrupted.jpg", "psnr": 7.81\}, "iterations": \{"1": \{"psnr": 19.73, "duration": "0:15:10"\}, "2": \{"psnr": 20.71, "duration": "0:19:35"\}, "3": \{"psnr": 21.11, "duration": "0:31:42"\}\}, "duration": "1:06:28"\}, "im12.jpg": \{"corrupted": \{"file": "im12\_corrupted.jpg", "psnr": 7.83\}, "iterations": \{"1": \{"psnr": 19.43, "duration": "0:15:52"\}, "2": \{"psnr": 20.06, "duration": "0:22:26"\}, "3": \{"psnr": 20.33, "duration": "0:31:57"\}\}, "duration": "1:10:17"\}, "im17.jpg": \{"corrupted": \{"file": "im17\_corrupted.jpg", "psnr": 5.56\}, "iterations": \{"1": \{"psnr": 19.25, "duration": "0:17:05"\}, "2": \{"psnr": 20.59, "duration": "0:22:14"\}, "3": \{"psnr": 21.08, "duration": "0:33:28"\}\}, "duration": "1:12:48"\}, "im20.jpg": \{"corrupted": \{"file": "im20\_corrupted.jpg", "psnr": 8.72\}, "iterations": \{"1": \{"psnr": 21.57, "duration": "0:22:33"\}, "2": \{"psnr": 22.47, "duration": "0:24:28"\}, "3": \{"psnr": 22.53, "duration": "0:45:22"\}\}, "duration": "1:32:24"\}\}}\else\ifnum\pdfstrcmp{#1}{cameraman.tif}=0\let\prberkley@out\prberkley@I\else\ifnum\pdfstrcmp{#1}{house.tif}=0\let\prberkley@out\prberkley@II\else\ifnum\pdfstrcmp{#1}{jetplane.tif}=0\let\prberkley@out\prberkley@III\else\ifnum\pdfstrcmp{#1}{lake.tif}=0\let\prberkley@out\prberkley@IV\else\ifnum\pdfstrcmp{#1}{lena.tif}=0\let\prberkley@out\prberkley@V\else\ifnum\pdfstrcmp{#1}{livingroom.tif}=0\let\prberkley@out\prberkley@VI\else\ifnum\pdfstrcmp{#1}{mandril.tif}=0\let\prberkley@out\prberkley@VII\else\ifnum\pdfstrcmp{#1}{peppers.tif}=0\let\prberkley@out\prberkley@VIII\else\ifnum\pdfstrcmp{#1}{pirate.tif}=0\let\prberkley@out\prberkley@IX\else\ifnum\pdfstrcmp{#1}{walkbridge.tif}=0\let\prberkley@out\prberkley@X\else\ifnum\pdfstrcmp{#1}{woman_blonde.tif}=0\let\prberkley@out\prberkley@XI\else\ifnum\pdfstrcmp{#1}{woman_darkhair.tif}=0\let\prberkley@out\prberkley@XII\else\ifnum\pdfstrcmp{#1}{im11.jpg}=0\let\prberkley@out\prberkley@XIII\else\ifnum\pdfstrcmp{#1}{im15.jpg}=0\let\prberkley@out\prberkley@XIV\else\ifnum\pdfstrcmp{#1}{im16.jpg}=0\let\prberkley@out\prberkley@XV\else\ifnum\pdfstrcmp{#1}{im18.jpg}=0\let\prberkley@out\prberkley@XVI\else\ifnum\pdfstrcmp{#1}{im19.jpg}=0\let\prberkley@out\prberkley@XVII\else\ifnum\pdfstrcmp{#1}{im24.jpg}=0\let\prberkley@out\prberkley@XVIII\else\ifnum\pdfstrcmp{#1}{im25.jpg}=0\let\prberkley@out\prberkley@XIX\else\ifnum\pdfstrcmp{#1}{im26.jpg}=0\let\prberkley@out\prberkley@XX\else\ifnum\pdfstrcmp{#1}{im28.jpg}=0\let\prberkley@out\prberkley@XXI\else\ifnum\pdfstrcmp{#1}{im29.jpg}=0\let\prberkley@out\prberkley@XXII\else\ifnum\pdfstrcmp{#1}{im30.jpg}=0\let\prberkley@out\prberkley@XXIII\else\ifnum\pdfstrcmp{#1}{im12.jpg}=0\let\prberkley@out\prberkley@XXIV\else\ifnum\pdfstrcmp{#1}{im17.jpg}=0\let\prberkley@out\prberkley@XXV\else\ifnum\pdfstrcmp{#1}{im20.jpg}=0\let\prberkley@out\prberkley@XXVI\else\def\prberkley@out{??}\fi\fi\fi\fi\fi\fi\fi\fi\fi\fi\fi\fi\fi\fi\fi\fi\fi\fi\fi\fi\fi\fi\fi\fi\fi\fi\fi\prberkley@out}\newcommand\prberkley@I[1][all]{\ifnum\pdfstrcmp{#1}{all}=0\def\prberkley@I@out{\{"corrupted": \{"file": "cameraman\_corrupted.tif", "psnr": 6.58\}, "iterations": \{"1": \{"psnr": 26.97, "duration": "0:26:53"\}, "2": \{"psnr": 29.68, "duration": "0:32:09"\}, "3": \{"psnr": 30.63, "duration": "0:51:35"\}\}, "duration": "1:50:39"\}}\else\ifnum\pdfstrcmp{#1}{corrupted}=0\let\prberkley@I@out\prberkley@XXVII\else\ifnum\pdfstrcmp{#1}{iterations}=0\let\prberkley@I@out\prberkley@XXVIII\else\ifnum\pdfstrcmp{#1}{duration}=0\def\prberkley@I@out{1:50:39}\else\def\prberkley@I@out{??}\fi\fi\fi\fi\prberkley@I@out}\newcommand\prberkley@II[1][all]{\ifnum\pdfstrcmp{#1}{all}=0\def\prberkley@II@out{\{"corrupted": \{"file": "house\_corrupted.tif", "psnr": 5.68\}, "iterations": \{"1": \{"psnr": 31.92, "duration": "0:26:42"\}, "2": \{"psnr": 35.63, "duration": "0:31:14"\}, "3": \{"psnr": 36.07, "duration": "0:50:59"\}\}, "duration": "1:48:56"\}}\else\ifnum\pdfstrcmp{#1}{corrupted}=0\let\prberkley@II@out\prberkley@XXIX\else\ifnum\pdfstrcmp{#1}{iterations}=0\let\prberkley@II@out\prberkley@XXX\else\ifnum\pdfstrcmp{#1}{duration}=0\def\prberkley@II@out{1:48:56}\else\def\prberkley@II@out{??}\fi\fi\fi\fi\prberkley@II@out}\newcommand\prberkley@III[1][all]{\ifnum\pdfstrcmp{#1}{all}=0\def\prberkley@III@out{\{"corrupted": \{"file": "jetplane\_corrupted.tif", "psnr": 3.81\}, "iterations": \{"1": \{"psnr": 26.24, "duration": "0:27:19"\}, "2": \{"psnr": 28.43, "duration": "0:31:01"\}, "3": \{"psnr": 29.2, "duration": "0:49:18"\}\}, "duration": "1:47:39"\}}\else\ifnum\pdfstrcmp{#1}{corrupted}=0\let\prberkley@III@out\prberkley@XXXI\else\ifnum\pdfstrcmp{#1}{iterations}=0\let\prberkley@III@out\prberkley@XXXII\else\ifnum\pdfstrcmp{#1}{duration}=0\def\prberkley@III@out{1:47:39}\else\def\prberkley@III@out{??}\fi\fi\fi\fi\prberkley@III@out}\newcommand\prberkley@IV[1][all]{\ifnum\pdfstrcmp{#1}{all}=0\def\prberkley@IV@out{\{"corrupted": \{"file": "lake\_corrupted.tif", "psnr": 6.14\}, "iterations": \{"1": \{"psnr": 24.42, "duration": "0:26:09"\}, "2": \{"psnr": 25.99, "duration": "0:29:18"\}, "3": \{"psnr": 26.48, "duration": "0:50:33"\}\}, "duration": "1:46:01"\}}\else\ifnum\pdfstrcmp{#1}{corrupted}=0\let\prberkley@IV@out\prberkley@XXXIII\else\ifnum\pdfstrcmp{#1}{iterations}=0\let\prberkley@IV@out\prberkley@XXXIV\else\ifnum\pdfstrcmp{#1}{duration}=0\def\prberkley@IV@out{1:46:01}\else\def\prberkley@IV@out{??}\fi\fi\fi\fi\prberkley@IV@out}\newcommand\prberkley@V[1][all]{\ifnum\pdfstrcmp{#1}{all}=0\def\prberkley@V@out{\{"corrupted": \{"file": "lena\_corrupted.tif", "psnr": 6.62\}, "iterations": \{"1": \{"psnr": 28.19, "duration": "0:29:43"\}, "2": \{"psnr": 30.26, "duration": "0:31:28"\}, "3": \{"psnr": 31.01, "duration": "0:51:15"\}\}, "duration": "1:52:26"\}}\else\ifnum\pdfstrcmp{#1}{corrupted}=0\let\prberkley@V@out\prberkley@XXXV\else\ifnum\pdfstrcmp{#1}{iterations}=0\let\prberkley@V@out\prberkley@XXXVI\else\ifnum\pdfstrcmp{#1}{duration}=0\def\prberkley@V@out{1:52:26}\else\def\prberkley@V@out{??}\fi\fi\fi\fi\prberkley@V@out}\newcommand\prberkley@VI[1][all]{\ifnum\pdfstrcmp{#1}{all}=0\def\prberkley@VI@out{\{"corrupted": \{"file": "livingroom\_corrupted.tif", "psnr": 6.9\}, "iterations": \{"1": \{"psnr": 25.18, "duration": "0:27:54"\}, "2": \{"psnr": 26.8, "duration": "0:30:51"\}, "3": \{"psnr": 27.53, "duration": "0:50:22"\}\}, "duration": "1:49:08"\}}\else\ifnum\pdfstrcmp{#1}{corrupted}=0\let\prberkley@VI@out\prberkley@XXXVII\else\ifnum\pdfstrcmp{#1}{iterations}=0\let\prberkley@VI@out\prberkley@XXXVIII\else\ifnum\pdfstrcmp{#1}{duration}=0\def\prberkley@VI@out{1:49:08}\else\def\prberkley@VI@out{??}\fi\fi\fi\fi\prberkley@VI@out}\newcommand\prberkley@VII[1][all]{\ifnum\pdfstrcmp{#1}{all}=0\def\prberkley@VII@out{\{"corrupted": \{"file": "mandril\_corrupted.tif", "psnr": 6.52\}, "iterations": \{"1": \{"psnr": 23.38, "duration": "0:26:30"\}, "2": \{"psnr": 24.31, "duration": "0:30:40"\}, "3": \{"psnr": 24.53, "duration": "0:51:02"\}\}, "duration": "1:48:14"\}}\else\ifnum\pdfstrcmp{#1}{corrupted}=0\let\prberkley@VII@out\prberkley@XXXIX\else\ifnum\pdfstrcmp{#1}{iterations}=0\let\prberkley@VII@out\prberkley@XL\else\ifnum\pdfstrcmp{#1}{duration}=0\def\prberkley@VII@out{1:48:14}\else\def\prberkley@VII@out{??}\fi\fi\fi\fi\prberkley@VII@out}\newcommand\prberkley@VIII[1][all]{\ifnum\pdfstrcmp{#1}{all}=0\def\prberkley@VIII@out{\{"corrupted": \{"file": "peppers\_corrupted.tif", "psnr": 6.91\}, "iterations": \{"1": \{"psnr": 27.71, "duration": "0:31:38"\}, "2": \{"psnr": 29.67, "duration": "0:32:17"\}, "3": \{"psnr": 30.54, "duration": "0:48:16"\}\}, "duration": "1:52:13"\}}\else\ifnum\pdfstrcmp{#1}{corrupted}=0\let\prberkley@VIII@out\prberkley@XLI\else\ifnum\pdfstrcmp{#1}{iterations}=0\let\prberkley@VIII@out\prberkley@XLII\else\ifnum\pdfstrcmp{#1}{duration}=0\def\prberkley@VIII@out{1:52:13}\else\def\prberkley@VIII@out{??}\fi\fi\fi\fi\prberkley@VIII@out}\newcommand\prberkley@IX[1][all]{\ifnum\pdfstrcmp{#1}{all}=0\def\prberkley@IX@out{\{"corrupted": \{"file": "pirate\_corrupted.tif", "psnr": 7.42\}, "iterations": \{"1": \{"psnr": 26.14, "duration": "0:25:38"\}, "2": \{"psnr": 27.67, "duration": "0:29:11"\}, "3": \{"psnr": 28.32, "duration": "0:48:03"\}\}, "duration": "1:42:53"\}}\else\ifnum\pdfstrcmp{#1}{corrupted}=0\let\prberkley@IX@out\prberkley@XLIII\else\ifnum\pdfstrcmp{#1}{iterations}=0\let\prberkley@IX@out\prberkley@XLIV\else\ifnum\pdfstrcmp{#1}{duration}=0\def\prberkley@IX@out{1:42:53}\else\def\prberkley@IX@out{??}\fi\fi\fi\fi\prberkley@IX@out}\newcommand\prberkley@X[1][all]{\ifnum\pdfstrcmp{#1}{all}=0\def\prberkley@X@out{\{"corrupted": \{"file": "walkbridge\_corrupted.tif", "psnr": 7.07\}, "iterations": \{"1": \{"psnr": 22.94, "duration": "0:25:37"\}, "2": \{"psnr": 23.88, "duration": "0:29:15"\}, "3": \{"psnr": 24.23, "duration": "0:48:02"\}\}, "duration": "1:42:56"\}}\else\ifnum\pdfstrcmp{#1}{corrupted}=0\let\prberkley@X@out\prberkley@XLV\else\ifnum\pdfstrcmp{#1}{iterations}=0\let\prberkley@X@out\prberkley@XLVI\else\ifnum\pdfstrcmp{#1}{duration}=0\def\prberkley@X@out{1:42:56}\else\def\prberkley@X@out{??}\fi\fi\fi\fi\prberkley@X@out}\newcommand\prberkley@XI[1][all]{\ifnum\pdfstrcmp{#1}{all}=0\def\prberkley@XI@out{\{"corrupted": \{"file": "woman\_blonde\_corrupted.tif", "psnr": 6.08\}, "iterations": \{"1": \{"psnr": 26.49, "duration": "0:25:35"\}, "2": \{"psnr": 27.87, "duration": "0:29:05"\}, "3": \{"psnr": 28.42, "duration": "0:47:55"\}\}, "duration": "1:42:36"\}}\else\ifnum\pdfstrcmp{#1}{corrupted}=0\let\prberkley@XI@out\prberkley@XLVII\else\ifnum\pdfstrcmp{#1}{iterations}=0\let\prberkley@XI@out\prberkley@XLVIII\else\ifnum\pdfstrcmp{#1}{duration}=0\def\prberkley@XI@out{1:42:36}\else\def\prberkley@XI@out{??}\fi\fi\fi\fi\prberkley@XI@out}\newcommand\prberkley@XII[1][all]{\ifnum\pdfstrcmp{#1}{all}=0\def\prberkley@XII@out{\{"corrupted": \{"file": "woman\_darkhair\_corrupted.tif", "psnr": 7.22\}, "iterations": \{"1": \{"psnr": 33.08, "duration": "0:25:32"\}, "2": \{"psnr": 35.36, "duration": "0:29:09"\}, "3": \{"psnr": 35.77, "duration": "0:48:10"\}\}, "duration": "1:42:52"\}}\else\ifnum\pdfstrcmp{#1}{corrupted}=0\let\prberkley@XII@out\prberkley@XLIX\else\ifnum\pdfstrcmp{#1}{iterations}=0\let\prberkley@XII@out\prberkley@L\else\ifnum\pdfstrcmp{#1}{duration}=0\def\prberkley@XII@out{1:42:52}\else\def\prberkley@XII@out{??}\fi\fi\fi\fi\prberkley@XII@out}\newcommand\prberkley@XIII[1][all]{\ifnum\pdfstrcmp{#1}{all}=0\def\prberkley@XIII@out{\{"corrupted": \{"file": "im11\_corrupted.jpg", "psnr": 9.83\}, "iterations": \{"1": \{"psnr": 25.34, "duration": "0:14:35"\}, "2": \{"psnr": 26.34, "duration": "0:19:09"\}, "3": \{"psnr": 26.92, "duration": "0:30:33"\}\}, "duration": "1:04:18"\}}\else\ifnum\pdfstrcmp{#1}{corrupted}=0\let\prberkley@XIII@out\prberkley@LI\else\ifnum\pdfstrcmp{#1}{iterations}=0\let\prberkley@XIII@out\prberkley@LII\else\ifnum\pdfstrcmp{#1}{duration}=0\def\prberkley@XIII@out{1:04:18}\else\def\prberkley@XIII@out{??}\fi\fi\fi\fi\prberkley@XIII@out}\newcommand\prberkley@XIV[1][all]{\ifnum\pdfstrcmp{#1}{all}=0\def\prberkley@XIV@out{\{"corrupted": \{"file": "im15\_corrupted.jpg", "psnr": 6.79\}, "iterations": \{"1": \{"psnr": 19.93, "duration": "0:17:52"\}, "2": \{"psnr": 20.25, "duration": "0:24:34"\}, "3": \{"psnr": 20.3, "duration": "0:40:22"\}\}, "duration": "1:22:49"\}}\else\ifnum\pdfstrcmp{#1}{corrupted}=0\let\prberkley@XIV@out\prberkley@LIII\else\ifnum\pdfstrcmp{#1}{iterations}=0\let\prberkley@XIV@out\prberkley@LIV\else\ifnum\pdfstrcmp{#1}{duration}=0\def\prberkley@XIV@out{1:22:49}\else\def\prberkley@XIV@out{??}\fi\fi\fi\fi\prberkley@XIV@out}\newcommand\prberkley@XV[1][all]{\ifnum\pdfstrcmp{#1}{all}=0\def\prberkley@XV@out{\{"corrupted": \{"file": "im16\_corrupted.jpg", "psnr": 5.06\}, "iterations": \{"1": \{"psnr": 18.49, "duration": "0:21:39"\}, "2": \{"psnr": 18.73, "duration": "0:25:55"\}, "3": \{"psnr": 18.71, "duration": "0:37:03"\}\}, "duration": "1:24:38"\}}\else\ifnum\pdfstrcmp{#1}{corrupted}=0\let\prberkley@XV@out\prberkley@LV\else\ifnum\pdfstrcmp{#1}{iterations}=0\let\prberkley@XV@out\prberkley@LVI\else\ifnum\pdfstrcmp{#1}{duration}=0\def\prberkley@XV@out{1:24:38}\else\def\prberkley@XV@out{??}\fi\fi\fi\fi\prberkley@XV@out}\newcommand\prberkley@XVI[1][all]{\ifnum\pdfstrcmp{#1}{all}=0\def\prberkley@XVI@out{\{"corrupted": \{"file": "im18\_corrupted.jpg", "psnr": 3.14\}, "iterations": \{"1": \{"psnr": 22.45, "duration": "0:15:03"\}, "2": \{"psnr": 22.93, "duration": "0:19:13"\}, "3": \{"psnr": 23.32, "duration": "0:32:04"\}\}, "duration": "1:06:22"\}}\else\ifnum\pdfstrcmp{#1}{corrupted}=0\let\prberkley@XVI@out\prberkley@LVII\else\ifnum\pdfstrcmp{#1}{iterations}=0\let\prberkley@XVI@out\prberkley@LVIII\else\ifnum\pdfstrcmp{#1}{duration}=0\def\prberkley@XVI@out{1:06:22}\else\def\prberkley@XVI@out{??}\fi\fi\fi\fi\prberkley@XVI@out}\newcommand\prberkley@XVII[1][all]{\ifnum\pdfstrcmp{#1}{all}=0\def\prberkley@XVII@out{\{"corrupted": \{"file": "im19\_corrupted.jpg", "psnr": 6.71\}, "iterations": \{"1": \{"psnr": 24.22, "duration": "0:15:06"\}, "2": \{"psnr": 24.7, "duration": "0:21:06"\}, "3": \{"psnr": 24.56, "duration": "0:39:00"\}\}, "duration": "1:15:13"\}}\else\ifnum\pdfstrcmp{#1}{corrupted}=0\let\prberkley@XVII@out\prberkley@LIX\else\ifnum\pdfstrcmp{#1}{iterations}=0\let\prberkley@XVII@out\prberkley@LX\else\ifnum\pdfstrcmp{#1}{duration}=0\def\prberkley@XVII@out{1:15:13}\else\def\prberkley@XVII@out{??}\fi\fi\fi\fi\prberkley@XVII@out}\newcommand\prberkley@XVIII[1][all]{\ifnum\pdfstrcmp{#1}{all}=0\def\prberkley@XVIII@out{\{"corrupted": \{"file": "im24\_corrupted.jpg", "psnr": 3.67\}, "iterations": \{"1": \{"psnr": 27.41, "duration": "0:18:44"\}, "2": \{"psnr": 29.09, "duration": "0:23:58"\}, "3": \{"psnr": 29.77, "duration": "0:32:30"\}\}, "duration": "1:15:13"\}}\else\ifnum\pdfstrcmp{#1}{corrupted}=0\let\prberkley@XVIII@out\prberkley@LXI\else\ifnum\pdfstrcmp{#1}{iterations}=0\let\prberkley@XVIII@out\prberkley@LXII\else\ifnum\pdfstrcmp{#1}{duration}=0\def\prberkley@XVIII@out{1:15:13}\else\def\prberkley@XVIII@out{??}\fi\fi\fi\fi\prberkley@XVIII@out}\newcommand\prberkley@XIX[1][all]{\ifnum\pdfstrcmp{#1}{all}=0\def\prberkley@XIX@out{\{"corrupted": \{"file": "im25\_corrupted.jpg", "psnr": 8.68\}, "iterations": \{"1": \{"psnr": 25.74, "duration": "0:15:12"\}, "2": \{"psnr": 26.53, "duration": "0:18:24"\}, "3": \{"psnr": 26.97, "duration": "0:28:41"\}\}, "duration": "1:02:18"\}}\else\ifnum\pdfstrcmp{#1}{corrupted}=0\let\prberkley@XIX@out\prberkley@LXIII\else\ifnum\pdfstrcmp{#1}{iterations}=0\let\prberkley@XIX@out\prberkley@LXIV\else\ifnum\pdfstrcmp{#1}{duration}=0\def\prberkley@XIX@out{1:02:18}\else\def\prberkley@XIX@out{??}\fi\fi\fi\fi\prberkley@XIX@out}\newcommand\prberkley@XX[1][all]{\ifnum\pdfstrcmp{#1}{all}=0\def\prberkley@XX@out{\{"corrupted": \{"file": "im26\_corrupted.jpg", "psnr": 5.78\}, "iterations": \{"1": \{"psnr": 21.38, "duration": "0:14:22"\}, "2": \{"psnr": 21.87, "duration": "0:18:20"\}, "3": \{"psnr": 21.91, "duration": "0:38:17"\}\}, "duration": "1:11:00"\}}\else\ifnum\pdfstrcmp{#1}{corrupted}=0\let\prberkley@XX@out\prberkley@LXV\else\ifnum\pdfstrcmp{#1}{iterations}=0\let\prberkley@XX@out\prberkley@LXVI\else\ifnum\pdfstrcmp{#1}{duration}=0\def\prberkley@XX@out{1:11:00}\else\def\prberkley@XX@out{??}\fi\fi\fi\fi\prberkley@XX@out}\newcommand\prberkley@XXI[1][all]{\ifnum\pdfstrcmp{#1}{all}=0\def\prberkley@XXI@out{\{"corrupted": \{"file": "im28\_corrupted.jpg", "psnr": 7.31\}, "iterations": \{"1": \{"psnr": 20.82, "duration": "0:14:46"\}, "2": \{"psnr": 22.02, "duration": "0:18:42"\}, "3": \{"psnr": 22.37, "duration": "0:30:16"\}\}, "duration": "1:03:45"\}}\else\ifnum\pdfstrcmp{#1}{corrupted}=0\let\prberkley@XXI@out\prberkley@LXVII\else\ifnum\pdfstrcmp{#1}{iterations}=0\let\prberkley@XXI@out\prberkley@LXVIII\else\ifnum\pdfstrcmp{#1}{duration}=0\def\prberkley@XXI@out{1:03:45}\else\def\prberkley@XXI@out{??}\fi\fi\fi\fi\prberkley@XXI@out}\newcommand\prberkley@XXII[1][all]{\ifnum\pdfstrcmp{#1}{all}=0\def\prberkley@XXII@out{\{"corrupted": \{"file": "im29\_corrupted.jpg", "psnr": 14.7\}, "iterations": \{"1": \{"psnr": 24.97, "duration": "0:15:14"\}, "2": \{"psnr": 26.53, "duration": "0:19:20"\}, "3": \{"psnr": 27.38, "duration": "0:29:06"\}\}, "duration": "1:03:41"\}}\else\ifnum\pdfstrcmp{#1}{corrupted}=0\let\prberkley@XXII@out\prberkley@LXIX\else\ifnum\pdfstrcmp{#1}{iterations}=0\let\prberkley@XXII@out\prberkley@LXX\else\ifnum\pdfstrcmp{#1}{duration}=0\def\prberkley@XXII@out{1:03:41}\else\def\prberkley@XXII@out{??}\fi\fi\fi\fi\prberkley@XXII@out}\newcommand\prberkley@XXIII[1][all]{\ifnum\pdfstrcmp{#1}{all}=0\def\prberkley@XXIII@out{\{"corrupted": \{"file": "im30\_corrupted.jpg", "psnr": 7.81\}, "iterations": \{"1": \{"psnr": 19.73, "duration": "0:15:10"\}, "2": \{"psnr": 20.71, "duration": "0:19:35"\}, "3": \{"psnr": 21.11, "duration": "0:31:42"\}\}, "duration": "1:06:28"\}}\else\ifnum\pdfstrcmp{#1}{corrupted}=0\let\prberkley@XXIII@out\prberkley@LXXI\else\ifnum\pdfstrcmp{#1}{iterations}=0\let\prberkley@XXIII@out\prberkley@LXXII\else\ifnum\pdfstrcmp{#1}{duration}=0\def\prberkley@XXIII@out{1:06:28}\else\def\prberkley@XXIII@out{??}\fi\fi\fi\fi\prberkley@XXIII@out}\newcommand\prberkley@XXIV[1][all]{\ifnum\pdfstrcmp{#1}{all}=0\def\prberkley@XXIV@out{\{"corrupted": \{"file": "im12\_corrupted.jpg", "psnr": 7.83\}, "iterations": \{"1": \{"psnr": 19.43, "duration": "0:15:52"\}, "2": \{"psnr": 20.06, "duration": "0:22:26"\}, "3": \{"psnr": 20.33, "duration": "0:31:57"\}\}, "duration": "1:10:17"\}}\else\ifnum\pdfstrcmp{#1}{corrupted}=0\let\prberkley@XXIV@out\prberkley@LXXIII\else\ifnum\pdfstrcmp{#1}{iterations}=0\let\prberkley@XXIV@out\prberkley@LXXIV\else\ifnum\pdfstrcmp{#1}{duration}=0\def\prberkley@XXIV@out{1:10:17}\else\def\prberkley@XXIV@out{??}\fi\fi\fi\fi\prberkley@XXIV@out}\newcommand\prberkley@XXV[1][all]{\ifnum\pdfstrcmp{#1}{all}=0\def\prberkley@XXV@out{\{"corrupted": \{"file": "im17\_corrupted.jpg", "psnr": 5.56\}, "iterations": \{"1": \{"psnr": 19.25, "duration": "0:17:05"\}, "2": \{"psnr": 20.59, "duration": "0:22:14"\}, "3": \{"psnr": 21.08, "duration": "0:33:28"\}\}, "duration": "1:12:48"\}}\else\ifnum\pdfstrcmp{#1}{corrupted}=0\let\prberkley@XXV@out\prberkley@LXXV\else\ifnum\pdfstrcmp{#1}{iterations}=0\let\prberkley@XXV@out\prberkley@LXXVI\else\ifnum\pdfstrcmp{#1}{duration}=0\def\prberkley@XXV@out{1:12:48}\else\def\prberkley@XXV@out{??}\fi\fi\fi\fi\prberkley@XXV@out}\newcommand\prberkley@XXVI[1][all]{\ifnum\pdfstrcmp{#1}{all}=0\def\prberkley@XXVI@out{\{"corrupted": \{"file": "im20\_corrupted.jpg", "psnr": 8.72\}, "iterations": \{"1": \{"psnr": 21.57, "duration": "0:22:33"\}, "2": \{"psnr": 22.47, "duration": "0:24:28"\}, "3": \{"psnr": 22.53, "duration": "0:45:22"\}\}, "duration": "1:32:24"\}}\else\ifnum\pdfstrcmp{#1}{corrupted}=0\let\prberkley@XXVI@out\prberkley@LXXVII\else\ifnum\pdfstrcmp{#1}{iterations}=0\let\prberkley@XXVI@out\prberkley@LXXVIII\else\ifnum\pdfstrcmp{#1}{duration}=0\def\prberkley@XXVI@out{1:32:24}\else\def\prberkley@XXVI@out{??}\fi\fi\fi\fi\prberkley@XXVI@out}\newcommand\prberkley@XXVII[1][all]{\ifnum\pdfstrcmp{#1}{all}=0\def\prberkley@XXVII@out{\{"file": "cameraman\_corrupted.tif", "psnr": 6.58\}}\else\ifnum\pdfstrcmp{#1}{file}=0\def\prberkley@XXVII@out{cameraman\_corrupted.tif}\else\ifnum\pdfstrcmp{#1}{psnr}=0\def\prberkley@XXVII@out{6.58}\else\def\prberkley@XXVII@out{??}\fi\fi\fi\prberkley@XXVII@out}\newcommand\prberkley@XXVIII[1][all]{\ifnum\pdfstrcmp{#1}{all}=0\def\prberkley@XXVIII@out{\{"1": \{"psnr": 26.97, "duration": "0:26:53"\}, "2": \{"psnr": 29.68, "duration": "0:32:09"\}, "3": \{"psnr": 30.63, "duration": "0:51:35"\}\}}\else\ifnum\pdfstrcmp{#1}{1}=0\let\prberkley@XXVIII@out\prberkley@LXXIX\else\ifnum\pdfstrcmp{#1}{2}=0\let\prberkley@XXVIII@out\prberkley@LXXX\else\ifnum\pdfstrcmp{#1}{3}=0\let\prberkley@XXVIII@out\prberkley@LXXXI\else\def\prberkley@XXVIII@out{??}\fi\fi\fi\fi\prberkley@XXVIII@out}\newcommand\prberkley@XXIX[1][all]{\ifnum\pdfstrcmp{#1}{all}=0\def\prberkley@XXIX@out{\{"file": "house\_corrupted.tif", "psnr": 5.68\}}\else\ifnum\pdfstrcmp{#1}{file}=0\def\prberkley@XXIX@out{house\_corrupted.tif}\else\ifnum\pdfstrcmp{#1}{psnr}=0\def\prberkley@XXIX@out{5.68}\else\def\prberkley@XXIX@out{??}\fi\fi\fi\prberkley@XXIX@out}\newcommand\prberkley@XXX[1][all]{\ifnum\pdfstrcmp{#1}{all}=0\def\prberkley@XXX@out{\{"1": \{"psnr": 31.92, "duration": "0:26:42"\}, "2": \{"psnr": 35.63, "duration": "0:31:14"\}, "3": \{"psnr": 36.07, "duration": "0:50:59"\}\}}\else\ifnum\pdfstrcmp{#1}{1}=0\let\prberkley@XXX@out\prberkley@LXXXII\else\ifnum\pdfstrcmp{#1}{2}=0\let\prberkley@XXX@out\prberkley@LXXXIII\else\ifnum\pdfstrcmp{#1}{3}=0\let\prberkley@XXX@out\prberkley@LXXXIV\else\def\prberkley@XXX@out{??}\fi\fi\fi\fi\prberkley@XXX@out}\newcommand\prberkley@XXXI[1][all]{\ifnum\pdfstrcmp{#1}{all}=0\def\prberkley@XXXI@out{\{"file": "jetplane\_corrupted.tif", "psnr": 3.81\}}\else\ifnum\pdfstrcmp{#1}{file}=0\def\prberkley@XXXI@out{jetplane\_corrupted.tif}\else\ifnum\pdfstrcmp{#1}{psnr}=0\def\prberkley@XXXI@out{3.81}\else\def\prberkley@XXXI@out{??}\fi\fi\fi\prberkley@XXXI@out}\newcommand\prberkley@XXXII[1][all]{\ifnum\pdfstrcmp{#1}{all}=0\def\prberkley@XXXII@out{\{"1": \{"psnr": 26.24, "duration": "0:27:19"\}, "2": \{"psnr": 28.43, "duration": "0:31:01"\}, "3": \{"psnr": 29.2, "duration": "0:49:18"\}\}}\else\ifnum\pdfstrcmp{#1}{1}=0\let\prberkley@XXXII@out\prberkley@LXXXV\else\ifnum\pdfstrcmp{#1}{2}=0\let\prberkley@XXXII@out\prberkley@LXXXVI\else\ifnum\pdfstrcmp{#1}{3}=0\let\prberkley@XXXII@out\prberkley@LXXXVII\else\def\prberkley@XXXII@out{??}\fi\fi\fi\fi\prberkley@XXXII@out}\newcommand\prberkley@XXXIII[1][all]{\ifnum\pdfstrcmp{#1}{all}=0\def\prberkley@XXXIII@out{\{"file": "lake\_corrupted.tif", "psnr": 6.14\}}\else\ifnum\pdfstrcmp{#1}{file}=0\def\prberkley@XXXIII@out{lake\_corrupted.tif}\else\ifnum\pdfstrcmp{#1}{psnr}=0\def\prberkley@XXXIII@out{6.14}\else\def\prberkley@XXXIII@out{??}\fi\fi\fi\prberkley@XXXIII@out}\newcommand\prberkley@XXXIV[1][all]{\ifnum\pdfstrcmp{#1}{all}=0\def\prberkley@XXXIV@out{\{"1": \{"psnr": 24.42, "duration": "0:26:09"\}, "2": \{"psnr": 25.99, "duration": "0:29:18"\}, "3": \{"psnr": 26.48, "duration": "0:50:33"\}\}}\else\ifnum\pdfstrcmp{#1}{1}=0\let\prberkley@XXXIV@out\prberkley@LXXXVIII\else\ifnum\pdfstrcmp{#1}{2}=0\let\prberkley@XXXIV@out\prberkley@LXXXIX\else\ifnum\pdfstrcmp{#1}{3}=0\let\prberkley@XXXIV@out\prberkley@XC\else\def\prberkley@XXXIV@out{??}\fi\fi\fi\fi\prberkley@XXXIV@out}\newcommand\prberkley@XXXV[1][all]{\ifnum\pdfstrcmp{#1}{all}=0\def\prberkley@XXXV@out{\{"file": "lena\_corrupted.tif", "psnr": 6.62\}}\else\ifnum\pdfstrcmp{#1}{file}=0\def\prberkley@XXXV@out{lena\_corrupted.tif}\else\ifnum\pdfstrcmp{#1}{psnr}=0\def\prberkley@XXXV@out{6.62}\else\def\prberkley@XXXV@out{??}\fi\fi\fi\prberkley@XXXV@out}\newcommand\prberkley@XXXVI[1][all]{\ifnum\pdfstrcmp{#1}{all}=0\def\prberkley@XXXVI@out{\{"1": \{"psnr": 28.19, "duration": "0:29:43"\}, "2": \{"psnr": 30.26, "duration": "0:31:28"\}, "3": \{"psnr": 31.01, "duration": "0:51:15"\}\}}\else\ifnum\pdfstrcmp{#1}{1}=0\let\prberkley@XXXVI@out\prberkley@XCI\else\ifnum\pdfstrcmp{#1}{2}=0\let\prberkley@XXXVI@out\prberkley@XCII\else\ifnum\pdfstrcmp{#1}{3}=0\let\prberkley@XXXVI@out\prberkley@XCIII\else\def\prberkley@XXXVI@out{??}\fi\fi\fi\fi\prberkley@XXXVI@out}\newcommand\prberkley@XXXVII[1][all]{\ifnum\pdfstrcmp{#1}{all}=0\def\prberkley@XXXVII@out{\{"file": "livingroom\_corrupted.tif", "psnr": 6.9\}}\else\ifnum\pdfstrcmp{#1}{file}=0\def\prberkley@XXXVII@out{livingroom\_corrupted.tif}\else\ifnum\pdfstrcmp{#1}{psnr}=0\def\prberkley@XXXVII@out{6.9}\else\def\prberkley@XXXVII@out{??}\fi\fi\fi\prberkley@XXXVII@out}\newcommand\prberkley@XXXVIII[1][all]{\ifnum\pdfstrcmp{#1}{all}=0\def\prberkley@XXXVIII@out{\{"1": \{"psnr": 25.18, "duration": "0:27:54"\}, "2": \{"psnr": 26.8, "duration": "0:30:51"\}, "3": \{"psnr": 27.53, "duration": "0:50:22"\}\}}\else\ifnum\pdfstrcmp{#1}{1}=0\let\prberkley@XXXVIII@out\prberkley@XCIV\else\ifnum\pdfstrcmp{#1}{2}=0\let\prberkley@XXXVIII@out\prberkley@XCV\else\ifnum\pdfstrcmp{#1}{3}=0\let\prberkley@XXXVIII@out\prberkley@XCVI\else\def\prberkley@XXXVIII@out{??}\fi\fi\fi\fi\prberkley@XXXVIII@out}\newcommand\prberkley@XXXIX[1][all]{\ifnum\pdfstrcmp{#1}{all}=0\def\prberkley@XXXIX@out{\{"file": "mandril\_corrupted.tif", "psnr": 6.52\}}\else\ifnum\pdfstrcmp{#1}{file}=0\def\prberkley@XXXIX@out{mandril\_corrupted.tif}\else\ifnum\pdfstrcmp{#1}{psnr}=0\def\prberkley@XXXIX@out{6.52}\else\def\prberkley@XXXIX@out{??}\fi\fi\fi\prberkley@XXXIX@out}\newcommand\prberkley@XL[1][all]{\ifnum\pdfstrcmp{#1}{all}=0\def\prberkley@XL@out{\{"1": \{"psnr": 23.38, "duration": "0:26:30"\}, "2": \{"psnr": 24.31, "duration": "0:30:40"\}, "3": \{"psnr": 24.53, "duration": "0:51:02"\}\}}\else\ifnum\pdfstrcmp{#1}{1}=0\let\prberkley@XL@out\prberkley@XCVII\else\ifnum\pdfstrcmp{#1}{2}=0\let\prberkley@XL@out\prberkley@XCVIII\else\ifnum\pdfstrcmp{#1}{3}=0\let\prberkley@XL@out\prberkley@XCIX\else\def\prberkley@XL@out{??}\fi\fi\fi\fi\prberkley@XL@out}\newcommand\prberkley@XLI[1][all]{\ifnum\pdfstrcmp{#1}{all}=0\def\prberkley@XLI@out{\{"file": "peppers\_corrupted.tif", "psnr": 6.91\}}\else\ifnum\pdfstrcmp{#1}{file}=0\def\prberkley@XLI@out{peppers\_corrupted.tif}\else\ifnum\pdfstrcmp{#1}{psnr}=0\def\prberkley@XLI@out{6.91}\else\def\prberkley@XLI@out{??}\fi\fi\fi\prberkley@XLI@out}\newcommand\prberkley@XLII[1][all]{\ifnum\pdfstrcmp{#1}{all}=0\def\prberkley@XLII@out{\{"1": \{"psnr": 27.71, "duration": "0:31:38"\}, "2": \{"psnr": 29.67, "duration": "0:32:17"\}, "3": \{"psnr": 30.54, "duration": "0:48:16"\}\}}\else\ifnum\pdfstrcmp{#1}{1}=0\let\prberkley@XLII@out\prberkley@C\else\ifnum\pdfstrcmp{#1}{2}=0\let\prberkley@XLII@out\prberkley@CI\else\ifnum\pdfstrcmp{#1}{3}=0\let\prberkley@XLII@out\prberkley@CII\else\def\prberkley@XLII@out{??}\fi\fi\fi\fi\prberkley@XLII@out}\newcommand\prberkley@XLIII[1][all]{\ifnum\pdfstrcmp{#1}{all}=0\def\prberkley@XLIII@out{\{"file": "pirate\_corrupted.tif", "psnr": 7.42\}}\else\ifnum\pdfstrcmp{#1}{file}=0\def\prberkley@XLIII@out{pirate\_corrupted.tif}\else\ifnum\pdfstrcmp{#1}{psnr}=0\def\prberkley@XLIII@out{7.42}\else\def\prberkley@XLIII@out{??}\fi\fi\fi\prberkley@XLIII@out}\newcommand\prberkley@XLIV[1][all]{\ifnum\pdfstrcmp{#1}{all}=0\def\prberkley@XLIV@out{\{"1": \{"psnr": 26.14, "duration": "0:25:38"\}, "2": \{"psnr": 27.67, "duration": "0:29:11"\}, "3": \{"psnr": 28.32, "duration": "0:48:03"\}\}}\else\ifnum\pdfstrcmp{#1}{1}=0\let\prberkley@XLIV@out\prberkley@CIII\else\ifnum\pdfstrcmp{#1}{2}=0\let\prberkley@XLIV@out\prberkley@CIV\else\ifnum\pdfstrcmp{#1}{3}=0\let\prberkley@XLIV@out\prberkley@CV\else\def\prberkley@XLIV@out{??}\fi\fi\fi\fi\prberkley@XLIV@out}\newcommand\prberkley@XLV[1][all]{\ifnum\pdfstrcmp{#1}{all}=0\def\prberkley@XLV@out{\{"file": "walkbridge\_corrupted.tif", "psnr": 7.07\}}\else\ifnum\pdfstrcmp{#1}{file}=0\def\prberkley@XLV@out{walkbridge\_corrupted.tif}\else\ifnum\pdfstrcmp{#1}{psnr}=0\def\prberkley@XLV@out{7.07}\else\def\prberkley@XLV@out{??}\fi\fi\fi\prberkley@XLV@out}\newcommand\prberkley@XLVI[1][all]{\ifnum\pdfstrcmp{#1}{all}=0\def\prberkley@XLVI@out{\{"1": \{"psnr": 22.94, "duration": "0:25:37"\}, "2": \{"psnr": 23.88, "duration": "0:29:15"\}, "3": \{"psnr": 24.23, "duration": "0:48:02"\}\}}\else\ifnum\pdfstrcmp{#1}{1}=0\let\prberkley@XLVI@out\prberkley@CVI\else\ifnum\pdfstrcmp{#1}{2}=0\let\prberkley@XLVI@out\prberkley@CVII\else\ifnum\pdfstrcmp{#1}{3}=0\let\prberkley@XLVI@out\prberkley@CVIII\else\def\prberkley@XLVI@out{??}\fi\fi\fi\fi\prberkley@XLVI@out}\newcommand\prberkley@XLVII[1][all]{\ifnum\pdfstrcmp{#1}{all}=0\def\prberkley@XLVII@out{\{"file": "woman\_blonde\_corrupted.tif", "psnr": 6.08\}}\else\ifnum\pdfstrcmp{#1}{file}=0\def\prberkley@XLVII@out{woman\_blonde\_corrupted.tif}\else\ifnum\pdfstrcmp{#1}{psnr}=0\def\prberkley@XLVII@out{6.08}\else\def\prberkley@XLVII@out{??}\fi\fi\fi\prberkley@XLVII@out}\newcommand\prberkley@XLVIII[1][all]{\ifnum\pdfstrcmp{#1}{all}=0\def\prberkley@XLVIII@out{\{"1": \{"psnr": 26.49, "duration": "0:25:35"\}, "2": \{"psnr": 27.87, "duration": "0:29:05"\}, "3": \{"psnr": 28.42, "duration": "0:47:55"\}\}}\else\ifnum\pdfstrcmp{#1}{1}=0\let\prberkley@XLVIII@out\prberkley@CIX\else\ifnum\pdfstrcmp{#1}{2}=0\let\prberkley@XLVIII@out\prberkley@CX\else\ifnum\pdfstrcmp{#1}{3}=0\let\prberkley@XLVIII@out\prberkley@CXI\else\def\prberkley@XLVIII@out{??}\fi\fi\fi\fi\prberkley@XLVIII@out}\newcommand\prberkley@XLIX[1][all]{\ifnum\pdfstrcmp{#1}{all}=0\def\prberkley@XLIX@out{\{"file": "woman\_darkhair\_corrupted.tif", "psnr": 7.22\}}\else\ifnum\pdfstrcmp{#1}{file}=0\def\prberkley@XLIX@out{woman\_darkhair\_corrupted.tif}\else\ifnum\pdfstrcmp{#1}{psnr}=0\def\prberkley@XLIX@out{7.22}\else\def\prberkley@XLIX@out{??}\fi\fi\fi\prberkley@XLIX@out}\newcommand\prberkley@L[1][all]{\ifnum\pdfstrcmp{#1}{all}=0\def\prberkley@L@out{\{"1": \{"psnr": 33.08, "duration": "0:25:32"\}, "2": \{"psnr": 35.36, "duration": "0:29:09"\}, "3": \{"psnr": 35.77, "duration": "0:48:10"\}\}}\else\ifnum\pdfstrcmp{#1}{1}=0\let\prberkley@L@out\prberkley@CXII\else\ifnum\pdfstrcmp{#1}{2}=0\let\prberkley@L@out\prberkley@CXIII\else\ifnum\pdfstrcmp{#1}{3}=0\let\prberkley@L@out\prberkley@CXIV\else\def\prberkley@L@out{??}\fi\fi\fi\fi\prberkley@L@out}\newcommand\prberkley@LI[1][all]{\ifnum\pdfstrcmp{#1}{all}=0\def\prberkley@LI@out{\{"file": "im11\_corrupted.jpg", "psnr": 9.83\}}\else\ifnum\pdfstrcmp{#1}{file}=0\def\prberkley@LI@out{im11\_corrupted.jpg}\else\ifnum\pdfstrcmp{#1}{psnr}=0\def\prberkley@LI@out{9.83}\else\def\prberkley@LI@out{??}\fi\fi\fi\prberkley@LI@out}\newcommand\prberkley@LII[1][all]{\ifnum\pdfstrcmp{#1}{all}=0\def\prberkley@LII@out{\{"1": \{"psnr": 25.34, "duration": "0:14:35"\}, "2": \{"psnr": 26.34, "duration": "0:19:09"\}, "3": \{"psnr": 26.92, "duration": "0:30:33"\}\}}\else\ifnum\pdfstrcmp{#1}{1}=0\let\prberkley@LII@out\prberkley@CXV\else\ifnum\pdfstrcmp{#1}{2}=0\let\prberkley@LII@out\prberkley@CXVI\else\ifnum\pdfstrcmp{#1}{3}=0\let\prberkley@LII@out\prberkley@CXVII\else\def\prberkley@LII@out{??}\fi\fi\fi\fi\prberkley@LII@out}\newcommand\prberkley@LIII[1][all]{\ifnum\pdfstrcmp{#1}{all}=0\def\prberkley@LIII@out{\{"file": "im15\_corrupted.jpg", "psnr": 6.79\}}\else\ifnum\pdfstrcmp{#1}{file}=0\def\prberkley@LIII@out{im15\_corrupted.jpg}\else\ifnum\pdfstrcmp{#1}{psnr}=0\def\prberkley@LIII@out{6.79}\else\def\prberkley@LIII@out{??}\fi\fi\fi\prberkley@LIII@out}\newcommand\prberkley@LIV[1][all]{\ifnum\pdfstrcmp{#1}{all}=0\def\prberkley@LIV@out{\{"1": \{"psnr": 19.93, "duration": "0:17:52"\}, "2": \{"psnr": 20.25, "duration": "0:24:34"\}, "3": \{"psnr": 20.3, "duration": "0:40:22"\}\}}\else\ifnum\pdfstrcmp{#1}{1}=0\let\prberkley@LIV@out\prberkley@CXVIII\else\ifnum\pdfstrcmp{#1}{2}=0\let\prberkley@LIV@out\prberkley@CXIX\else\ifnum\pdfstrcmp{#1}{3}=0\let\prberkley@LIV@out\prberkley@CXX\else\def\prberkley@LIV@out{??}\fi\fi\fi\fi\prberkley@LIV@out}\newcommand\prberkley@LV[1][all]{\ifnum\pdfstrcmp{#1}{all}=0\def\prberkley@LV@out{\{"file": "im16\_corrupted.jpg", "psnr": 5.06\}}\else\ifnum\pdfstrcmp{#1}{file}=0\def\prberkley@LV@out{im16\_corrupted.jpg}\else\ifnum\pdfstrcmp{#1}{psnr}=0\def\prberkley@LV@out{5.06}\else\def\prberkley@LV@out{??}\fi\fi\fi\prberkley@LV@out}\newcommand\prberkley@LVI[1][all]{\ifnum\pdfstrcmp{#1}{all}=0\def\prberkley@LVI@out{\{"1": \{"psnr": 18.49, "duration": "0:21:39"\}, "2": \{"psnr": 18.73, "duration": "0:25:55"\}, "3": \{"psnr": 18.71, "duration": "0:37:03"\}\}}\else\ifnum\pdfstrcmp{#1}{1}=0\let\prberkley@LVI@out\prberkley@CXXI\else\ifnum\pdfstrcmp{#1}{2}=0\let\prberkley@LVI@out\prberkley@CXXII\else\ifnum\pdfstrcmp{#1}{3}=0\let\prberkley@LVI@out\prberkley@CXXIII\else\def\prberkley@LVI@out{??}\fi\fi\fi\fi\prberkley@LVI@out}\newcommand\prberkley@LVII[1][all]{\ifnum\pdfstrcmp{#1}{all}=0\def\prberkley@LVII@out{\{"file": "im18\_corrupted.jpg", "psnr": 3.14\}}\else\ifnum\pdfstrcmp{#1}{file}=0\def\prberkley@LVII@out{im18\_corrupted.jpg}\else\ifnum\pdfstrcmp{#1}{psnr}=0\def\prberkley@LVII@out{3.14}\else\def\prberkley@LVII@out{??}\fi\fi\fi\prberkley@LVII@out}\newcommand\prberkley@LVIII[1][all]{\ifnum\pdfstrcmp{#1}{all}=0\def\prberkley@LVIII@out{\{"1": \{"psnr": 22.45, "duration": "0:15:03"\}, "2": \{"psnr": 22.93, "duration": "0:19:13"\}, "3": \{"psnr": 23.32, "duration": "0:32:04"\}\}}\else\ifnum\pdfstrcmp{#1}{1}=0\let\prberkley@LVIII@out\prberkley@CXXIV\else\ifnum\pdfstrcmp{#1}{2}=0\let\prberkley@LVIII@out\prberkley@CXXV\else\ifnum\pdfstrcmp{#1}{3}=0\let\prberkley@LVIII@out\prberkley@CXXVI\else\def\prberkley@LVIII@out{??}\fi\fi\fi\fi\prberkley@LVIII@out}\newcommand\prberkley@LIX[1][all]{\ifnum\pdfstrcmp{#1}{all}=0\def\prberkley@LIX@out{\{"file": "im19\_corrupted.jpg", "psnr": 6.71\}}\else\ifnum\pdfstrcmp{#1}{file}=0\def\prberkley@LIX@out{im19\_corrupted.jpg}\else\ifnum\pdfstrcmp{#1}{psnr}=0\def\prberkley@LIX@out{6.71}\else\def\prberkley@LIX@out{??}\fi\fi\fi\prberkley@LIX@out}\newcommand\prberkley@LX[1][all]{\ifnum\pdfstrcmp{#1}{all}=0\def\prberkley@LX@out{\{"1": \{"psnr": 24.22, "duration": "0:15:06"\}, "2": \{"psnr": 24.7, "duration": "0:21:06"\}, "3": \{"psnr": 24.56, "duration": "0:39:00"\}\}}\else\ifnum\pdfstrcmp{#1}{1}=0\let\prberkley@LX@out\prberkley@CXXVII\else\ifnum\pdfstrcmp{#1}{2}=0\let\prberkley@LX@out\prberkley@CXXVIII\else\ifnum\pdfstrcmp{#1}{3}=0\let\prberkley@LX@out\prberkley@CXXIX\else\def\prberkley@LX@out{??}\fi\fi\fi\fi\prberkley@LX@out}\newcommand\prberkley@LXI[1][all]{\ifnum\pdfstrcmp{#1}{all}=0\def\prberkley@LXI@out{\{"file": "im24\_corrupted.jpg", "psnr": 3.67\}}\else\ifnum\pdfstrcmp{#1}{file}=0\def\prberkley@LXI@out{im24\_corrupted.jpg}\else\ifnum\pdfstrcmp{#1}{psnr}=0\def\prberkley@LXI@out{3.67}\else\def\prberkley@LXI@out{??}\fi\fi\fi\prberkley@LXI@out}\newcommand\prberkley@LXII[1][all]{\ifnum\pdfstrcmp{#1}{all}=0\def\prberkley@LXII@out{\{"1": \{"psnr": 27.41, "duration": "0:18:44"\}, "2": \{"psnr": 29.09, "duration": "0:23:58"\}, "3": \{"psnr": 29.77, "duration": "0:32:30"\}\}}\else\ifnum\pdfstrcmp{#1}{1}=0\let\prberkley@LXII@out\prberkley@CXXX\else\ifnum\pdfstrcmp{#1}{2}=0\let\prberkley@LXII@out\prberkley@CXXXI\else\ifnum\pdfstrcmp{#1}{3}=0\let\prberkley@LXII@out\prberkley@CXXXII\else\def\prberkley@LXII@out{??}\fi\fi\fi\fi\prberkley@LXII@out}\newcommand\prberkley@LXIII[1][all]{\ifnum\pdfstrcmp{#1}{all}=0\def\prberkley@LXIII@out{\{"file": "im25\_corrupted.jpg", "psnr": 8.68\}}\else\ifnum\pdfstrcmp{#1}{file}=0\def\prberkley@LXIII@out{im25\_corrupted.jpg}\else\ifnum\pdfstrcmp{#1}{psnr}=0\def\prberkley@LXIII@out{8.68}\else\def\prberkley@LXIII@out{??}\fi\fi\fi\prberkley@LXIII@out}\newcommand\prberkley@LXIV[1][all]{\ifnum\pdfstrcmp{#1}{all}=0\def\prberkley@LXIV@out{\{"1": \{"psnr": 25.74, "duration": "0:15:12"\}, "2": \{"psnr": 26.53, "duration": "0:18:24"\}, "3": \{"psnr": 26.97, "duration": "0:28:41"\}\}}\else\ifnum\pdfstrcmp{#1}{1}=0\let\prberkley@LXIV@out\prberkley@CXXXIII\else\ifnum\pdfstrcmp{#1}{2}=0\let\prberkley@LXIV@out\prberkley@CXXXIV\else\ifnum\pdfstrcmp{#1}{3}=0\let\prberkley@LXIV@out\prberkley@CXXXV\else\def\prberkley@LXIV@out{??}\fi\fi\fi\fi\prberkley@LXIV@out}\newcommand\prberkley@LXV[1][all]{\ifnum\pdfstrcmp{#1}{all}=0\def\prberkley@LXV@out{\{"file": "im26\_corrupted.jpg", "psnr": 5.78\}}\else\ifnum\pdfstrcmp{#1}{file}=0\def\prberkley@LXV@out{im26\_corrupted.jpg}\else\ifnum\pdfstrcmp{#1}{psnr}=0\def\prberkley@LXV@out{5.78}\else\def\prberkley@LXV@out{??}\fi\fi\fi\prberkley@LXV@out}\newcommand\prberkley@LXVI[1][all]{\ifnum\pdfstrcmp{#1}{all}=0\def\prberkley@LXVI@out{\{"1": \{"psnr": 21.38, "duration": "0:14:22"\}, "2": \{"psnr": 21.87, "duration": "0:18:20"\}, "3": \{"psnr": 21.91, "duration": "0:38:17"\}\}}\else\ifnum\pdfstrcmp{#1}{1}=0\let\prberkley@LXVI@out\prberkley@CXXXVI\else\ifnum\pdfstrcmp{#1}{2}=0\let\prberkley@LXVI@out\prberkley@CXXXVII\else\ifnum\pdfstrcmp{#1}{3}=0\let\prberkley@LXVI@out\prberkley@CXXXVIII\else\def\prberkley@LXVI@out{??}\fi\fi\fi\fi\prberkley@LXVI@out}\newcommand\prberkley@LXVII[1][all]{\ifnum\pdfstrcmp{#1}{all}=0\def\prberkley@LXVII@out{\{"file": "im28\_corrupted.jpg", "psnr": 7.31\}}\else\ifnum\pdfstrcmp{#1}{file}=0\def\prberkley@LXVII@out{im28\_corrupted.jpg}\else\ifnum\pdfstrcmp{#1}{psnr}=0\def\prberkley@LXVII@out{7.31}\else\def\prberkley@LXVII@out{??}\fi\fi\fi\prberkley@LXVII@out}\newcommand\prberkley@LXVIII[1][all]{\ifnum\pdfstrcmp{#1}{all}=0\def\prberkley@LXVIII@out{\{"1": \{"psnr": 20.82, "duration": "0:14:46"\}, "2": \{"psnr": 22.02, "duration": "0:18:42"\}, "3": \{"psnr": 22.37, "duration": "0:30:16"\}\}}\else\ifnum\pdfstrcmp{#1}{1}=0\let\prberkley@LXVIII@out\prberkley@CXXXIX\else\ifnum\pdfstrcmp{#1}{2}=0\let\prberkley@LXVIII@out\prberkley@CXL\else\ifnum\pdfstrcmp{#1}{3}=0\let\prberkley@LXVIII@out\prberkley@CXLI\else\def\prberkley@LXVIII@out{??}\fi\fi\fi\fi\prberkley@LXVIII@out}\newcommand\prberkley@LXIX[1][all]{\ifnum\pdfstrcmp{#1}{all}=0\def\prberkley@LXIX@out{\{"file": "im29\_corrupted.jpg", "psnr": 14.7\}}\else\ifnum\pdfstrcmp{#1}{file}=0\def\prberkley@LXIX@out{im29\_corrupted.jpg}\else\ifnum\pdfstrcmp{#1}{psnr}=0\def\prberkley@LXIX@out{14.7}\else\def\prberkley@LXIX@out{??}\fi\fi\fi\prberkley@LXIX@out}\newcommand\prberkley@LXX[1][all]{\ifnum\pdfstrcmp{#1}{all}=0\def\prberkley@LXX@out{\{"1": \{"psnr": 24.97, "duration": "0:15:14"\}, "2": \{"psnr": 26.53, "duration": "0:19:20"\}, "3": \{"psnr": 27.38, "duration": "0:29:06"\}\}}\else\ifnum\pdfstrcmp{#1}{1}=0\let\prberkley@LXX@out\prberkley@CXLII\else\ifnum\pdfstrcmp{#1}{2}=0\let\prberkley@LXX@out\prberkley@CXLIII\else\ifnum\pdfstrcmp{#1}{3}=0\let\prberkley@LXX@out\prberkley@CXLIV\else\def\prberkley@LXX@out{??}\fi\fi\fi\fi\prberkley@LXX@out}\newcommand\prberkley@LXXI[1][all]{\ifnum\pdfstrcmp{#1}{all}=0\def\prberkley@LXXI@out{\{"file": "im30\_corrupted.jpg", "psnr": 7.81\}}\else\ifnum\pdfstrcmp{#1}{file}=0\def\prberkley@LXXI@out{im30\_corrupted.jpg}\else\ifnum\pdfstrcmp{#1}{psnr}=0\def\prberkley@LXXI@out{7.81}\else\def\prberkley@LXXI@out{??}\fi\fi\fi\prberkley@LXXI@out}\newcommand\prberkley@LXXII[1][all]{\ifnum\pdfstrcmp{#1}{all}=0\def\prberkley@LXXII@out{\{"1": \{"psnr": 19.73, "duration": "0:15:10"\}, "2": \{"psnr": 20.71, "duration": "0:19:35"\}, "3": \{"psnr": 21.11, "duration": "0:31:42"\}\}}\else\ifnum\pdfstrcmp{#1}{1}=0\let\prberkley@LXXII@out\prberkley@CXLV\else\ifnum\pdfstrcmp{#1}{2}=0\let\prberkley@LXXII@out\prberkley@CXLVI\else\ifnum\pdfstrcmp{#1}{3}=0\let\prberkley@LXXII@out\prberkley@CXLVII\else\def\prberkley@LXXII@out{??}\fi\fi\fi\fi\prberkley@LXXII@out}\newcommand\prberkley@LXXIII[1][all]{\ifnum\pdfstrcmp{#1}{all}=0\def\prberkley@LXXIII@out{\{"file": "im12\_corrupted.jpg", "psnr": 7.83\}}\else\ifnum\pdfstrcmp{#1}{file}=0\def\prberkley@LXXIII@out{im12\_corrupted.jpg}\else\ifnum\pdfstrcmp{#1}{psnr}=0\def\prberkley@LXXIII@out{7.83}\else\def\prberkley@LXXIII@out{??}\fi\fi\fi\prberkley@LXXIII@out}\newcommand\prberkley@LXXIV[1][all]{\ifnum\pdfstrcmp{#1}{all}=0\def\prberkley@LXXIV@out{\{"1": \{"psnr": 19.43, "duration": "0:15:52"\}, "2": \{"psnr": 20.06, "duration": "0:22:26"\}, "3": \{"psnr": 20.33, "duration": "0:31:57"\}\}}\else\ifnum\pdfstrcmp{#1}{1}=0\let\prberkley@LXXIV@out\prberkley@CXLVIII\else\ifnum\pdfstrcmp{#1}{2}=0\let\prberkley@LXXIV@out\prberkley@CXLIX\else\ifnum\pdfstrcmp{#1}{3}=0\let\prberkley@LXXIV@out\prberkley@CL\else\def\prberkley@LXXIV@out{??}\fi\fi\fi\fi\prberkley@LXXIV@out}\newcommand\prberkley@LXXV[1][all]{\ifnum\pdfstrcmp{#1}{all}=0\def\prberkley@LXXV@out{\{"file": "im17\_corrupted.jpg", "psnr": 5.56\}}\else\ifnum\pdfstrcmp{#1}{file}=0\def\prberkley@LXXV@out{im17\_corrupted.jpg}\else\ifnum\pdfstrcmp{#1}{psnr}=0\def\prberkley@LXXV@out{5.56}\else\def\prberkley@LXXV@out{??}\fi\fi\fi\prberkley@LXXV@out}\newcommand\prberkley@LXXVI[1][all]{\ifnum\pdfstrcmp{#1}{all}=0\def\prberkley@LXXVI@out{\{"1": \{"psnr": 19.25, "duration": "0:17:05"\}, "2": \{"psnr": 20.59, "duration": "0:22:14"\}, "3": \{"psnr": 21.08, "duration": "0:33:28"\}\}}\else\ifnum\pdfstrcmp{#1}{1}=0\let\prberkley@LXXVI@out\prberkley@CLI\else\ifnum\pdfstrcmp{#1}{2}=0\let\prberkley@LXXVI@out\prberkley@CLII\else\ifnum\pdfstrcmp{#1}{3}=0\let\prberkley@LXXVI@out\prberkley@CLIII\else\def\prberkley@LXXVI@out{??}\fi\fi\fi\fi\prberkley@LXXVI@out}\newcommand\prberkley@LXXVII[1][all]{\ifnum\pdfstrcmp{#1}{all}=0\def\prberkley@LXXVII@out{\{"file": "im20\_corrupted.jpg", "psnr": 8.72\}}\else\ifnum\pdfstrcmp{#1}{file}=0\def\prberkley@LXXVII@out{im20\_corrupted.jpg}\else\ifnum\pdfstrcmp{#1}{psnr}=0\def\prberkley@LXXVII@out{8.72}\else\def\prberkley@LXXVII@out{??}\fi\fi\fi\prberkley@LXXVII@out}\newcommand\prberkley@LXXVIII[1][all]{\ifnum\pdfstrcmp{#1}{all}=0\def\prberkley@LXXVIII@out{\{"1": \{"psnr": 21.57, "duration": "0:22:33"\}, "2": \{"psnr": 22.47, "duration": "0:24:28"\}, "3": \{"psnr": 22.53, "duration": "0:45:22"\}\}}\else\ifnum\pdfstrcmp{#1}{1}=0\let\prberkley@LXXVIII@out\prberkley@CLIV\else\ifnum\pdfstrcmp{#1}{2}=0\let\prberkley@LXXVIII@out\prberkley@CLV\else\ifnum\pdfstrcmp{#1}{3}=0\let\prberkley@LXXVIII@out\prberkley@CLVI\else\def\prberkley@LXXVIII@out{??}\fi\fi\fi\fi\prberkley@LXXVIII@out}\newcommand\prberkley@LXXIX[1][all]{\ifnum\pdfstrcmp{#1}{all}=0\def\prberkley@LXXIX@out{\{"psnr": 26.97, "duration": "0:26:53"\}}\else\ifnum\pdfstrcmp{#1}{psnr}=0\def\prberkley@LXXIX@out{26.97}\else\ifnum\pdfstrcmp{#1}{duration}=0\def\prberkley@LXXIX@out{0:26:53}\else\def\prberkley@LXXIX@out{??}\fi\fi\fi\prberkley@LXXIX@out}\newcommand\prberkley@LXXX[1][all]{\ifnum\pdfstrcmp{#1}{all}=0\def\prberkley@LXXX@out{\{"psnr": 29.68, "duration": "0:32:09"\}}\else\ifnum\pdfstrcmp{#1}{psnr}=0\def\prberkley@LXXX@out{29.68}\else\ifnum\pdfstrcmp{#1}{duration}=0\def\prberkley@LXXX@out{0:32:09}\else\def\prberkley@LXXX@out{??}\fi\fi\fi\prberkley@LXXX@out}\newcommand\prberkley@LXXXI[1][all]{\ifnum\pdfstrcmp{#1}{all}=0\def\prberkley@LXXXI@out{\{"psnr": 30.63, "duration": "0:51:35"\}}\else\ifnum\pdfstrcmp{#1}{psnr}=0\def\prberkley@LXXXI@out{30.63}\else\ifnum\pdfstrcmp{#1}{duration}=0\def\prberkley@LXXXI@out{0:51:35}\else\def\prberkley@LXXXI@out{??}\fi\fi\fi\prberkley@LXXXI@out}\newcommand\prberkley@LXXXII[1][all]{\ifnum\pdfstrcmp{#1}{all}=0\def\prberkley@LXXXII@out{\{"psnr": 31.92, "duration": "0:26:42"\}}\else\ifnum\pdfstrcmp{#1}{psnr}=0\def\prberkley@LXXXII@out{31.92}\else\ifnum\pdfstrcmp{#1}{duration}=0\def\prberkley@LXXXII@out{0:26:42}\else\def\prberkley@LXXXII@out{??}\fi\fi\fi\prberkley@LXXXII@out}\newcommand\prberkley@LXXXIII[1][all]{\ifnum\pdfstrcmp{#1}{all}=0\def\prberkley@LXXXIII@out{\{"psnr": 35.63, "duration": "0:31:14"\}}\else\ifnum\pdfstrcmp{#1}{psnr}=0\def\prberkley@LXXXIII@out{35.63}\else\ifnum\pdfstrcmp{#1}{duration}=0\def\prberkley@LXXXIII@out{0:31:14}\else\def\prberkley@LXXXIII@out{??}\fi\fi\fi\prberkley@LXXXIII@out}\newcommand\prberkley@LXXXIV[1][all]{\ifnum\pdfstrcmp{#1}{all}=0\def\prberkley@LXXXIV@out{\{"psnr": 36.07, "duration": "0:50:59"\}}\else\ifnum\pdfstrcmp{#1}{psnr}=0\def\prberkley@LXXXIV@out{36.07}\else\ifnum\pdfstrcmp{#1}{duration}=0\def\prberkley@LXXXIV@out{0:50:59}\else\def\prberkley@LXXXIV@out{??}\fi\fi\fi\prberkley@LXXXIV@out}\newcommand\prberkley@LXXXV[1][all]{\ifnum\pdfstrcmp{#1}{all}=0\def\prberkley@LXXXV@out{\{"psnr": 26.24, "duration": "0:27:19"\}}\else\ifnum\pdfstrcmp{#1}{psnr}=0\def\prberkley@LXXXV@out{26.24}\else\ifnum\pdfstrcmp{#1}{duration}=0\def\prberkley@LXXXV@out{0:27:19}\else\def\prberkley@LXXXV@out{??}\fi\fi\fi\prberkley@LXXXV@out}\newcommand\prberkley@LXXXVI[1][all]{\ifnum\pdfstrcmp{#1}{all}=0\def\prberkley@LXXXVI@out{\{"psnr": 28.43, "duration": "0:31:01"\}}\else\ifnum\pdfstrcmp{#1}{psnr}=0\def\prberkley@LXXXVI@out{28.43}\else\ifnum\pdfstrcmp{#1}{duration}=0\def\prberkley@LXXXVI@out{0:31:01}\else\def\prberkley@LXXXVI@out{??}\fi\fi\fi\prberkley@LXXXVI@out}\newcommand\prberkley@LXXXVII[1][all]{\ifnum\pdfstrcmp{#1}{all}=0\def\prberkley@LXXXVII@out{\{"psnr": 29.2, "duration": "0:49:18"\}}\else\ifnum\pdfstrcmp{#1}{psnr}=0\def\prberkley@LXXXVII@out{29.2}\else\ifnum\pdfstrcmp{#1}{duration}=0\def\prberkley@LXXXVII@out{0:49:18}\else\def\prberkley@LXXXVII@out{??}\fi\fi\fi\prberkley@LXXXVII@out}\newcommand\prberkley@LXXXVIII[1][all]{\ifnum\pdfstrcmp{#1}{all}=0\def\prberkley@LXXXVIII@out{\{"psnr": 24.42, "duration": "0:26:09"\}}\else\ifnum\pdfstrcmp{#1}{psnr}=0\def\prberkley@LXXXVIII@out{24.42}\else\ifnum\pdfstrcmp{#1}{duration}=0\def\prberkley@LXXXVIII@out{0:26:09}\else\def\prberkley@LXXXVIII@out{??}\fi\fi\fi\prberkley@LXXXVIII@out}\newcommand\prberkley@LXXXIX[1][all]{\ifnum\pdfstrcmp{#1}{all}=0\def\prberkley@LXXXIX@out{\{"psnr": 25.99, "duration": "0:29:18"\}}\else\ifnum\pdfstrcmp{#1}{psnr}=0\def\prberkley@LXXXIX@out{25.99}\else\ifnum\pdfstrcmp{#1}{duration}=0\def\prberkley@LXXXIX@out{0:29:18}\else\def\prberkley@LXXXIX@out{??}\fi\fi\fi\prberkley@LXXXIX@out}\newcommand\prberkley@XC[1][all]{\ifnum\pdfstrcmp{#1}{all}=0\def\prberkley@XC@out{\{"psnr": 26.48, "duration": "0:50:33"\}}\else\ifnum\pdfstrcmp{#1}{psnr}=0\def\prberkley@XC@out{26.48}\else\ifnum\pdfstrcmp{#1}{duration}=0\def\prberkley@XC@out{0:50:33}\else\def\prberkley@XC@out{??}\fi\fi\fi\prberkley@XC@out}\newcommand\prberkley@XCI[1][all]{\ifnum\pdfstrcmp{#1}{all}=0\def\prberkley@XCI@out{\{"psnr": 28.19, "duration": "0:29:43"\}}\else\ifnum\pdfstrcmp{#1}{psnr}=0\def\prberkley@XCI@out{28.19}\else\ifnum\pdfstrcmp{#1}{duration}=0\def\prberkley@XCI@out{0:29:43}\else\def\prberkley@XCI@out{??}\fi\fi\fi\prberkley@XCI@out}\newcommand\prberkley@XCII[1][all]{\ifnum\pdfstrcmp{#1}{all}=0\def\prberkley@XCII@out{\{"psnr": 30.26, "duration": "0:31:28"\}}\else\ifnum\pdfstrcmp{#1}{psnr}=0\def\prberkley@XCII@out{30.26}\else\ifnum\pdfstrcmp{#1}{duration}=0\def\prberkley@XCII@out{0:31:28}\else\def\prberkley@XCII@out{??}\fi\fi\fi\prberkley@XCII@out}\newcommand\prberkley@XCIII[1][all]{\ifnum\pdfstrcmp{#1}{all}=0\def\prberkley@XCIII@out{\{"psnr": 31.01, "duration": "0:51:15"\}}\else\ifnum\pdfstrcmp{#1}{psnr}=0\def\prberkley@XCIII@out{31.01}\else\ifnum\pdfstrcmp{#1}{duration}=0\def\prberkley@XCIII@out{0:51:15}\else\def\prberkley@XCIII@out{??}\fi\fi\fi\prberkley@XCIII@out}\newcommand\prberkley@XCIV[1][all]{\ifnum\pdfstrcmp{#1}{all}=0\def\prberkley@XCIV@out{\{"psnr": 25.18, "duration": "0:27:54"\}}\else\ifnum\pdfstrcmp{#1}{psnr}=0\def\prberkley@XCIV@out{25.18}\else\ifnum\pdfstrcmp{#1}{duration}=0\def\prberkley@XCIV@out{0:27:54}\else\def\prberkley@XCIV@out{??}\fi\fi\fi\prberkley@XCIV@out}\newcommand\prberkley@XCV[1][all]{\ifnum\pdfstrcmp{#1}{all}=0\def\prberkley@XCV@out{\{"psnr": 26.8, "duration": "0:30:51"\}}\else\ifnum\pdfstrcmp{#1}{psnr}=0\def\prberkley@XCV@out{26.8}\else\ifnum\pdfstrcmp{#1}{duration}=0\def\prberkley@XCV@out{0:30:51}\else\def\prberkley@XCV@out{??}\fi\fi\fi\prberkley@XCV@out}\newcommand\prberkley@XCVI[1][all]{\ifnum\pdfstrcmp{#1}{all}=0\def\prberkley@XCVI@out{\{"psnr": 27.53, "duration": "0:50:22"\}}\else\ifnum\pdfstrcmp{#1}{psnr}=0\def\prberkley@XCVI@out{27.53}\else\ifnum\pdfstrcmp{#1}{duration}=0\def\prberkley@XCVI@out{0:50:22}\else\def\prberkley@XCVI@out{??}\fi\fi\fi\prberkley@XCVI@out}\newcommand\prberkley@XCVII[1][all]{\ifnum\pdfstrcmp{#1}{all}=0\def\prberkley@XCVII@out{\{"psnr": 23.38, "duration": "0:26:30"\}}\else\ifnum\pdfstrcmp{#1}{psnr}=0\def\prberkley@XCVII@out{23.38}\else\ifnum\pdfstrcmp{#1}{duration}=0\def\prberkley@XCVII@out{0:26:30}\else\def\prberkley@XCVII@out{??}\fi\fi\fi\prberkley@XCVII@out}\newcommand\prberkley@XCVIII[1][all]{\ifnum\pdfstrcmp{#1}{all}=0\def\prberkley@XCVIII@out{\{"psnr": 24.31, "duration": "0:30:40"\}}\else\ifnum\pdfstrcmp{#1}{psnr}=0\def\prberkley@XCVIII@out{24.31}\else\ifnum\pdfstrcmp{#1}{duration}=0\def\prberkley@XCVIII@out{0:30:40}\else\def\prberkley@XCVIII@out{??}\fi\fi\fi\prberkley@XCVIII@out}\newcommand\prberkley@XCIX[1][all]{\ifnum\pdfstrcmp{#1}{all}=0\def\prberkley@XCIX@out{\{"psnr": 24.53, "duration": "0:51:02"\}}\else\ifnum\pdfstrcmp{#1}{psnr}=0\def\prberkley@XCIX@out{24.53}\else\ifnum\pdfstrcmp{#1}{duration}=0\def\prberkley@XCIX@out{0:51:02}\else\def\prberkley@XCIX@out{??}\fi\fi\fi\prberkley@XCIX@out}\newcommand\prberkley@C[1][all]{\ifnum\pdfstrcmp{#1}{all}=0\def\prberkley@C@out{\{"psnr": 27.71, "duration": "0:31:38"\}}\else\ifnum\pdfstrcmp{#1}{psnr}=0\def\prberkley@C@out{27.71}\else\ifnum\pdfstrcmp{#1}{duration}=0\def\prberkley@C@out{0:31:38}\else\def\prberkley@C@out{??}\fi\fi\fi\prberkley@C@out}\newcommand\prberkley@CI[1][all]{\ifnum\pdfstrcmp{#1}{all}=0\def\prberkley@CI@out{\{"psnr": 29.67, "duration": "0:32:17"\}}\else\ifnum\pdfstrcmp{#1}{psnr}=0\def\prberkley@CI@out{29.67}\else\ifnum\pdfstrcmp{#1}{duration}=0\def\prberkley@CI@out{0:32:17}\else\def\prberkley@CI@out{??}\fi\fi\fi\prberkley@CI@out}\newcommand\prberkley@CII[1][all]{\ifnum\pdfstrcmp{#1}{all}=0\def\prberkley@CII@out{\{"psnr": 30.54, "duration": "0:48:16"\}}\else\ifnum\pdfstrcmp{#1}{psnr}=0\def\prberkley@CII@out{30.54}\else\ifnum\pdfstrcmp{#1}{duration}=0\def\prberkley@CII@out{0:48:16}\else\def\prberkley@CII@out{??}\fi\fi\fi\prberkley@CII@out}\newcommand\prberkley@CIII[1][all]{\ifnum\pdfstrcmp{#1}{all}=0\def\prberkley@CIII@out{\{"psnr": 26.14, "duration": "0:25:38"\}}\else\ifnum\pdfstrcmp{#1}{psnr}=0\def\prberkley@CIII@out{26.14}\else\ifnum\pdfstrcmp{#1}{duration}=0\def\prberkley@CIII@out{0:25:38}\else\def\prberkley@CIII@out{??}\fi\fi\fi\prberkley@CIII@out}\newcommand\prberkley@CIV[1][all]{\ifnum\pdfstrcmp{#1}{all}=0\def\prberkley@CIV@out{\{"psnr": 27.67, "duration": "0:29:11"\}}\else\ifnum\pdfstrcmp{#1}{psnr}=0\def\prberkley@CIV@out{27.67}\else\ifnum\pdfstrcmp{#1}{duration}=0\def\prberkley@CIV@out{0:29:11}\else\def\prberkley@CIV@out{??}\fi\fi\fi\prberkley@CIV@out}\newcommand\prberkley@CV[1][all]{\ifnum\pdfstrcmp{#1}{all}=0\def\prberkley@CV@out{\{"psnr": 28.32, "duration": "0:48:03"\}}\else\ifnum\pdfstrcmp{#1}{psnr}=0\def\prberkley@CV@out{28.32}\else\ifnum\pdfstrcmp{#1}{duration}=0\def\prberkley@CV@out{0:48:03}\else\def\prberkley@CV@out{??}\fi\fi\fi\prberkley@CV@out}\newcommand\prberkley@CVI[1][all]{\ifnum\pdfstrcmp{#1}{all}=0\def\prberkley@CVI@out{\{"psnr": 22.94, "duration": "0:25:37"\}}\else\ifnum\pdfstrcmp{#1}{psnr}=0\def\prberkley@CVI@out{22.94}\else\ifnum\pdfstrcmp{#1}{duration}=0\def\prberkley@CVI@out{0:25:37}\else\def\prberkley@CVI@out{??}\fi\fi\fi\prberkley@CVI@out}\newcommand\prberkley@CVII[1][all]{\ifnum\pdfstrcmp{#1}{all}=0\def\prberkley@CVII@out{\{"psnr": 23.88, "duration": "0:29:15"\}}\else\ifnum\pdfstrcmp{#1}{psnr}=0\def\prberkley@CVII@out{23.88}\else\ifnum\pdfstrcmp{#1}{duration}=0\def\prberkley@CVII@out{0:29:15}\else\def\prberkley@CVII@out{??}\fi\fi\fi\prberkley@CVII@out}\newcommand\prberkley@CVIII[1][all]{\ifnum\pdfstrcmp{#1}{all}=0\def\prberkley@CVIII@out{\{"psnr": 24.23, "duration": "0:48:02"\}}\else\ifnum\pdfstrcmp{#1}{psnr}=0\def\prberkley@CVIII@out{24.23}\else\ifnum\pdfstrcmp{#1}{duration}=0\def\prberkley@CVIII@out{0:48:02}\else\def\prberkley@CVIII@out{??}\fi\fi\fi\prberkley@CVIII@out}\newcommand\prberkley@CIX[1][all]{\ifnum\pdfstrcmp{#1}{all}=0\def\prberkley@CIX@out{\{"psnr": 26.49, "duration": "0:25:35"\}}\else\ifnum\pdfstrcmp{#1}{psnr}=0\def\prberkley@CIX@out{26.49}\else\ifnum\pdfstrcmp{#1}{duration}=0\def\prberkley@CIX@out{0:25:35}\else\def\prberkley@CIX@out{??}\fi\fi\fi\prberkley@CIX@out}\newcommand\prberkley@CX[1][all]{\ifnum\pdfstrcmp{#1}{all}=0\def\prberkley@CX@out{\{"psnr": 27.87, "duration": "0:29:05"\}}\else\ifnum\pdfstrcmp{#1}{psnr}=0\def\prberkley@CX@out{27.87}\else\ifnum\pdfstrcmp{#1}{duration}=0\def\prberkley@CX@out{0:29:05}\else\def\prberkley@CX@out{??}\fi\fi\fi\prberkley@CX@out}\newcommand\prberkley@CXI[1][all]{\ifnum\pdfstrcmp{#1}{all}=0\def\prberkley@CXI@out{\{"psnr": 28.42, "duration": "0:47:55"\}}\else\ifnum\pdfstrcmp{#1}{psnr}=0\def\prberkley@CXI@out{28.42}\else\ifnum\pdfstrcmp{#1}{duration}=0\def\prberkley@CXI@out{0:47:55}\else\def\prberkley@CXI@out{??}\fi\fi\fi\prberkley@CXI@out}\newcommand\prberkley@CXII[1][all]{\ifnum\pdfstrcmp{#1}{all}=0\def\prberkley@CXII@out{\{"psnr": 33.08, "duration": "0:25:32"\}}\else\ifnum\pdfstrcmp{#1}{psnr}=0\def\prberkley@CXII@out{33.08}\else\ifnum\pdfstrcmp{#1}{duration}=0\def\prberkley@CXII@out{0:25:32}\else\def\prberkley@CXII@out{??}\fi\fi\fi\prberkley@CXII@out}\newcommand\prberkley@CXIII[1][all]{\ifnum\pdfstrcmp{#1}{all}=0\def\prberkley@CXIII@out{\{"psnr": 35.36, "duration": "0:29:09"\}}\else\ifnum\pdfstrcmp{#1}{psnr}=0\def\prberkley@CXIII@out{35.36}\else\ifnum\pdfstrcmp{#1}{duration}=0\def\prberkley@CXIII@out{0:29:09}\else\def\prberkley@CXIII@out{??}\fi\fi\fi\prberkley@CXIII@out}\newcommand\prberkley@CXIV[1][all]{\ifnum\pdfstrcmp{#1}{all}=0\def\prberkley@CXIV@out{\{"psnr": 35.77, "duration": "0:48:10"\}}\else\ifnum\pdfstrcmp{#1}{psnr}=0\def\prberkley@CXIV@out{35.77}\else\ifnum\pdfstrcmp{#1}{duration}=0\def\prberkley@CXIV@out{0:48:10}\else\def\prberkley@CXIV@out{??}\fi\fi\fi\prberkley@CXIV@out}\newcommand\prberkley@CXV[1][all]{\ifnum\pdfstrcmp{#1}{all}=0\def\prberkley@CXV@out{\{"psnr": 25.34, "duration": "0:14:35"\}}\else\ifnum\pdfstrcmp{#1}{psnr}=0\def\prberkley@CXV@out{25.34}\else\ifnum\pdfstrcmp{#1}{duration}=0\def\prberkley@CXV@out{0:14:35}\else\def\prberkley@CXV@out{??}\fi\fi\fi\prberkley@CXV@out}\newcommand\prberkley@CXVI[1][all]{\ifnum\pdfstrcmp{#1}{all}=0\def\prberkley@CXVI@out{\{"psnr": 26.34, "duration": "0:19:09"\}}\else\ifnum\pdfstrcmp{#1}{psnr}=0\def\prberkley@CXVI@out{26.34}\else\ifnum\pdfstrcmp{#1}{duration}=0\def\prberkley@CXVI@out{0:19:09}\else\def\prberkley@CXVI@out{??}\fi\fi\fi\prberkley@CXVI@out}\newcommand\prberkley@CXVII[1][all]{\ifnum\pdfstrcmp{#1}{all}=0\def\prberkley@CXVII@out{\{"psnr": 26.92, "duration": "0:30:33"\}}\else\ifnum\pdfstrcmp{#1}{psnr}=0\def\prberkley@CXVII@out{26.92}\else\ifnum\pdfstrcmp{#1}{duration}=0\def\prberkley@CXVII@out{0:30:33}\else\def\prberkley@CXVII@out{??}\fi\fi\fi\prberkley@CXVII@out}\newcommand\prberkley@CXVIII[1][all]{\ifnum\pdfstrcmp{#1}{all}=0\def\prberkley@CXVIII@out{\{"psnr": 19.93, "duration": "0:17:52"\}}\else\ifnum\pdfstrcmp{#1}{psnr}=0\def\prberkley@CXVIII@out{19.93}\else\ifnum\pdfstrcmp{#1}{duration}=0\def\prberkley@CXVIII@out{0:17:52}\else\def\prberkley@CXVIII@out{??}\fi\fi\fi\prberkley@CXVIII@out}\newcommand\prberkley@CXIX[1][all]{\ifnum\pdfstrcmp{#1}{all}=0\def\prberkley@CXIX@out{\{"psnr": 20.25, "duration": "0:24:34"\}}\else\ifnum\pdfstrcmp{#1}{psnr}=0\def\prberkley@CXIX@out{20.25}\else\ifnum\pdfstrcmp{#1}{duration}=0\def\prberkley@CXIX@out{0:24:34}\else\def\prberkley@CXIX@out{??}\fi\fi\fi\prberkley@CXIX@out}\newcommand\prberkley@CXX[1][all]{\ifnum\pdfstrcmp{#1}{all}=0\def\prberkley@CXX@out{\{"psnr": 20.3, "duration": "0:40:22"\}}\else\ifnum\pdfstrcmp{#1}{psnr}=0\def\prberkley@CXX@out{20.3}\else\ifnum\pdfstrcmp{#1}{duration}=0\def\prberkley@CXX@out{0:40:22}\else\def\prberkley@CXX@out{??}\fi\fi\fi\prberkley@CXX@out}\newcommand\prberkley@CXXI[1][all]{\ifnum\pdfstrcmp{#1}{all}=0\def\prberkley@CXXI@out{\{"psnr": 18.49, "duration": "0:21:39"\}}\else\ifnum\pdfstrcmp{#1}{psnr}=0\def\prberkley@CXXI@out{18.49}\else\ifnum\pdfstrcmp{#1}{duration}=0\def\prberkley@CXXI@out{0:21:39}\else\def\prberkley@CXXI@out{??}\fi\fi\fi\prberkley@CXXI@out}\newcommand\prberkley@CXXII[1][all]{\ifnum\pdfstrcmp{#1}{all}=0\def\prberkley@CXXII@out{\{"psnr": 18.73, "duration": "0:25:55"\}}\else\ifnum\pdfstrcmp{#1}{psnr}=0\def\prberkley@CXXII@out{18.73}\else\ifnum\pdfstrcmp{#1}{duration}=0\def\prberkley@CXXII@out{0:25:55}\else\def\prberkley@CXXII@out{??}\fi\fi\fi\prberkley@CXXII@out}\newcommand\prberkley@CXXIII[1][all]{\ifnum\pdfstrcmp{#1}{all}=0\def\prberkley@CXXIII@out{\{"psnr": 18.71, "duration": "0:37:03"\}}\else\ifnum\pdfstrcmp{#1}{psnr}=0\def\prberkley@CXXIII@out{18.71}\else\ifnum\pdfstrcmp{#1}{duration}=0\def\prberkley@CXXIII@out{0:37:03}\else\def\prberkley@CXXIII@out{??}\fi\fi\fi\prberkley@CXXIII@out}\newcommand\prberkley@CXXIV[1][all]{\ifnum\pdfstrcmp{#1}{all}=0\def\prberkley@CXXIV@out{\{"psnr": 22.45, "duration": "0:15:03"\}}\else\ifnum\pdfstrcmp{#1}{psnr}=0\def\prberkley@CXXIV@out{22.45}\else\ifnum\pdfstrcmp{#1}{duration}=0\def\prberkley@CXXIV@out{0:15:03}\else\def\prberkley@CXXIV@out{??}\fi\fi\fi\prberkley@CXXIV@out}\newcommand\prberkley@CXXV[1][all]{\ifnum\pdfstrcmp{#1}{all}=0\def\prberkley@CXXV@out{\{"psnr": 22.93, "duration": "0:19:13"\}}\else\ifnum\pdfstrcmp{#1}{psnr}=0\def\prberkley@CXXV@out{22.93}\else\ifnum\pdfstrcmp{#1}{duration}=0\def\prberkley@CXXV@out{0:19:13}\else\def\prberkley@CXXV@out{??}\fi\fi\fi\prberkley@CXXV@out}\newcommand\prberkley@CXXVI[1][all]{\ifnum\pdfstrcmp{#1}{all}=0\def\prberkley@CXXVI@out{\{"psnr": 23.32, "duration": "0:32:04"\}}\else\ifnum\pdfstrcmp{#1}{psnr}=0\def\prberkley@CXXVI@out{23.32}\else\ifnum\pdfstrcmp{#1}{duration}=0\def\prberkley@CXXVI@out{0:32:04}\else\def\prberkley@CXXVI@out{??}\fi\fi\fi\prberkley@CXXVI@out}\newcommand\prberkley@CXXVII[1][all]{\ifnum\pdfstrcmp{#1}{all}=0\def\prberkley@CXXVII@out{\{"psnr": 24.22, "duration": "0:15:06"\}}\else\ifnum\pdfstrcmp{#1}{psnr}=0\def\prberkley@CXXVII@out{24.22}\else\ifnum\pdfstrcmp{#1}{duration}=0\def\prberkley@CXXVII@out{0:15:06}\else\def\prberkley@CXXVII@out{??}\fi\fi\fi\prberkley@CXXVII@out}\newcommand\prberkley@CXXVIII[1][all]{\ifnum\pdfstrcmp{#1}{all}=0\def\prberkley@CXXVIII@out{\{"psnr": 24.7, "duration": "0:21:06"\}}\else\ifnum\pdfstrcmp{#1}{psnr}=0\def\prberkley@CXXVIII@out{24.7}\else\ifnum\pdfstrcmp{#1}{duration}=0\def\prberkley@CXXVIII@out{0:21:06}\else\def\prberkley@CXXVIII@out{??}\fi\fi\fi\prberkley@CXXVIII@out}\newcommand\prberkley@CXXIX[1][all]{\ifnum\pdfstrcmp{#1}{all}=0\def\prberkley@CXXIX@out{\{"psnr": 24.56, "duration": "0:39:00"\}}\else\ifnum\pdfstrcmp{#1}{psnr}=0\def\prberkley@CXXIX@out{24.56}\else\ifnum\pdfstrcmp{#1}{duration}=0\def\prberkley@CXXIX@out{0:39:00}\else\def\prberkley@CXXIX@out{??}\fi\fi\fi\prberkley@CXXIX@out}\newcommand\prberkley@CXXX[1][all]{\ifnum\pdfstrcmp{#1}{all}=0\def\prberkley@CXXX@out{\{"psnr": 27.41, "duration": "0:18:44"\}}\else\ifnum\pdfstrcmp{#1}{psnr}=0\def\prberkley@CXXX@out{27.41}\else\ifnum\pdfstrcmp{#1}{duration}=0\def\prberkley@CXXX@out{0:18:44}\else\def\prberkley@CXXX@out{??}\fi\fi\fi\prberkley@CXXX@out}\newcommand\prberkley@CXXXI[1][all]{\ifnum\pdfstrcmp{#1}{all}=0\def\prberkley@CXXXI@out{\{"psnr": 29.09, "duration": "0:23:58"\}}\else\ifnum\pdfstrcmp{#1}{psnr}=0\def\prberkley@CXXXI@out{29.09}\else\ifnum\pdfstrcmp{#1}{duration}=0\def\prberkley@CXXXI@out{0:23:58}\else\def\prberkley@CXXXI@out{??}\fi\fi\fi\prberkley@CXXXI@out}\newcommand\prberkley@CXXXII[1][all]{\ifnum\pdfstrcmp{#1}{all}=0\def\prberkley@CXXXII@out{\{"psnr": 29.77, "duration": "0:32:30"\}}\else\ifnum\pdfstrcmp{#1}{psnr}=0\def\prberkley@CXXXII@out{29.77}\else\ifnum\pdfstrcmp{#1}{duration}=0\def\prberkley@CXXXII@out{0:32:30}\else\def\prberkley@CXXXII@out{??}\fi\fi\fi\prberkley@CXXXII@out}\newcommand\prberkley@CXXXIII[1][all]{\ifnum\pdfstrcmp{#1}{all}=0\def\prberkley@CXXXIII@out{\{"psnr": 25.74, "duration": "0:15:12"\}}\else\ifnum\pdfstrcmp{#1}{psnr}=0\def\prberkley@CXXXIII@out{25.74}\else\ifnum\pdfstrcmp{#1}{duration}=0\def\prberkley@CXXXIII@out{0:15:12}\else\def\prberkley@CXXXIII@out{??}\fi\fi\fi\prberkley@CXXXIII@out}\newcommand\prberkley@CXXXIV[1][all]{\ifnum\pdfstrcmp{#1}{all}=0\def\prberkley@CXXXIV@out{\{"psnr": 26.53, "duration": "0:18:24"\}}\else\ifnum\pdfstrcmp{#1}{psnr}=0\def\prberkley@CXXXIV@out{26.53}\else\ifnum\pdfstrcmp{#1}{duration}=0\def\prberkley@CXXXIV@out{0:18:24}\else\def\prberkley@CXXXIV@out{??}\fi\fi\fi\prberkley@CXXXIV@out}\newcommand\prberkley@CXXXV[1][all]{\ifnum\pdfstrcmp{#1}{all}=0\def\prberkley@CXXXV@out{\{"psnr": 26.97, "duration": "0:28:41"\}}\else\ifnum\pdfstrcmp{#1}{psnr}=0\def\prberkley@CXXXV@out{26.97}\else\ifnum\pdfstrcmp{#1}{duration}=0\def\prberkley@CXXXV@out{0:28:41}\else\def\prberkley@CXXXV@out{??}\fi\fi\fi\prberkley@CXXXV@out}\newcommand\prberkley@CXXXVI[1][all]{\ifnum\pdfstrcmp{#1}{all}=0\def\prberkley@CXXXVI@out{\{"psnr": 21.38, "duration": "0:14:22"\}}\else\ifnum\pdfstrcmp{#1}{psnr}=0\def\prberkley@CXXXVI@out{21.38}\else\ifnum\pdfstrcmp{#1}{duration}=0\def\prberkley@CXXXVI@out{0:14:22}\else\def\prberkley@CXXXVI@out{??}\fi\fi\fi\prberkley@CXXXVI@out}\newcommand\prberkley@CXXXVII[1][all]{\ifnum\pdfstrcmp{#1}{all}=0\def\prberkley@CXXXVII@out{\{"psnr": 21.87, "duration": "0:18:20"\}}\else\ifnum\pdfstrcmp{#1}{psnr}=0\def\prberkley@CXXXVII@out{21.87}\else\ifnum\pdfstrcmp{#1}{duration}=0\def\prberkley@CXXXVII@out{0:18:20}\else\def\prberkley@CXXXVII@out{??}\fi\fi\fi\prberkley@CXXXVII@out}\newcommand\prberkley@CXXXVIII[1][all]{\ifnum\pdfstrcmp{#1}{all}=0\def\prberkley@CXXXVIII@out{\{"psnr": 21.91, "duration": "0:38:17"\}}\else\ifnum\pdfstrcmp{#1}{psnr}=0\def\prberkley@CXXXVIII@out{21.91}\else\ifnum\pdfstrcmp{#1}{duration}=0\def\prberkley@CXXXVIII@out{0:38:17}\else\def\prberkley@CXXXVIII@out{??}\fi\fi\fi\prberkley@CXXXVIII@out}\newcommand\prberkley@CXXXIX[1][all]{\ifnum\pdfstrcmp{#1}{all}=0\def\prberkley@CXXXIX@out{\{"psnr": 20.82, "duration": "0:14:46"\}}\else\ifnum\pdfstrcmp{#1}{psnr}=0\def\prberkley@CXXXIX@out{20.82}\else\ifnum\pdfstrcmp{#1}{duration}=0\def\prberkley@CXXXIX@out{0:14:46}\else\def\prberkley@CXXXIX@out{??}\fi\fi\fi\prberkley@CXXXIX@out}\newcommand\prberkley@CXL[1][all]{\ifnum\pdfstrcmp{#1}{all}=0\def\prberkley@CXL@out{\{"psnr": 22.02, "duration": "0:18:42"\}}\else\ifnum\pdfstrcmp{#1}{psnr}=0\def\prberkley@CXL@out{22.02}\else\ifnum\pdfstrcmp{#1}{duration}=0\def\prberkley@CXL@out{0:18:42}\else\def\prberkley@CXL@out{??}\fi\fi\fi\prberkley@CXL@out}\newcommand\prberkley@CXLI[1][all]{\ifnum\pdfstrcmp{#1}{all}=0\def\prberkley@CXLI@out{\{"psnr": 22.37, "duration": "0:30:16"\}}\else\ifnum\pdfstrcmp{#1}{psnr}=0\def\prberkley@CXLI@out{22.37}\else\ifnum\pdfstrcmp{#1}{duration}=0\def\prberkley@CXLI@out{0:30:16}\else\def\prberkley@CXLI@out{??}\fi\fi\fi\prberkley@CXLI@out}\newcommand\prberkley@CXLII[1][all]{\ifnum\pdfstrcmp{#1}{all}=0\def\prberkley@CXLII@out{\{"psnr": 24.97, "duration": "0:15:14"\}}\else\ifnum\pdfstrcmp{#1}{psnr}=0\def\prberkley@CXLII@out{24.97}\else\ifnum\pdfstrcmp{#1}{duration}=0\def\prberkley@CXLII@out{0:15:14}\else\def\prberkley@CXLII@out{??}\fi\fi\fi\prberkley@CXLII@out}\newcommand\prberkley@CXLIII[1][all]{\ifnum\pdfstrcmp{#1}{all}=0\def\prberkley@CXLIII@out{\{"psnr": 26.53, "duration": "0:19:20"\}}\else\ifnum\pdfstrcmp{#1}{psnr}=0\def\prberkley@CXLIII@out{26.53}\else\ifnum\pdfstrcmp{#1}{duration}=0\def\prberkley@CXLIII@out{0:19:20}\else\def\prberkley@CXLIII@out{??}\fi\fi\fi\prberkley@CXLIII@out}\newcommand\prberkley@CXLIV[1][all]{\ifnum\pdfstrcmp{#1}{all}=0\def\prberkley@CXLIV@out{\{"psnr": 27.38, "duration": "0:29:06"\}}\else\ifnum\pdfstrcmp{#1}{psnr}=0\def\prberkley@CXLIV@out{27.38}\else\ifnum\pdfstrcmp{#1}{duration}=0\def\prberkley@CXLIV@out{0:29:06}\else\def\prberkley@CXLIV@out{??}\fi\fi\fi\prberkley@CXLIV@out}\newcommand\prberkley@CXLV[1][all]{\ifnum\pdfstrcmp{#1}{all}=0\def\prberkley@CXLV@out{\{"psnr": 19.73, "duration": "0:15:10"\}}\else\ifnum\pdfstrcmp{#1}{psnr}=0\def\prberkley@CXLV@out{19.73}\else\ifnum\pdfstrcmp{#1}{duration}=0\def\prberkley@CXLV@out{0:15:10}\else\def\prberkley@CXLV@out{??}\fi\fi\fi\prberkley@CXLV@out}\newcommand\prberkley@CXLVI[1][all]{\ifnum\pdfstrcmp{#1}{all}=0\def\prberkley@CXLVI@out{\{"psnr": 20.71, "duration": "0:19:35"\}}\else\ifnum\pdfstrcmp{#1}{psnr}=0\def\prberkley@CXLVI@out{20.71}\else\ifnum\pdfstrcmp{#1}{duration}=0\def\prberkley@CXLVI@out{0:19:35}\else\def\prberkley@CXLVI@out{??}\fi\fi\fi\prberkley@CXLVI@out}\newcommand\prberkley@CXLVII[1][all]{\ifnum\pdfstrcmp{#1}{all}=0\def\prberkley@CXLVII@out{\{"psnr": 21.11, "duration": "0:31:42"\}}\else\ifnum\pdfstrcmp{#1}{psnr}=0\def\prberkley@CXLVII@out{21.11}\else\ifnum\pdfstrcmp{#1}{duration}=0\def\prberkley@CXLVII@out{0:31:42}\else\def\prberkley@CXLVII@out{??}\fi\fi\fi\prberkley@CXLVII@out}\newcommand\prberkley@CXLVIII[1][all]{\ifnum\pdfstrcmp{#1}{all}=0\def\prberkley@CXLVIII@out{\{"psnr": 19.43, "duration": "0:15:52"\}}\else\ifnum\pdfstrcmp{#1}{psnr}=0\def\prberkley@CXLVIII@out{19.43}\else\ifnum\pdfstrcmp{#1}{duration}=0\def\prberkley@CXLVIII@out{0:15:52}\else\def\prberkley@CXLVIII@out{??}\fi\fi\fi\prberkley@CXLVIII@out}\newcommand\prberkley@CXLIX[1][all]{\ifnum\pdfstrcmp{#1}{all}=0\def\prberkley@CXLIX@out{\{"psnr": 20.06, "duration": "0:22:26"\}}\else\ifnum\pdfstrcmp{#1}{psnr}=0\def\prberkley@CXLIX@out{20.06}\else\ifnum\pdfstrcmp{#1}{duration}=0\def\prberkley@CXLIX@out{0:22:26}\else\def\prberkley@CXLIX@out{??}\fi\fi\fi\prberkley@CXLIX@out}\newcommand\prberkley@CL[1][all]{\ifnum\pdfstrcmp{#1}{all}=0\def\prberkley@CL@out{\{"psnr": 20.33, "duration": "0:31:57"\}}\else\ifnum\pdfstrcmp{#1}{psnr}=0\def\prberkley@CL@out{20.33}\else\ifnum\pdfstrcmp{#1}{duration}=0\def\prberkley@CL@out{0:31:57}\else\def\prberkley@CL@out{??}\fi\fi\fi\prberkley@CL@out}\newcommand\prberkley@CLI[1][all]{\ifnum\pdfstrcmp{#1}{all}=0\def\prberkley@CLI@out{\{"psnr": 19.25, "duration": "0:17:05"\}}\else\ifnum\pdfstrcmp{#1}{psnr}=0\def\prberkley@CLI@out{19.25}\else\ifnum\pdfstrcmp{#1}{duration}=0\def\prberkley@CLI@out{0:17:05}\else\def\prberkley@CLI@out{??}\fi\fi\fi\prberkley@CLI@out}\newcommand\prberkley@CLII[1][all]{\ifnum\pdfstrcmp{#1}{all}=0\def\prberkley@CLII@out{\{"psnr": 20.59, "duration": "0:22:14"\}}\else\ifnum\pdfstrcmp{#1}{psnr}=0\def\prberkley@CLII@out{20.59}\else\ifnum\pdfstrcmp{#1}{duration}=0\def\prberkley@CLII@out{0:22:14}\else\def\prberkley@CLII@out{??}\fi\fi\fi\prberkley@CLII@out}\newcommand\prberkley@CLIII[1][all]{\ifnum\pdfstrcmp{#1}{all}=0\def\prberkley@CLIII@out{\{"psnr": 21.08, "duration": "0:33:28"\}}\else\ifnum\pdfstrcmp{#1}{psnr}=0\def\prberkley@CLIII@out{21.08}\else\ifnum\pdfstrcmp{#1}{duration}=0\def\prberkley@CLIII@out{0:33:28}\else\def\prberkley@CLIII@out{??}\fi\fi\fi\prberkley@CLIII@out}\newcommand\prberkley@CLIV[1][all]{\ifnum\pdfstrcmp{#1}{all}=0\def\prberkley@CLIV@out{\{"psnr": 21.57, "duration": "0:22:33"\}}\else\ifnum\pdfstrcmp{#1}{psnr}=0\def\prberkley@CLIV@out{21.57}\else\ifnum\pdfstrcmp{#1}{duration}=0\def\prberkley@CLIV@out{0:22:33}\else\def\prberkley@CLIV@out{??}\fi\fi\fi\prberkley@CLIV@out}\newcommand\prberkley@CLV[1][all]{\ifnum\pdfstrcmp{#1}{all}=0\def\prberkley@CLV@out{\{"psnr": 22.47, "duration": "0:24:28"\}}\else\ifnum\pdfstrcmp{#1}{psnr}=0\def\prberkley@CLV@out{22.47}\else\ifnum\pdfstrcmp{#1}{duration}=0\def\prberkley@CLV@out{0:24:28}\else\def\prberkley@CLV@out{??}\fi\fi\fi\prberkley@CLV@out}\newcommand\prberkley@CLVI[1][all]{\ifnum\pdfstrcmp{#1}{all}=0\def\prberkley@CLVI@out{\{"psnr": 22.53, "duration": "0:45:22"\}}\else\ifnum\pdfstrcmp{#1}{psnr}=0\def\prberkley@CLVI@out{22.53}\else\ifnum\pdfstrcmp{#1}{duration}=0\def\prberkley@CLVI@out{0:45:22}\else\def\prberkley@CLVI@out{??}\fi\fi\fi\prberkley@CLVI@out}\makeatother
\makeatletter\newcommand\cvberkley[1][all]{\ifnum\pdfstrcmp{#1}{all}=0\def\cvberkley@out{\{"cameraman.tif": \{"iterations": \{"TELEA": \{"psnr": 24.05, "duration": "0:00:42"\}, "NS": \{"psnr": 27.33, "duration": "0:00:35"\}\}, "duration": "0:01:17"\}, "house.tif": \{"iterations": \{"TELEA": \{"psnr": 26.48, "duration": "0:00:55"\}, "NS": \{"psnr": 31.79, "duration": "0:00:45"\}\}, "duration": "0:01:41"\}, "jetplane.tif": \{"iterations": \{"TELEA": \{"psnr": 23.63, "duration": "0:00:46"\}, "NS": \{"psnr": 26.62, "duration": "0:00:43"\}\}, "duration": "0:01:29"\}, "lake.tif": \{"iterations": \{"TELEA": \{"psnr": 22.33, "duration": "0:00:42"\}, "NS": \{"psnr": 24.86, "duration": "0:00:44"\}\}, "duration": "0:01:27"\}, "lena.tif": \{"iterations": \{"TELEA": \{"psnr": 25.37, "duration": "0:00:41"\}, "NS": \{"psnr": 28.33, "duration": "0:00:35"\}\}, "duration": "0:01:16"\}, "livingroom.tif": \{"iterations": \{"TELEA": \{"psnr": 23.53, "duration": "0:00:30"\}, "NS": \{"psnr": 25.46, "duration": "0:00:31"\}\}, "duration": "0:01:02"\}, "mandril.tif": \{"iterations": \{"TELEA": \{"psnr": 22.7, "duration": "0:00:31"\}, "NS": \{"psnr": 24.08, "duration": "0:00:31"\}\}, "duration": "0:01:02"\}, "peppers.tif": \{"iterations": \{"TELEA": \{"psnr": 24.85, "duration": "0:00:32"\}, "NS": \{"psnr": 28.01, "duration": "0:00:31"\}\}, "duration": "0:01:04"\}, "pirate.tif": \{"iterations": \{"TELEA": \{"psnr": 24.25, "duration": "0:00:31"\}, "NS": \{"psnr": 26.51, "duration": "0:00:31"\}\}, "duration": "0:01:02"\}, "walkbridge.tif": \{"iterations": \{"TELEA": \{"psnr": 21.71, "duration": "0:00:31"\}, "NS": \{"psnr": 23.32, "duration": "0:00:27"\}\}, "duration": "0:00:59"\}, "woman\_blonde.tif": \{"iterations": \{"TELEA": \{"psnr": 24.51, "duration": "0:00:32"\}, "NS": \{"psnr": 26.37, "duration": "0:00:39"\}\}, "duration": "0:01:11"\}, "woman\_darkhair.tif": \{"iterations": \{"TELEA": \{"psnr": 28.89, "duration": "0:00:39"\}, "NS": \{"psnr": 33.69, "duration": "0:00:39"\}\}, "duration": "0:01:19"\}, "im11.jpg": \{"iterations": \{"TELEA": \{"psnr": 24.55, "duration": "0:00:06"\}, "NS": \{"psnr": 26.04, "duration": "0:00:09"\}\}, "duration": "0:00:15"\}, "im15.jpg": \{"iterations": \{"TELEA": \{"psnr": 20.0, "duration": "0:00:08"\}, "NS": \{"psnr": 20.17, "duration": "0:00:08"\}\}, "duration": "0:00:17"\}, "im16.jpg": \{"iterations": \{"TELEA": \{"psnr": 18.62, "duration": "0:00:08"\}, "NS": \{"psnr": 18.73, "duration": "0:00:09"\}\}, "duration": "0:00:18"\}, "im18.jpg": \{"iterations": \{"TELEA": \{"psnr": 21.81, "duration": "0:00:11"\}, "NS": \{"psnr": 22.96, "duration": "0:00:13"\}\}, "duration": "0:00:24"\}, "im19.jpg": \{"iterations": \{"TELEA": \{"psnr": 24.24, "duration": "0:00:11"\}, "NS": \{"psnr": 24.66, "duration": "0:00:09"\}\}, "duration": "0:00:20"\}, "im24.jpg": \{"iterations": \{"TELEA": \{"psnr": 26.07, "duration": "0:00:09"\}, "NS": \{"psnr": 28.52, "duration": "0:00:08"\}\}, "duration": "0:00:17"\}, "im25.jpg": \{"iterations": \{"TELEA": \{"psnr": 24.74, "duration": "0:00:08"\}, "NS": \{"psnr": 26.33, "duration": "0:00:09"\}\}, "duration": "0:00:18"\}, "im26.jpg": \{"iterations": \{"TELEA": \{"psnr": 21.29, "duration": "0:00:09"\}, "NS": \{"psnr": 21.72, "duration": "0:00:08"\}\}, "duration": "0:00:17"\}, "im28.jpg": \{"iterations": \{"TELEA": \{"psnr": 19.92, "duration": "0:00:09"\}, "NS": \{"psnr": 21.43, "duration": "0:00:09"\}\}, "duration": "0:00:18"\}, "im29.jpg": \{"iterations": \{"TELEA": \{"psnr": 23.95, "duration": "0:00:08"\}, "NS": \{"psnr": 25.42, "duration": "0:00:08"\}\}, "duration": "0:00:17"\}, "im30.jpg": \{"iterations": \{"TELEA": \{"psnr": 19.41, "duration": "0:00:08"\}, "NS": \{"psnr": 20.25, "duration": "0:00:09"\}\}, "duration": "0:00:17"\}, "im12.jpg": \{"iterations": \{"TELEA": \{"psnr": 19.29, "duration": "0:00:08"\}, "NS": \{"psnr": 19.69, "duration": "0:00:08"\}\}, "duration": "0:00:16"\}, "im17.jpg": \{"iterations": \{"TELEA": \{"psnr": 18.64, "duration": "0:00:08"\}, "NS": \{"psnr": 19.87, "duration": "0:00:10"\}\}, "duration": "0:00:19"\}, "im20.jpg": \{"iterations": \{"TELEA": \{"psnr": 21.32, "duration": "0:00:07"\}, "NS": \{"psnr": 22.22, "duration": "0:00:07"\}\}, "duration": "0:00:14"\}\}}\else\ifnum\pdfstrcmp{#1}{cameraman.tif}=0\let\cvberkley@out\cvberkley@I\else\ifnum\pdfstrcmp{#1}{house.tif}=0\let\cvberkley@out\cvberkley@II\else\ifnum\pdfstrcmp{#1}{jetplane.tif}=0\let\cvberkley@out\cvberkley@III\else\ifnum\pdfstrcmp{#1}{lake.tif}=0\let\cvberkley@out\cvberkley@IV\else\ifnum\pdfstrcmp{#1}{lena.tif}=0\let\cvberkley@out\cvberkley@V\else\ifnum\pdfstrcmp{#1}{livingroom.tif}=0\let\cvberkley@out\cvberkley@VI\else\ifnum\pdfstrcmp{#1}{mandril.tif}=0\let\cvberkley@out\cvberkley@VII\else\ifnum\pdfstrcmp{#1}{peppers.tif}=0\let\cvberkley@out\cvberkley@VIII\else\ifnum\pdfstrcmp{#1}{pirate.tif}=0\let\cvberkley@out\cvberkley@IX\else\ifnum\pdfstrcmp{#1}{walkbridge.tif}=0\let\cvberkley@out\cvberkley@X\else\ifnum\pdfstrcmp{#1}{woman_blonde.tif}=0\let\cvberkley@out\cvberkley@XI\else\ifnum\pdfstrcmp{#1}{woman_darkhair.tif}=0\let\cvberkley@out\cvberkley@XII\else\ifnum\pdfstrcmp{#1}{im11.jpg}=0\let\cvberkley@out\cvberkley@XIII\else\ifnum\pdfstrcmp{#1}{im15.jpg}=0\let\cvberkley@out\cvberkley@XIV\else\ifnum\pdfstrcmp{#1}{im16.jpg}=0\let\cvberkley@out\cvberkley@XV\else\ifnum\pdfstrcmp{#1}{im18.jpg}=0\let\cvberkley@out\cvberkley@XVI\else\ifnum\pdfstrcmp{#1}{im19.jpg}=0\let\cvberkley@out\cvberkley@XVII\else\ifnum\pdfstrcmp{#1}{im24.jpg}=0\let\cvberkley@out\cvberkley@XVIII\else\ifnum\pdfstrcmp{#1}{im25.jpg}=0\let\cvberkley@out\cvberkley@XIX\else\ifnum\pdfstrcmp{#1}{im26.jpg}=0\let\cvberkley@out\cvberkley@XX\else\ifnum\pdfstrcmp{#1}{im28.jpg}=0\let\cvberkley@out\cvberkley@XXI\else\ifnum\pdfstrcmp{#1}{im29.jpg}=0\let\cvberkley@out\cvberkley@XXII\else\ifnum\pdfstrcmp{#1}{im30.jpg}=0\let\cvberkley@out\cvberkley@XXIII\else\ifnum\pdfstrcmp{#1}{im12.jpg}=0\let\cvberkley@out\cvberkley@XXIV\else\ifnum\pdfstrcmp{#1}{im17.jpg}=0\let\cvberkley@out\cvberkley@XXV\else\ifnum\pdfstrcmp{#1}{im20.jpg}=0\let\cvberkley@out\cvberkley@XXVI\else\def\cvberkley@out{??}\fi\fi\fi\fi\fi\fi\fi\fi\fi\fi\fi\fi\fi\fi\fi\fi\fi\fi\fi\fi\fi\fi\fi\fi\fi\fi\fi\cvberkley@out}\newcommand\cvberkley@I[1][all]{\ifnum\pdfstrcmp{#1}{all}=0\def\cvberkley@I@out{\{"iterations": \{"TELEA": \{"psnr": 24.05, "duration": "0:00:42"\}, "NS": \{"psnr": 27.33, "duration": "0:00:35"\}\}, "duration": "0:01:17"\}}\else\ifnum\pdfstrcmp{#1}{iterations}=0\let\cvberkley@I@out\cvberkley@XXVII\else\ifnum\pdfstrcmp{#1}{duration}=0\def\cvberkley@I@out{0:01:17}\else\def\cvberkley@I@out{??}\fi\fi\fi\cvberkley@I@out}\newcommand\cvberkley@II[1][all]{\ifnum\pdfstrcmp{#1}{all}=0\def\cvberkley@II@out{\{"iterations": \{"TELEA": \{"psnr": 26.48, "duration": "0:00:55"\}, "NS": \{"psnr": 31.79, "duration": "0:00:45"\}\}, "duration": "0:01:41"\}}\else\ifnum\pdfstrcmp{#1}{iterations}=0\let\cvberkley@II@out\cvberkley@XXVIII\else\ifnum\pdfstrcmp{#1}{duration}=0\def\cvberkley@II@out{0:01:41}\else\def\cvberkley@II@out{??}\fi\fi\fi\cvberkley@II@out}\newcommand\cvberkley@III[1][all]{\ifnum\pdfstrcmp{#1}{all}=0\def\cvberkley@III@out{\{"iterations": \{"TELEA": \{"psnr": 23.63, "duration": "0:00:46"\}, "NS": \{"psnr": 26.62, "duration": "0:00:43"\}\}, "duration": "0:01:29"\}}\else\ifnum\pdfstrcmp{#1}{iterations}=0\let\cvberkley@III@out\cvberkley@XXIX\else\ifnum\pdfstrcmp{#1}{duration}=0\def\cvberkley@III@out{0:01:29}\else\def\cvberkley@III@out{??}\fi\fi\fi\cvberkley@III@out}\newcommand\cvberkley@IV[1][all]{\ifnum\pdfstrcmp{#1}{all}=0\def\cvberkley@IV@out{\{"iterations": \{"TELEA": \{"psnr": 22.33, "duration": "0:00:42"\}, "NS": \{"psnr": 24.86, "duration": "0:00:44"\}\}, "duration": "0:01:27"\}}\else\ifnum\pdfstrcmp{#1}{iterations}=0\let\cvberkley@IV@out\cvberkley@XXX\else\ifnum\pdfstrcmp{#1}{duration}=0\def\cvberkley@IV@out{0:01:27}\else\def\cvberkley@IV@out{??}\fi\fi\fi\cvberkley@IV@out}\newcommand\cvberkley@V[1][all]{\ifnum\pdfstrcmp{#1}{all}=0\def\cvberkley@V@out{\{"iterations": \{"TELEA": \{"psnr": 25.37, "duration": "0:00:41"\}, "NS": \{"psnr": 28.33, "duration": "0:00:35"\}\}, "duration": "0:01:16"\}}\else\ifnum\pdfstrcmp{#1}{iterations}=0\let\cvberkley@V@out\cvberkley@XXXI\else\ifnum\pdfstrcmp{#1}{duration}=0\def\cvberkley@V@out{0:01:16}\else\def\cvberkley@V@out{??}\fi\fi\fi\cvberkley@V@out}\newcommand\cvberkley@VI[1][all]{\ifnum\pdfstrcmp{#1}{all}=0\def\cvberkley@VI@out{\{"iterations": \{"TELEA": \{"psnr": 23.53, "duration": "0:00:30"\}, "NS": \{"psnr": 25.46, "duration": "0:00:31"\}\}, "duration": "0:01:02"\}}\else\ifnum\pdfstrcmp{#1}{iterations}=0\let\cvberkley@VI@out\cvberkley@XXXII\else\ifnum\pdfstrcmp{#1}{duration}=0\def\cvberkley@VI@out{0:01:02}\else\def\cvberkley@VI@out{??}\fi\fi\fi\cvberkley@VI@out}\newcommand\cvberkley@VII[1][all]{\ifnum\pdfstrcmp{#1}{all}=0\def\cvberkley@VII@out{\{"iterations": \{"TELEA": \{"psnr": 22.7, "duration": "0:00:31"\}, "NS": \{"psnr": 24.08, "duration": "0:00:31"\}\}, "duration": "0:01:02"\}}\else\ifnum\pdfstrcmp{#1}{iterations}=0\let\cvberkley@VII@out\cvberkley@XXXIII\else\ifnum\pdfstrcmp{#1}{duration}=0\def\cvberkley@VII@out{0:01:02}\else\def\cvberkley@VII@out{??}\fi\fi\fi\cvberkley@VII@out}\newcommand\cvberkley@VIII[1][all]{\ifnum\pdfstrcmp{#1}{all}=0\def\cvberkley@VIII@out{\{"iterations": \{"TELEA": \{"psnr": 24.85, "duration": "0:00:32"\}, "NS": \{"psnr": 28.01, "duration": "0:00:31"\}\}, "duration": "0:01:04"\}}\else\ifnum\pdfstrcmp{#1}{iterations}=0\let\cvberkley@VIII@out\cvberkley@XXXIV\else\ifnum\pdfstrcmp{#1}{duration}=0\def\cvberkley@VIII@out{0:01:04}\else\def\cvberkley@VIII@out{??}\fi\fi\fi\cvberkley@VIII@out}\newcommand\cvberkley@IX[1][all]{\ifnum\pdfstrcmp{#1}{all}=0\def\cvberkley@IX@out{\{"iterations": \{"TELEA": \{"psnr": 24.25, "duration": "0:00:31"\}, "NS": \{"psnr": 26.51, "duration": "0:00:31"\}\}, "duration": "0:01:02"\}}\else\ifnum\pdfstrcmp{#1}{iterations}=0\let\cvberkley@IX@out\cvberkley@XXXV\else\ifnum\pdfstrcmp{#1}{duration}=0\def\cvberkley@IX@out{0:01:02}\else\def\cvberkley@IX@out{??}\fi\fi\fi\cvberkley@IX@out}\newcommand\cvberkley@X[1][all]{\ifnum\pdfstrcmp{#1}{all}=0\def\cvberkley@X@out{\{"iterations": \{"TELEA": \{"psnr": 21.71, "duration": "0:00:31"\}, "NS": \{"psnr": 23.32, "duration": "0:00:27"\}\}, "duration": "0:00:59"\}}\else\ifnum\pdfstrcmp{#1}{iterations}=0\let\cvberkley@X@out\cvberkley@XXXVI\else\ifnum\pdfstrcmp{#1}{duration}=0\def\cvberkley@X@out{0:00:59}\else\def\cvberkley@X@out{??}\fi\fi\fi\cvberkley@X@out}\newcommand\cvberkley@XI[1][all]{\ifnum\pdfstrcmp{#1}{all}=0\def\cvberkley@XI@out{\{"iterations": \{"TELEA": \{"psnr": 24.51, "duration": "0:00:32"\}, "NS": \{"psnr": 26.37, "duration": "0:00:39"\}\}, "duration": "0:01:11"\}}\else\ifnum\pdfstrcmp{#1}{iterations}=0\let\cvberkley@XI@out\cvberkley@XXXVII\else\ifnum\pdfstrcmp{#1}{duration}=0\def\cvberkley@XI@out{0:01:11}\else\def\cvberkley@XI@out{??}\fi\fi\fi\cvberkley@XI@out}\newcommand\cvberkley@XII[1][all]{\ifnum\pdfstrcmp{#1}{all}=0\def\cvberkley@XII@out{\{"iterations": \{"TELEA": \{"psnr": 28.89, "duration": "0:00:39"\}, "NS": \{"psnr": 33.69, "duration": "0:00:39"\}\}, "duration": "0:01:19"\}}\else\ifnum\pdfstrcmp{#1}{iterations}=0\let\cvberkley@XII@out\cvberkley@XXXVIII\else\ifnum\pdfstrcmp{#1}{duration}=0\def\cvberkley@XII@out{0:01:19}\else\def\cvberkley@XII@out{??}\fi\fi\fi\cvberkley@XII@out}\newcommand\cvberkley@XIII[1][all]{\ifnum\pdfstrcmp{#1}{all}=0\def\cvberkley@XIII@out{\{"iterations": \{"TELEA": \{"psnr": 24.55, "duration": "0:00:06"\}, "NS": \{"psnr": 26.04, "duration": "0:00:09"\}\}, "duration": "0:00:15"\}}\else\ifnum\pdfstrcmp{#1}{iterations}=0\let\cvberkley@XIII@out\cvberkley@XXXIX\else\ifnum\pdfstrcmp{#1}{duration}=0\def\cvberkley@XIII@out{0:00:15}\else\def\cvberkley@XIII@out{??}\fi\fi\fi\cvberkley@XIII@out}\newcommand\cvberkley@XIV[1][all]{\ifnum\pdfstrcmp{#1}{all}=0\def\cvberkley@XIV@out{\{"iterations": \{"TELEA": \{"psnr": 20.0, "duration": "0:00:08"\}, "NS": \{"psnr": 20.17, "duration": "0:00:08"\}\}, "duration": "0:00:17"\}}\else\ifnum\pdfstrcmp{#1}{iterations}=0\let\cvberkley@XIV@out\cvberkley@XL\else\ifnum\pdfstrcmp{#1}{duration}=0\def\cvberkley@XIV@out{0:00:17}\else\def\cvberkley@XIV@out{??}\fi\fi\fi\cvberkley@XIV@out}\newcommand\cvberkley@XV[1][all]{\ifnum\pdfstrcmp{#1}{all}=0\def\cvberkley@XV@out{\{"iterations": \{"TELEA": \{"psnr": 18.62, "duration": "0:00:08"\}, "NS": \{"psnr": 18.73, "duration": "0:00:09"\}\}, "duration": "0:00:18"\}}\else\ifnum\pdfstrcmp{#1}{iterations}=0\let\cvberkley@XV@out\cvberkley@XLI\else\ifnum\pdfstrcmp{#1}{duration}=0\def\cvberkley@XV@out{0:00:18}\else\def\cvberkley@XV@out{??}\fi\fi\fi\cvberkley@XV@out}\newcommand\cvberkley@XVI[1][all]{\ifnum\pdfstrcmp{#1}{all}=0\def\cvberkley@XVI@out{\{"iterations": \{"TELEA": \{"psnr": 21.81, "duration": "0:00:11"\}, "NS": \{"psnr": 22.96, "duration": "0:00:13"\}\}, "duration": "0:00:24"\}}\else\ifnum\pdfstrcmp{#1}{iterations}=0\let\cvberkley@XVI@out\cvberkley@XLII\else\ifnum\pdfstrcmp{#1}{duration}=0\def\cvberkley@XVI@out{0:00:24}\else\def\cvberkley@XVI@out{??}\fi\fi\fi\cvberkley@XVI@out}\newcommand\cvberkley@XVII[1][all]{\ifnum\pdfstrcmp{#1}{all}=0\def\cvberkley@XVII@out{\{"iterations": \{"TELEA": \{"psnr": 24.24, "duration": "0:00:11"\}, "NS": \{"psnr": 24.66, "duration": "0:00:09"\}\}, "duration": "0:00:20"\}}\else\ifnum\pdfstrcmp{#1}{iterations}=0\let\cvberkley@XVII@out\cvberkley@XLIII\else\ifnum\pdfstrcmp{#1}{duration}=0\def\cvberkley@XVII@out{0:00:20}\else\def\cvberkley@XVII@out{??}\fi\fi\fi\cvberkley@XVII@out}\newcommand\cvberkley@XVIII[1][all]{\ifnum\pdfstrcmp{#1}{all}=0\def\cvberkley@XVIII@out{\{"iterations": \{"TELEA": \{"psnr": 26.07, "duration": "0:00:09"\}, "NS": \{"psnr": 28.52, "duration": "0:00:08"\}\}, "duration": "0:00:17"\}}\else\ifnum\pdfstrcmp{#1}{iterations}=0\let\cvberkley@XVIII@out\cvberkley@XLIV\else\ifnum\pdfstrcmp{#1}{duration}=0\def\cvberkley@XVIII@out{0:00:17}\else\def\cvberkley@XVIII@out{??}\fi\fi\fi\cvberkley@XVIII@out}\newcommand\cvberkley@XIX[1][all]{\ifnum\pdfstrcmp{#1}{all}=0\def\cvberkley@XIX@out{\{"iterations": \{"TELEA": \{"psnr": 24.74, "duration": "0:00:08"\}, "NS": \{"psnr": 26.33, "duration": "0:00:09"\}\}, "duration": "0:00:18"\}}\else\ifnum\pdfstrcmp{#1}{iterations}=0\let\cvberkley@XIX@out\cvberkley@XLV\else\ifnum\pdfstrcmp{#1}{duration}=0\def\cvberkley@XIX@out{0:00:18}\else\def\cvberkley@XIX@out{??}\fi\fi\fi\cvberkley@XIX@out}\newcommand\cvberkley@XX[1][all]{\ifnum\pdfstrcmp{#1}{all}=0\def\cvberkley@XX@out{\{"iterations": \{"TELEA": \{"psnr": 21.29, "duration": "0:00:09"\}, "NS": \{"psnr": 21.72, "duration": "0:00:08"\}\}, "duration": "0:00:17"\}}\else\ifnum\pdfstrcmp{#1}{iterations}=0\let\cvberkley@XX@out\cvberkley@XLVI\else\ifnum\pdfstrcmp{#1}{duration}=0\def\cvberkley@XX@out{0:00:17}\else\def\cvberkley@XX@out{??}\fi\fi\fi\cvberkley@XX@out}\newcommand\cvberkley@XXI[1][all]{\ifnum\pdfstrcmp{#1}{all}=0\def\cvberkley@XXI@out{\{"iterations": \{"TELEA": \{"psnr": 19.92, "duration": "0:00:09"\}, "NS": \{"psnr": 21.43, "duration": "0:00:09"\}\}, "duration": "0:00:18"\}}\else\ifnum\pdfstrcmp{#1}{iterations}=0\let\cvberkley@XXI@out\cvberkley@XLVII\else\ifnum\pdfstrcmp{#1}{duration}=0\def\cvberkley@XXI@out{0:00:18}\else\def\cvberkley@XXI@out{??}\fi\fi\fi\cvberkley@XXI@out}\newcommand\cvberkley@XXII[1][all]{\ifnum\pdfstrcmp{#1}{all}=0\def\cvberkley@XXII@out{\{"iterations": \{"TELEA": \{"psnr": 23.95, "duration": "0:00:08"\}, "NS": \{"psnr": 25.42, "duration": "0:00:08"\}\}, "duration": "0:00:17"\}}\else\ifnum\pdfstrcmp{#1}{iterations}=0\let\cvberkley@XXII@out\cvberkley@XLVIII\else\ifnum\pdfstrcmp{#1}{duration}=0\def\cvberkley@XXII@out{0:00:17}\else\def\cvberkley@XXII@out{??}\fi\fi\fi\cvberkley@XXII@out}\newcommand\cvberkley@XXIII[1][all]{\ifnum\pdfstrcmp{#1}{all}=0\def\cvberkley@XXIII@out{\{"iterations": \{"TELEA": \{"psnr": 19.41, "duration": "0:00:08"\}, "NS": \{"psnr": 20.25, "duration": "0:00:09"\}\}, "duration": "0:00:17"\}}\else\ifnum\pdfstrcmp{#1}{iterations}=0\let\cvberkley@XXIII@out\cvberkley@XLIX\else\ifnum\pdfstrcmp{#1}{duration}=0\def\cvberkley@XXIII@out{0:00:17}\else\def\cvberkley@XXIII@out{??}\fi\fi\fi\cvberkley@XXIII@out}\newcommand\cvberkley@XXIV[1][all]{\ifnum\pdfstrcmp{#1}{all}=0\def\cvberkley@XXIV@out{\{"iterations": \{"TELEA": \{"psnr": 19.29, "duration": "0:00:08"\}, "NS": \{"psnr": 19.69, "duration": "0:00:08"\}\}, "duration": "0:00:16"\}}\else\ifnum\pdfstrcmp{#1}{iterations}=0\let\cvberkley@XXIV@out\cvberkley@L\else\ifnum\pdfstrcmp{#1}{duration}=0\def\cvberkley@XXIV@out{0:00:16}\else\def\cvberkley@XXIV@out{??}\fi\fi\fi\cvberkley@XXIV@out}\newcommand\cvberkley@XXV[1][all]{\ifnum\pdfstrcmp{#1}{all}=0\def\cvberkley@XXV@out{\{"iterations": \{"TELEA": \{"psnr": 18.64, "duration": "0:00:08"\}, "NS": \{"psnr": 19.87, "duration": "0:00:10"\}\}, "duration": "0:00:19"\}}\else\ifnum\pdfstrcmp{#1}{iterations}=0\let\cvberkley@XXV@out\cvberkley@LI\else\ifnum\pdfstrcmp{#1}{duration}=0\def\cvberkley@XXV@out{0:00:19}\else\def\cvberkley@XXV@out{??}\fi\fi\fi\cvberkley@XXV@out}\newcommand\cvberkley@XXVI[1][all]{\ifnum\pdfstrcmp{#1}{all}=0\def\cvberkley@XXVI@out{\{"iterations": \{"TELEA": \{"psnr": 21.32, "duration": "0:00:07"\}, "NS": \{"psnr": 22.22, "duration": "0:00:07"\}\}, "duration": "0:00:14"\}}\else\ifnum\pdfstrcmp{#1}{iterations}=0\let\cvberkley@XXVI@out\cvberkley@LII\else\ifnum\pdfstrcmp{#1}{duration}=0\def\cvberkley@XXVI@out{0:00:14}\else\def\cvberkley@XXVI@out{??}\fi\fi\fi\cvberkley@XXVI@out}\newcommand\cvberkley@XXVII[1][all]{\ifnum\pdfstrcmp{#1}{all}=0\def\cvberkley@XXVII@out{\{"TELEA": \{"psnr": 24.05, "duration": "0:00:42"\}, "NS": \{"psnr": 27.33, "duration": "0:00:35"\}\}}\else\ifnum\pdfstrcmp{#1}{TELEA}=0\let\cvberkley@XXVII@out\cvberkley@LIII\else\ifnum\pdfstrcmp{#1}{NS}=0\let\cvberkley@XXVII@out\cvberkley@LIV\else\def\cvberkley@XXVII@out{??}\fi\fi\fi\cvberkley@XXVII@out}\newcommand\cvberkley@XXVIII[1][all]{\ifnum\pdfstrcmp{#1}{all}=0\def\cvberkley@XXVIII@out{\{"TELEA": \{"psnr": 26.48, "duration": "0:00:55"\}, "NS": \{"psnr": 31.79, "duration": "0:00:45"\}\}}\else\ifnum\pdfstrcmp{#1}{TELEA}=0\let\cvberkley@XXVIII@out\cvberkley@LV\else\ifnum\pdfstrcmp{#1}{NS}=0\let\cvberkley@XXVIII@out\cvberkley@LVI\else\def\cvberkley@XXVIII@out{??}\fi\fi\fi\cvberkley@XXVIII@out}\newcommand\cvberkley@XXIX[1][all]{\ifnum\pdfstrcmp{#1}{all}=0\def\cvberkley@XXIX@out{\{"TELEA": \{"psnr": 23.63, "duration": "0:00:46"\}, "NS": \{"psnr": 26.62, "duration": "0:00:43"\}\}}\else\ifnum\pdfstrcmp{#1}{TELEA}=0\let\cvberkley@XXIX@out\cvberkley@LVII\else\ifnum\pdfstrcmp{#1}{NS}=0\let\cvberkley@XXIX@out\cvberkley@LVIII\else\def\cvberkley@XXIX@out{??}\fi\fi\fi\cvberkley@XXIX@out}\newcommand\cvberkley@XXX[1][all]{\ifnum\pdfstrcmp{#1}{all}=0\def\cvberkley@XXX@out{\{"TELEA": \{"psnr": 22.33, "duration": "0:00:42"\}, "NS": \{"psnr": 24.86, "duration": "0:00:44"\}\}}\else\ifnum\pdfstrcmp{#1}{TELEA}=0\let\cvberkley@XXX@out\cvberkley@LIX\else\ifnum\pdfstrcmp{#1}{NS}=0\let\cvberkley@XXX@out\cvberkley@LX\else\def\cvberkley@XXX@out{??}\fi\fi\fi\cvberkley@XXX@out}\newcommand\cvberkley@XXXI[1][all]{\ifnum\pdfstrcmp{#1}{all}=0\def\cvberkley@XXXI@out{\{"TELEA": \{"psnr": 25.37, "duration": "0:00:41"\}, "NS": \{"psnr": 28.33, "duration": "0:00:35"\}\}}\else\ifnum\pdfstrcmp{#1}{TELEA}=0\let\cvberkley@XXXI@out\cvberkley@LXI\else\ifnum\pdfstrcmp{#1}{NS}=0\let\cvberkley@XXXI@out\cvberkley@LXII\else\def\cvberkley@XXXI@out{??}\fi\fi\fi\cvberkley@XXXI@out}\newcommand\cvberkley@XXXII[1][all]{\ifnum\pdfstrcmp{#1}{all}=0\def\cvberkley@XXXII@out{\{"TELEA": \{"psnr": 23.53, "duration": "0:00:30"\}, "NS": \{"psnr": 25.46, "duration": "0:00:31"\}\}}\else\ifnum\pdfstrcmp{#1}{TELEA}=0\let\cvberkley@XXXII@out\cvberkley@LXIII\else\ifnum\pdfstrcmp{#1}{NS}=0\let\cvberkley@XXXII@out\cvberkley@LXIV\else\def\cvberkley@XXXII@out{??}\fi\fi\fi\cvberkley@XXXII@out}\newcommand\cvberkley@XXXIII[1][all]{\ifnum\pdfstrcmp{#1}{all}=0\def\cvberkley@XXXIII@out{\{"TELEA": \{"psnr": 22.7, "duration": "0:00:31"\}, "NS": \{"psnr": 24.08, "duration": "0:00:31"\}\}}\else\ifnum\pdfstrcmp{#1}{TELEA}=0\let\cvberkley@XXXIII@out\cvberkley@LXV\else\ifnum\pdfstrcmp{#1}{NS}=0\let\cvberkley@XXXIII@out\cvberkley@LXVI\else\def\cvberkley@XXXIII@out{??}\fi\fi\fi\cvberkley@XXXIII@out}\newcommand\cvberkley@XXXIV[1][all]{\ifnum\pdfstrcmp{#1}{all}=0\def\cvberkley@XXXIV@out{\{"TELEA": \{"psnr": 24.85, "duration": "0:00:32"\}, "NS": \{"psnr": 28.01, "duration": "0:00:31"\}\}}\else\ifnum\pdfstrcmp{#1}{TELEA}=0\let\cvberkley@XXXIV@out\cvberkley@LXVII\else\ifnum\pdfstrcmp{#1}{NS}=0\let\cvberkley@XXXIV@out\cvberkley@LXVIII\else\def\cvberkley@XXXIV@out{??}\fi\fi\fi\cvberkley@XXXIV@out}\newcommand\cvberkley@XXXV[1][all]{\ifnum\pdfstrcmp{#1}{all}=0\def\cvberkley@XXXV@out{\{"TELEA": \{"psnr": 24.25, "duration": "0:00:31"\}, "NS": \{"psnr": 26.51, "duration": "0:00:31"\}\}}\else\ifnum\pdfstrcmp{#1}{TELEA}=0\let\cvberkley@XXXV@out\cvberkley@LXIX\else\ifnum\pdfstrcmp{#1}{NS}=0\let\cvberkley@XXXV@out\cvberkley@LXX\else\def\cvberkley@XXXV@out{??}\fi\fi\fi\cvberkley@XXXV@out}\newcommand\cvberkley@XXXVI[1][all]{\ifnum\pdfstrcmp{#1}{all}=0\def\cvberkley@XXXVI@out{\{"TELEA": \{"psnr": 21.71, "duration": "0:00:31"\}, "NS": \{"psnr": 23.32, "duration": "0:00:27"\}\}}\else\ifnum\pdfstrcmp{#1}{TELEA}=0\let\cvberkley@XXXVI@out\cvberkley@LXXI\else\ifnum\pdfstrcmp{#1}{NS}=0\let\cvberkley@XXXVI@out\cvberkley@LXXII\else\def\cvberkley@XXXVI@out{??}\fi\fi\fi\cvberkley@XXXVI@out}\newcommand\cvberkley@XXXVII[1][all]{\ifnum\pdfstrcmp{#1}{all}=0\def\cvberkley@XXXVII@out{\{"TELEA": \{"psnr": 24.51, "duration": "0:00:32"\}, "NS": \{"psnr": 26.37, "duration": "0:00:39"\}\}}\else\ifnum\pdfstrcmp{#1}{TELEA}=0\let\cvberkley@XXXVII@out\cvberkley@LXXIII\else\ifnum\pdfstrcmp{#1}{NS}=0\let\cvberkley@XXXVII@out\cvberkley@LXXIV\else\def\cvberkley@XXXVII@out{??}\fi\fi\fi\cvberkley@XXXVII@out}\newcommand\cvberkley@XXXVIII[1][all]{\ifnum\pdfstrcmp{#1}{all}=0\def\cvberkley@XXXVIII@out{\{"TELEA": \{"psnr": 28.89, "duration": "0:00:39"\}, "NS": \{"psnr": 33.69, "duration": "0:00:39"\}\}}\else\ifnum\pdfstrcmp{#1}{TELEA}=0\let\cvberkley@XXXVIII@out\cvberkley@LXXV\else\ifnum\pdfstrcmp{#1}{NS}=0\let\cvberkley@XXXVIII@out\cvberkley@LXXVI\else\def\cvberkley@XXXVIII@out{??}\fi\fi\fi\cvberkley@XXXVIII@out}\newcommand\cvberkley@XXXIX[1][all]{\ifnum\pdfstrcmp{#1}{all}=0\def\cvberkley@XXXIX@out{\{"TELEA": \{"psnr": 24.55, "duration": "0:00:06"\}, "NS": \{"psnr": 26.04, "duration": "0:00:09"\}\}}\else\ifnum\pdfstrcmp{#1}{TELEA}=0\let\cvberkley@XXXIX@out\cvberkley@LXXVII\else\ifnum\pdfstrcmp{#1}{NS}=0\let\cvberkley@XXXIX@out\cvberkley@LXXVIII\else\def\cvberkley@XXXIX@out{??}\fi\fi\fi\cvberkley@XXXIX@out}\newcommand\cvberkley@XL[1][all]{\ifnum\pdfstrcmp{#1}{all}=0\def\cvberkley@XL@out{\{"TELEA": \{"psnr": 20.0, "duration": "0:00:08"\}, "NS": \{"psnr": 20.17, "duration": "0:00:08"\}\}}\else\ifnum\pdfstrcmp{#1}{TELEA}=0\let\cvberkley@XL@out\cvberkley@LXXIX\else\ifnum\pdfstrcmp{#1}{NS}=0\let\cvberkley@XL@out\cvberkley@LXXX\else\def\cvberkley@XL@out{??}\fi\fi\fi\cvberkley@XL@out}\newcommand\cvberkley@XLI[1][all]{\ifnum\pdfstrcmp{#1}{all}=0\def\cvberkley@XLI@out{\{"TELEA": \{"psnr": 18.62, "duration": "0:00:08"\}, "NS": \{"psnr": 18.73, "duration": "0:00:09"\}\}}\else\ifnum\pdfstrcmp{#1}{TELEA}=0\let\cvberkley@XLI@out\cvberkley@LXXXI\else\ifnum\pdfstrcmp{#1}{NS}=0\let\cvberkley@XLI@out\cvberkley@LXXXII\else\def\cvberkley@XLI@out{??}\fi\fi\fi\cvberkley@XLI@out}\newcommand\cvberkley@XLII[1][all]{\ifnum\pdfstrcmp{#1}{all}=0\def\cvberkley@XLII@out{\{"TELEA": \{"psnr": 21.81, "duration": "0:00:11"\}, "NS": \{"psnr": 22.96, "duration": "0:00:13"\}\}}\else\ifnum\pdfstrcmp{#1}{TELEA}=0\let\cvberkley@XLII@out\cvberkley@LXXXIII\else\ifnum\pdfstrcmp{#1}{NS}=0\let\cvberkley@XLII@out\cvberkley@LXXXIV\else\def\cvberkley@XLII@out{??}\fi\fi\fi\cvberkley@XLII@out}\newcommand\cvberkley@XLIII[1][all]{\ifnum\pdfstrcmp{#1}{all}=0\def\cvberkley@XLIII@out{\{"TELEA": \{"psnr": 24.24, "duration": "0:00:11"\}, "NS": \{"psnr": 24.66, "duration": "0:00:09"\}\}}\else\ifnum\pdfstrcmp{#1}{TELEA}=0\let\cvberkley@XLIII@out\cvberkley@LXXXV\else\ifnum\pdfstrcmp{#1}{NS}=0\let\cvberkley@XLIII@out\cvberkley@LXXXVI\else\def\cvberkley@XLIII@out{??}\fi\fi\fi\cvberkley@XLIII@out}\newcommand\cvberkley@XLIV[1][all]{\ifnum\pdfstrcmp{#1}{all}=0\def\cvberkley@XLIV@out{\{"TELEA": \{"psnr": 26.07, "duration": "0:00:09"\}, "NS": \{"psnr": 28.52, "duration": "0:00:08"\}\}}\else\ifnum\pdfstrcmp{#1}{TELEA}=0\let\cvberkley@XLIV@out\cvberkley@LXXXVII\else\ifnum\pdfstrcmp{#1}{NS}=0\let\cvberkley@XLIV@out\cvberkley@LXXXVIII\else\def\cvberkley@XLIV@out{??}\fi\fi\fi\cvberkley@XLIV@out}\newcommand\cvberkley@XLV[1][all]{\ifnum\pdfstrcmp{#1}{all}=0\def\cvberkley@XLV@out{\{"TELEA": \{"psnr": 24.74, "duration": "0:00:08"\}, "NS": \{"psnr": 26.33, "duration": "0:00:09"\}\}}\else\ifnum\pdfstrcmp{#1}{TELEA}=0\let\cvberkley@XLV@out\cvberkley@LXXXIX\else\ifnum\pdfstrcmp{#1}{NS}=0\let\cvberkley@XLV@out\cvberkley@XC\else\def\cvberkley@XLV@out{??}\fi\fi\fi\cvberkley@XLV@out}\newcommand\cvberkley@XLVI[1][all]{\ifnum\pdfstrcmp{#1}{all}=0\def\cvberkley@XLVI@out{\{"TELEA": \{"psnr": 21.29, "duration": "0:00:09"\}, "NS": \{"psnr": 21.72, "duration": "0:00:08"\}\}}\else\ifnum\pdfstrcmp{#1}{TELEA}=0\let\cvberkley@XLVI@out\cvberkley@XCI\else\ifnum\pdfstrcmp{#1}{NS}=0\let\cvberkley@XLVI@out\cvberkley@XCII\else\def\cvberkley@XLVI@out{??}\fi\fi\fi\cvberkley@XLVI@out}\newcommand\cvberkley@XLVII[1][all]{\ifnum\pdfstrcmp{#1}{all}=0\def\cvberkley@XLVII@out{\{"TELEA": \{"psnr": 19.92, "duration": "0:00:09"\}, "NS": \{"psnr": 21.43, "duration": "0:00:09"\}\}}\else\ifnum\pdfstrcmp{#1}{TELEA}=0\let\cvberkley@XLVII@out\cvberkley@XCIII\else\ifnum\pdfstrcmp{#1}{NS}=0\let\cvberkley@XLVII@out\cvberkley@XCIV\else\def\cvberkley@XLVII@out{??}\fi\fi\fi\cvberkley@XLVII@out}\newcommand\cvberkley@XLVIII[1][all]{\ifnum\pdfstrcmp{#1}{all}=0\def\cvberkley@XLVIII@out{\{"TELEA": \{"psnr": 23.95, "duration": "0:00:08"\}, "NS": \{"psnr": 25.42, "duration": "0:00:08"\}\}}\else\ifnum\pdfstrcmp{#1}{TELEA}=0\let\cvberkley@XLVIII@out\cvberkley@XCV\else\ifnum\pdfstrcmp{#1}{NS}=0\let\cvberkley@XLVIII@out\cvberkley@XCVI\else\def\cvberkley@XLVIII@out{??}\fi\fi\fi\cvberkley@XLVIII@out}\newcommand\cvberkley@XLIX[1][all]{\ifnum\pdfstrcmp{#1}{all}=0\def\cvberkley@XLIX@out{\{"TELEA": \{"psnr": 19.41, "duration": "0:00:08"\}, "NS": \{"psnr": 20.25, "duration": "0:00:09"\}\}}\else\ifnum\pdfstrcmp{#1}{TELEA}=0\let\cvberkley@XLIX@out\cvberkley@XCVII\else\ifnum\pdfstrcmp{#1}{NS}=0\let\cvberkley@XLIX@out\cvberkley@XCVIII\else\def\cvberkley@XLIX@out{??}\fi\fi\fi\cvberkley@XLIX@out}\newcommand\cvberkley@L[1][all]{\ifnum\pdfstrcmp{#1}{all}=0\def\cvberkley@L@out{\{"TELEA": \{"psnr": 19.29, "duration": "0:00:08"\}, "NS": \{"psnr": 19.69, "duration": "0:00:08"\}\}}\else\ifnum\pdfstrcmp{#1}{TELEA}=0\let\cvberkley@L@out\cvberkley@XCIX\else\ifnum\pdfstrcmp{#1}{NS}=0\let\cvberkley@L@out\cvberkley@C\else\def\cvberkley@L@out{??}\fi\fi\fi\cvberkley@L@out}\newcommand\cvberkley@LI[1][all]{\ifnum\pdfstrcmp{#1}{all}=0\def\cvberkley@LI@out{\{"TELEA": \{"psnr": 18.64, "duration": "0:00:08"\}, "NS": \{"psnr": 19.87, "duration": "0:00:10"\}\}}\else\ifnum\pdfstrcmp{#1}{TELEA}=0\let\cvberkley@LI@out\cvberkley@CI\else\ifnum\pdfstrcmp{#1}{NS}=0\let\cvberkley@LI@out\cvberkley@CII\else\def\cvberkley@LI@out{??}\fi\fi\fi\cvberkley@LI@out}\newcommand\cvberkley@LII[1][all]{\ifnum\pdfstrcmp{#1}{all}=0\def\cvberkley@LII@out{\{"TELEA": \{"psnr": 21.32, "duration": "0:00:07"\}, "NS": \{"psnr": 22.22, "duration": "0:00:07"\}\}}\else\ifnum\pdfstrcmp{#1}{TELEA}=0\let\cvberkley@LII@out\cvberkley@CIII\else\ifnum\pdfstrcmp{#1}{NS}=0\let\cvberkley@LII@out\cvberkley@CIV\else\def\cvberkley@LII@out{??}\fi\fi\fi\cvberkley@LII@out}\newcommand\cvberkley@LIII[1][all]{\ifnum\pdfstrcmp{#1}{all}=0\def\cvberkley@LIII@out{\{"psnr": 24.05, "duration": "0:00:42"\}}\else\ifnum\pdfstrcmp{#1}{psnr}=0\def\cvberkley@LIII@out{24.05}\else\ifnum\pdfstrcmp{#1}{duration}=0\def\cvberkley@LIII@out{0:00:42}\else\def\cvberkley@LIII@out{??}\fi\fi\fi\cvberkley@LIII@out}\newcommand\cvberkley@LIV[1][all]{\ifnum\pdfstrcmp{#1}{all}=0\def\cvberkley@LIV@out{\{"psnr": 27.33, "duration": "0:00:35"\}}\else\ifnum\pdfstrcmp{#1}{psnr}=0\def\cvberkley@LIV@out{27.33}\else\ifnum\pdfstrcmp{#1}{duration}=0\def\cvberkley@LIV@out{0:00:35}\else\def\cvberkley@LIV@out{??}\fi\fi\fi\cvberkley@LIV@out}\newcommand\cvberkley@LV[1][all]{\ifnum\pdfstrcmp{#1}{all}=0\def\cvberkley@LV@out{\{"psnr": 26.48, "duration": "0:00:55"\}}\else\ifnum\pdfstrcmp{#1}{psnr}=0\def\cvberkley@LV@out{26.48}\else\ifnum\pdfstrcmp{#1}{duration}=0\def\cvberkley@LV@out{0:00:55}\else\def\cvberkley@LV@out{??}\fi\fi\fi\cvberkley@LV@out}\newcommand\cvberkley@LVI[1][all]{\ifnum\pdfstrcmp{#1}{all}=0\def\cvberkley@LVI@out{\{"psnr": 31.79, "duration": "0:00:45"\}}\else\ifnum\pdfstrcmp{#1}{psnr}=0\def\cvberkley@LVI@out{31.79}\else\ifnum\pdfstrcmp{#1}{duration}=0\def\cvberkley@LVI@out{0:00:45}\else\def\cvberkley@LVI@out{??}\fi\fi\fi\cvberkley@LVI@out}\newcommand\cvberkley@LVII[1][all]{\ifnum\pdfstrcmp{#1}{all}=0\def\cvberkley@LVII@out{\{"psnr": 23.63, "duration": "0:00:46"\}}\else\ifnum\pdfstrcmp{#1}{psnr}=0\def\cvberkley@LVII@out{23.63}\else\ifnum\pdfstrcmp{#1}{duration}=0\def\cvberkley@LVII@out{0:00:46}\else\def\cvberkley@LVII@out{??}\fi\fi\fi\cvberkley@LVII@out}\newcommand\cvberkley@LVIII[1][all]{\ifnum\pdfstrcmp{#1}{all}=0\def\cvberkley@LVIII@out{\{"psnr": 26.62, "duration": "0:00:43"\}}\else\ifnum\pdfstrcmp{#1}{psnr}=0\def\cvberkley@LVIII@out{26.62}\else\ifnum\pdfstrcmp{#1}{duration}=0\def\cvberkley@LVIII@out{0:00:43}\else\def\cvberkley@LVIII@out{??}\fi\fi\fi\cvberkley@LVIII@out}\newcommand\cvberkley@LIX[1][all]{\ifnum\pdfstrcmp{#1}{all}=0\def\cvberkley@LIX@out{\{"psnr": 22.33, "duration": "0:00:42"\}}\else\ifnum\pdfstrcmp{#1}{psnr}=0\def\cvberkley@LIX@out{22.33}\else\ifnum\pdfstrcmp{#1}{duration}=0\def\cvberkley@LIX@out{0:00:42}\else\def\cvberkley@LIX@out{??}\fi\fi\fi\cvberkley@LIX@out}\newcommand\cvberkley@LX[1][all]{\ifnum\pdfstrcmp{#1}{all}=0\def\cvberkley@LX@out{\{"psnr": 24.86, "duration": "0:00:44"\}}\else\ifnum\pdfstrcmp{#1}{psnr}=0\def\cvberkley@LX@out{24.86}\else\ifnum\pdfstrcmp{#1}{duration}=0\def\cvberkley@LX@out{0:00:44}\else\def\cvberkley@LX@out{??}\fi\fi\fi\cvberkley@LX@out}\newcommand\cvberkley@LXI[1][all]{\ifnum\pdfstrcmp{#1}{all}=0\def\cvberkley@LXI@out{\{"psnr": 25.37, "duration": "0:00:41"\}}\else\ifnum\pdfstrcmp{#1}{psnr}=0\def\cvberkley@LXI@out{25.37}\else\ifnum\pdfstrcmp{#1}{duration}=0\def\cvberkley@LXI@out{0:00:41}\else\def\cvberkley@LXI@out{??}\fi\fi\fi\cvberkley@LXI@out}\newcommand\cvberkley@LXII[1][all]{\ifnum\pdfstrcmp{#1}{all}=0\def\cvberkley@LXII@out{\{"psnr": 28.33, "duration": "0:00:35"\}}\else\ifnum\pdfstrcmp{#1}{psnr}=0\def\cvberkley@LXII@out{28.33}\else\ifnum\pdfstrcmp{#1}{duration}=0\def\cvberkley@LXII@out{0:00:35}\else\def\cvberkley@LXII@out{??}\fi\fi\fi\cvberkley@LXII@out}\newcommand\cvberkley@LXIII[1][all]{\ifnum\pdfstrcmp{#1}{all}=0\def\cvberkley@LXIII@out{\{"psnr": 23.53, "duration": "0:00:30"\}}\else\ifnum\pdfstrcmp{#1}{psnr}=0\def\cvberkley@LXIII@out{23.53}\else\ifnum\pdfstrcmp{#1}{duration}=0\def\cvberkley@LXIII@out{0:00:30}\else\def\cvberkley@LXIII@out{??}\fi\fi\fi\cvberkley@LXIII@out}\newcommand\cvberkley@LXIV[1][all]{\ifnum\pdfstrcmp{#1}{all}=0\def\cvberkley@LXIV@out{\{"psnr": 25.46, "duration": "0:00:31"\}}\else\ifnum\pdfstrcmp{#1}{psnr}=0\def\cvberkley@LXIV@out{25.46}\else\ifnum\pdfstrcmp{#1}{duration}=0\def\cvberkley@LXIV@out{0:00:31}\else\def\cvberkley@LXIV@out{??}\fi\fi\fi\cvberkley@LXIV@out}\newcommand\cvberkley@LXV[1][all]{\ifnum\pdfstrcmp{#1}{all}=0\def\cvberkley@LXV@out{\{"psnr": 22.7, "duration": "0:00:31"\}}\else\ifnum\pdfstrcmp{#1}{psnr}=0\def\cvberkley@LXV@out{22.7}\else\ifnum\pdfstrcmp{#1}{duration}=0\def\cvberkley@LXV@out{0:00:31}\else\def\cvberkley@LXV@out{??}\fi\fi\fi\cvberkley@LXV@out}\newcommand\cvberkley@LXVI[1][all]{\ifnum\pdfstrcmp{#1}{all}=0\def\cvberkley@LXVI@out{\{"psnr": 24.08, "duration": "0:00:31"\}}\else\ifnum\pdfstrcmp{#1}{psnr}=0\def\cvberkley@LXVI@out{24.08}\else\ifnum\pdfstrcmp{#1}{duration}=0\def\cvberkley@LXVI@out{0:00:31}\else\def\cvberkley@LXVI@out{??}\fi\fi\fi\cvberkley@LXVI@out}\newcommand\cvberkley@LXVII[1][all]{\ifnum\pdfstrcmp{#1}{all}=0\def\cvberkley@LXVII@out{\{"psnr": 24.85, "duration": "0:00:32"\}}\else\ifnum\pdfstrcmp{#1}{psnr}=0\def\cvberkley@LXVII@out{24.85}\else\ifnum\pdfstrcmp{#1}{duration}=0\def\cvberkley@LXVII@out{0:00:32}\else\def\cvberkley@LXVII@out{??}\fi\fi\fi\cvberkley@LXVII@out}\newcommand\cvberkley@LXVIII[1][all]{\ifnum\pdfstrcmp{#1}{all}=0\def\cvberkley@LXVIII@out{\{"psnr": 28.01, "duration": "0:00:31"\}}\else\ifnum\pdfstrcmp{#1}{psnr}=0\def\cvberkley@LXVIII@out{28.01}\else\ifnum\pdfstrcmp{#1}{duration}=0\def\cvberkley@LXVIII@out{0:00:31}\else\def\cvberkley@LXVIII@out{??}\fi\fi\fi\cvberkley@LXVIII@out}\newcommand\cvberkley@LXIX[1][all]{\ifnum\pdfstrcmp{#1}{all}=0\def\cvberkley@LXIX@out{\{"psnr": 24.25, "duration": "0:00:31"\}}\else\ifnum\pdfstrcmp{#1}{psnr}=0\def\cvberkley@LXIX@out{24.25}\else\ifnum\pdfstrcmp{#1}{duration}=0\def\cvberkley@LXIX@out{0:00:31}\else\def\cvberkley@LXIX@out{??}\fi\fi\fi\cvberkley@LXIX@out}\newcommand\cvberkley@LXX[1][all]{\ifnum\pdfstrcmp{#1}{all}=0\def\cvberkley@LXX@out{\{"psnr": 26.51, "duration": "0:00:31"\}}\else\ifnum\pdfstrcmp{#1}{psnr}=0\def\cvberkley@LXX@out{26.51}\else\ifnum\pdfstrcmp{#1}{duration}=0\def\cvberkley@LXX@out{0:00:31}\else\def\cvberkley@LXX@out{??}\fi\fi\fi\cvberkley@LXX@out}\newcommand\cvberkley@LXXI[1][all]{\ifnum\pdfstrcmp{#1}{all}=0\def\cvberkley@LXXI@out{\{"psnr": 21.71, "duration": "0:00:31"\}}\else\ifnum\pdfstrcmp{#1}{psnr}=0\def\cvberkley@LXXI@out{21.71}\else\ifnum\pdfstrcmp{#1}{duration}=0\def\cvberkley@LXXI@out{0:00:31}\else\def\cvberkley@LXXI@out{??}\fi\fi\fi\cvberkley@LXXI@out}\newcommand\cvberkley@LXXII[1][all]{\ifnum\pdfstrcmp{#1}{all}=0\def\cvberkley@LXXII@out{\{"psnr": 23.32, "duration": "0:00:27"\}}\else\ifnum\pdfstrcmp{#1}{psnr}=0\def\cvberkley@LXXII@out{23.32}\else\ifnum\pdfstrcmp{#1}{duration}=0\def\cvberkley@LXXII@out{0:00:27}\else\def\cvberkley@LXXII@out{??}\fi\fi\fi\cvberkley@LXXII@out}\newcommand\cvberkley@LXXIII[1][all]{\ifnum\pdfstrcmp{#1}{all}=0\def\cvberkley@LXXIII@out{\{"psnr": 24.51, "duration": "0:00:32"\}}\else\ifnum\pdfstrcmp{#1}{psnr}=0\def\cvberkley@LXXIII@out{24.51}\else\ifnum\pdfstrcmp{#1}{duration}=0\def\cvberkley@LXXIII@out{0:00:32}\else\def\cvberkley@LXXIII@out{??}\fi\fi\fi\cvberkley@LXXIII@out}\newcommand\cvberkley@LXXIV[1][all]{\ifnum\pdfstrcmp{#1}{all}=0\def\cvberkley@LXXIV@out{\{"psnr": 26.37, "duration": "0:00:39"\}}\else\ifnum\pdfstrcmp{#1}{psnr}=0\def\cvberkley@LXXIV@out{26.37}\else\ifnum\pdfstrcmp{#1}{duration}=0\def\cvberkley@LXXIV@out{0:00:39}\else\def\cvberkley@LXXIV@out{??}\fi\fi\fi\cvberkley@LXXIV@out}\newcommand\cvberkley@LXXV[1][all]{\ifnum\pdfstrcmp{#1}{all}=0\def\cvberkley@LXXV@out{\{"psnr": 28.89, "duration": "0:00:39"\}}\else\ifnum\pdfstrcmp{#1}{psnr}=0\def\cvberkley@LXXV@out{28.89}\else\ifnum\pdfstrcmp{#1}{duration}=0\def\cvberkley@LXXV@out{0:00:39}\else\def\cvberkley@LXXV@out{??}\fi\fi\fi\cvberkley@LXXV@out}\newcommand\cvberkley@LXXVI[1][all]{\ifnum\pdfstrcmp{#1}{all}=0\def\cvberkley@LXXVI@out{\{"psnr": 33.69, "duration": "0:00:39"\}}\else\ifnum\pdfstrcmp{#1}{psnr}=0\def\cvberkley@LXXVI@out{33.69}\else\ifnum\pdfstrcmp{#1}{duration}=0\def\cvberkley@LXXVI@out{0:00:39}\else\def\cvberkley@LXXVI@out{??}\fi\fi\fi\cvberkley@LXXVI@out}\newcommand\cvberkley@LXXVII[1][all]{\ifnum\pdfstrcmp{#1}{all}=0\def\cvberkley@LXXVII@out{\{"psnr": 24.55, "duration": "0:00:06"\}}\else\ifnum\pdfstrcmp{#1}{psnr}=0\def\cvberkley@LXXVII@out{24.55}\else\ifnum\pdfstrcmp{#1}{duration}=0\def\cvberkley@LXXVII@out{0:00:06}\else\def\cvberkley@LXXVII@out{??}\fi\fi\fi\cvberkley@LXXVII@out}\newcommand\cvberkley@LXXVIII[1][all]{\ifnum\pdfstrcmp{#1}{all}=0\def\cvberkley@LXXVIII@out{\{"psnr": 26.04, "duration": "0:00:09"\}}\else\ifnum\pdfstrcmp{#1}{psnr}=0\def\cvberkley@LXXVIII@out{26.04}\else\ifnum\pdfstrcmp{#1}{duration}=0\def\cvberkley@LXXVIII@out{0:00:09}\else\def\cvberkley@LXXVIII@out{??}\fi\fi\fi\cvberkley@LXXVIII@out}\newcommand\cvberkley@LXXIX[1][all]{\ifnum\pdfstrcmp{#1}{all}=0\def\cvberkley@LXXIX@out{\{"psnr": 20.0, "duration": "0:00:08"\}}\else\ifnum\pdfstrcmp{#1}{psnr}=0\def\cvberkley@LXXIX@out{20.0}\else\ifnum\pdfstrcmp{#1}{duration}=0\def\cvberkley@LXXIX@out{0:00:08}\else\def\cvberkley@LXXIX@out{??}\fi\fi\fi\cvberkley@LXXIX@out}\newcommand\cvberkley@LXXX[1][all]{\ifnum\pdfstrcmp{#1}{all}=0\def\cvberkley@LXXX@out{\{"psnr": 20.17, "duration": "0:00:08"\}}\else\ifnum\pdfstrcmp{#1}{psnr}=0\def\cvberkley@LXXX@out{20.17}\else\ifnum\pdfstrcmp{#1}{duration}=0\def\cvberkley@LXXX@out{0:00:08}\else\def\cvberkley@LXXX@out{??}\fi\fi\fi\cvberkley@LXXX@out}\newcommand\cvberkley@LXXXI[1][all]{\ifnum\pdfstrcmp{#1}{all}=0\def\cvberkley@LXXXI@out{\{"psnr": 18.62, "duration": "0:00:08"\}}\else\ifnum\pdfstrcmp{#1}{psnr}=0\def\cvberkley@LXXXI@out{18.62}\else\ifnum\pdfstrcmp{#1}{duration}=0\def\cvberkley@LXXXI@out{0:00:08}\else\def\cvberkley@LXXXI@out{??}\fi\fi\fi\cvberkley@LXXXI@out}\newcommand\cvberkley@LXXXII[1][all]{\ifnum\pdfstrcmp{#1}{all}=0\def\cvberkley@LXXXII@out{\{"psnr": 18.73, "duration": "0:00:09"\}}\else\ifnum\pdfstrcmp{#1}{psnr}=0\def\cvberkley@LXXXII@out{18.73}\else\ifnum\pdfstrcmp{#1}{duration}=0\def\cvberkley@LXXXII@out{0:00:09}\else\def\cvberkley@LXXXII@out{??}\fi\fi\fi\cvberkley@LXXXII@out}\newcommand\cvberkley@LXXXIII[1][all]{\ifnum\pdfstrcmp{#1}{all}=0\def\cvberkley@LXXXIII@out{\{"psnr": 21.81, "duration": "0:00:11"\}}\else\ifnum\pdfstrcmp{#1}{psnr}=0\def\cvberkley@LXXXIII@out{21.81}\else\ifnum\pdfstrcmp{#1}{duration}=0\def\cvberkley@LXXXIII@out{0:00:11}\else\def\cvberkley@LXXXIII@out{??}\fi\fi\fi\cvberkley@LXXXIII@out}\newcommand\cvberkley@LXXXIV[1][all]{\ifnum\pdfstrcmp{#1}{all}=0\def\cvberkley@LXXXIV@out{\{"psnr": 22.96, "duration": "0:00:13"\}}\else\ifnum\pdfstrcmp{#1}{psnr}=0\def\cvberkley@LXXXIV@out{22.96}\else\ifnum\pdfstrcmp{#1}{duration}=0\def\cvberkley@LXXXIV@out{0:00:13}\else\def\cvberkley@LXXXIV@out{??}\fi\fi\fi\cvberkley@LXXXIV@out}\newcommand\cvberkley@LXXXV[1][all]{\ifnum\pdfstrcmp{#1}{all}=0\def\cvberkley@LXXXV@out{\{"psnr": 24.24, "duration": "0:00:11"\}}\else\ifnum\pdfstrcmp{#1}{psnr}=0\def\cvberkley@LXXXV@out{24.24}\else\ifnum\pdfstrcmp{#1}{duration}=0\def\cvberkley@LXXXV@out{0:00:11}\else\def\cvberkley@LXXXV@out{??}\fi\fi\fi\cvberkley@LXXXV@out}\newcommand\cvberkley@LXXXVI[1][all]{\ifnum\pdfstrcmp{#1}{all}=0\def\cvberkley@LXXXVI@out{\{"psnr": 24.66, "duration": "0:00:09"\}}\else\ifnum\pdfstrcmp{#1}{psnr}=0\def\cvberkley@LXXXVI@out{24.66}\else\ifnum\pdfstrcmp{#1}{duration}=0\def\cvberkley@LXXXVI@out{0:00:09}\else\def\cvberkley@LXXXVI@out{??}\fi\fi\fi\cvberkley@LXXXVI@out}\newcommand\cvberkley@LXXXVII[1][all]{\ifnum\pdfstrcmp{#1}{all}=0\def\cvberkley@LXXXVII@out{\{"psnr": 26.07, "duration": "0:00:09"\}}\else\ifnum\pdfstrcmp{#1}{psnr}=0\def\cvberkley@LXXXVII@out{26.07}\else\ifnum\pdfstrcmp{#1}{duration}=0\def\cvberkley@LXXXVII@out{0:00:09}\else\def\cvberkley@LXXXVII@out{??}\fi\fi\fi\cvberkley@LXXXVII@out}\newcommand\cvberkley@LXXXVIII[1][all]{\ifnum\pdfstrcmp{#1}{all}=0\def\cvberkley@LXXXVIII@out{\{"psnr": 28.52, "duration": "0:00:08"\}}\else\ifnum\pdfstrcmp{#1}{psnr}=0\def\cvberkley@LXXXVIII@out{28.52}\else\ifnum\pdfstrcmp{#1}{duration}=0\def\cvberkley@LXXXVIII@out{0:00:08}\else\def\cvberkley@LXXXVIII@out{??}\fi\fi\fi\cvberkley@LXXXVIII@out}\newcommand\cvberkley@LXXXIX[1][all]{\ifnum\pdfstrcmp{#1}{all}=0\def\cvberkley@LXXXIX@out{\{"psnr": 24.74, "duration": "0:00:08"\}}\else\ifnum\pdfstrcmp{#1}{psnr}=0\def\cvberkley@LXXXIX@out{24.74}\else\ifnum\pdfstrcmp{#1}{duration}=0\def\cvberkley@LXXXIX@out{0:00:08}\else\def\cvberkley@LXXXIX@out{??}\fi\fi\fi\cvberkley@LXXXIX@out}\newcommand\cvberkley@XC[1][all]{\ifnum\pdfstrcmp{#1}{all}=0\def\cvberkley@XC@out{\{"psnr": 26.33, "duration": "0:00:09"\}}\else\ifnum\pdfstrcmp{#1}{psnr}=0\def\cvberkley@XC@out{26.33}\else\ifnum\pdfstrcmp{#1}{duration}=0\def\cvberkley@XC@out{0:00:09}\else\def\cvberkley@XC@out{??}\fi\fi\fi\cvberkley@XC@out}\newcommand\cvberkley@XCI[1][all]{\ifnum\pdfstrcmp{#1}{all}=0\def\cvberkley@XCI@out{\{"psnr": 21.29, "duration": "0:00:09"\}}\else\ifnum\pdfstrcmp{#1}{psnr}=0\def\cvberkley@XCI@out{21.29}\else\ifnum\pdfstrcmp{#1}{duration}=0\def\cvberkley@XCI@out{0:00:09}\else\def\cvberkley@XCI@out{??}\fi\fi\fi\cvberkley@XCI@out}\newcommand\cvberkley@XCII[1][all]{\ifnum\pdfstrcmp{#1}{all}=0\def\cvberkley@XCII@out{\{"psnr": 21.72, "duration": "0:00:08"\}}\else\ifnum\pdfstrcmp{#1}{psnr}=0\def\cvberkley@XCII@out{21.72}\else\ifnum\pdfstrcmp{#1}{duration}=0\def\cvberkley@XCII@out{0:00:08}\else\def\cvberkley@XCII@out{??}\fi\fi\fi\cvberkley@XCII@out}\newcommand\cvberkley@XCIII[1][all]{\ifnum\pdfstrcmp{#1}{all}=0\def\cvberkley@XCIII@out{\{"psnr": 19.92, "duration": "0:00:09"\}}\else\ifnum\pdfstrcmp{#1}{psnr}=0\def\cvberkley@XCIII@out{19.92}\else\ifnum\pdfstrcmp{#1}{duration}=0\def\cvberkley@XCIII@out{0:00:09}\else\def\cvberkley@XCIII@out{??}\fi\fi\fi\cvberkley@XCIII@out}\newcommand\cvberkley@XCIV[1][all]{\ifnum\pdfstrcmp{#1}{all}=0\def\cvberkley@XCIV@out{\{"psnr": 21.43, "duration": "0:00:09"\}}\else\ifnum\pdfstrcmp{#1}{psnr}=0\def\cvberkley@XCIV@out{21.43}\else\ifnum\pdfstrcmp{#1}{duration}=0\def\cvberkley@XCIV@out{0:00:09}\else\def\cvberkley@XCIV@out{??}\fi\fi\fi\cvberkley@XCIV@out}\newcommand\cvberkley@XCV[1][all]{\ifnum\pdfstrcmp{#1}{all}=0\def\cvberkley@XCV@out{\{"psnr": 23.95, "duration": "0:00:08"\}}\else\ifnum\pdfstrcmp{#1}{psnr}=0\def\cvberkley@XCV@out{23.95}\else\ifnum\pdfstrcmp{#1}{duration}=0\def\cvberkley@XCV@out{0:00:08}\else\def\cvberkley@XCV@out{??}\fi\fi\fi\cvberkley@XCV@out}\newcommand\cvberkley@XCVI[1][all]{\ifnum\pdfstrcmp{#1}{all}=0\def\cvberkley@XCVI@out{\{"psnr": 25.42, "duration": "0:00:08"\}}\else\ifnum\pdfstrcmp{#1}{psnr}=0\def\cvberkley@XCVI@out{25.42}\else\ifnum\pdfstrcmp{#1}{duration}=0\def\cvberkley@XCVI@out{0:00:08}\else\def\cvberkley@XCVI@out{??}\fi\fi\fi\cvberkley@XCVI@out}\newcommand\cvberkley@XCVII[1][all]{\ifnum\pdfstrcmp{#1}{all}=0\def\cvberkley@XCVII@out{\{"psnr": 19.41, "duration": "0:00:08"\}}\else\ifnum\pdfstrcmp{#1}{psnr}=0\def\cvberkley@XCVII@out{19.41}\else\ifnum\pdfstrcmp{#1}{duration}=0\def\cvberkley@XCVII@out{0:00:08}\else\def\cvberkley@XCVII@out{??}\fi\fi\fi\cvberkley@XCVII@out}\newcommand\cvberkley@XCVIII[1][all]{\ifnum\pdfstrcmp{#1}{all}=0\def\cvberkley@XCVIII@out{\{"psnr": 20.25, "duration": "0:00:09"\}}\else\ifnum\pdfstrcmp{#1}{psnr}=0\def\cvberkley@XCVIII@out{20.25}\else\ifnum\pdfstrcmp{#1}{duration}=0\def\cvberkley@XCVIII@out{0:00:09}\else\def\cvberkley@XCVIII@out{??}\fi\fi\fi\cvberkley@XCVIII@out}\newcommand\cvberkley@XCIX[1][all]{\ifnum\pdfstrcmp{#1}{all}=0\def\cvberkley@XCIX@out{\{"psnr": 19.29, "duration": "0:00:08"\}}\else\ifnum\pdfstrcmp{#1}{psnr}=0\def\cvberkley@XCIX@out{19.29}\else\ifnum\pdfstrcmp{#1}{duration}=0\def\cvberkley@XCIX@out{0:00:08}\else\def\cvberkley@XCIX@out{??}\fi\fi\fi\cvberkley@XCIX@out}\newcommand\cvberkley@C[1][all]{\ifnum\pdfstrcmp{#1}{all}=0\def\cvberkley@C@out{\{"psnr": 19.69, "duration": "0:00:08"\}}\else\ifnum\pdfstrcmp{#1}{psnr}=0\def\cvberkley@C@out{19.69}\else\ifnum\pdfstrcmp{#1}{duration}=0\def\cvberkley@C@out{0:00:08}\else\def\cvberkley@C@out{??}\fi\fi\fi\cvberkley@C@out}\newcommand\cvberkley@CI[1][all]{\ifnum\pdfstrcmp{#1}{all}=0\def\cvberkley@CI@out{\{"psnr": 18.64, "duration": "0:00:08"\}}\else\ifnum\pdfstrcmp{#1}{psnr}=0\def\cvberkley@CI@out{18.64}\else\ifnum\pdfstrcmp{#1}{duration}=0\def\cvberkley@CI@out{0:00:08}\else\def\cvberkley@CI@out{??}\fi\fi\fi\cvberkley@CI@out}\newcommand\cvberkley@CII[1][all]{\ifnum\pdfstrcmp{#1}{all}=0\def\cvberkley@CII@out{\{"psnr": 19.87, "duration": "0:00:10"\}}\else\ifnum\pdfstrcmp{#1}{psnr}=0\def\cvberkley@CII@out{19.87}\else\ifnum\pdfstrcmp{#1}{duration}=0\def\cvberkley@CII@out{0:00:10}\else\def\cvberkley@CII@out{??}\fi\fi\fi\cvberkley@CII@out}\newcommand\cvberkley@CIII[1][all]{\ifnum\pdfstrcmp{#1}{all}=0\def\cvberkley@CIII@out{\{"psnr": 21.32, "duration": "0:00:07"\}}\else\ifnum\pdfstrcmp{#1}{psnr}=0\def\cvberkley@CIII@out{21.32}\else\ifnum\pdfstrcmp{#1}{duration}=0\def\cvberkley@CIII@out{0:00:07}\else\def\cvberkley@CIII@out{??}\fi\fi\fi\cvberkley@CIII@out}\newcommand\cvberkley@CIV[1][all]{\ifnum\pdfstrcmp{#1}{all}=0\def\cvberkley@CIV@out{\{"psnr": 22.22, "duration": "0:00:07"\}}\else\ifnum\pdfstrcmp{#1}{psnr}=0\def\cvberkley@CIV@out{22.22}\else\ifnum\pdfstrcmp{#1}{duration}=0\def\cvberkley@CIV@out{0:00:07}\else\def\cvberkley@CIV@out{??}\fi\fi\fi\cvberkley@CIV@out}\makeatother
\newcommand{\berkley}[1]{
\texttt{\detokenize{#1}} &
\prberkley[#1][iterations][1][psnr] &
\prberkley[#1][iterations][2][psnr] &
\prberkley[#1][iterations][3][psnr] &
\cvberkley[#1][iterations][TELEA][psnr] &
\cvberkley[#1][iterations][NS][psnr] \\
}

\begin{table}[H]
	\centering
	\begin{tabular}{p{4cm}ccccc}\hline
		Imagen & Iteraci\'on 1 & Iteraci\'on 2 & Iteraci\'on 3 & \textbf{NS} & \textbf{TELEA} \\\hline
		\berkley{cameraman.tif}
		\berkley{house.tif}
		\berkley{jetplane.tif}
		\berkley{lake.tif}
		\berkley{lena.tif}
		\berkley{livingroom.tif}
		\berkley{mandril.tif}
		\berkley{peppers.tif}
		\berkley{pirate.tif}
		\berkley{walkbridge.tif}
		\berkley{woman_blonde.tif}
		\berkley{woman_darkhair.tif}\hline
	\end{tabular}
	\caption{Resultados de los tres tipos de restauraciones para el grupo (A)}
	\label{tab:berkleyA}
\end{table}


\begin{table}[H]
	\centering
	\begin{tabular}{p{4cm}ccccc}\hline
		Imagen & Iteraci\'on 1 & Iteraci\'on 2 & Iteraci\'on 3 & \textbf{NS} & \textbf{TELEA} \\\hline
		\berkley{im11.jpg}
		\berkley{im15.jpg}
		\berkley{im16.jpg}
		\berkley{im18.jpg}
		\berkley{im19.jpg}
		\berkley{im24.jpg}
		\berkley{im25.jpg}
		\berkley{im26.jpg}
		\berkley{im28.jpg}
		\berkley{im29.jpg}
		\berkley{im30.jpg}
		\berkley{im12.jpg}
		\berkley{im17.jpg}
		\berkley{im20.jpg}\hline
	\end{tabular}
	\caption{Resultados de los tres tipos de restauraciones para el grupo (B)}
	\label{tab:berkleyB}
\end{table}