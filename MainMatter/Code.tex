\chapter{Implementaci\'on}\label{chapter:CODE}

En este capítulo se discutirá acerca de la implementación del algoritmo \ref{al:PRA} presentado en el capítulo \ref{chapter:scheme} y del esquema final de la restauraci\'on presentado en el epígrafe \ref{sec:final_scheme}.

\section{Tecnolog\'ias utilizadas}
Todos los algoritmos y el c\'odigo fuente en general ha sido implementado en el lenguaje de programaci\'on \texttt{Python} \textbf{3}, específicamente en su versi\'on \texttt{Python 3.7.3}. Para el trabajo con los archivos de im\'agenes digitales, como son las operaciones de \textit{lectura}\footnote{Operaci\'on de cargar determinados datos en un disco duro o dispositivo de almacenamiento hacia la memoria \textbf{RAM}} y \textit{escritura}\footnote{Operaci\'on inversa de la lectura, copiar determinados datos de la memoria hacia el dispositivo de almacenamiento} se us\'o la librer\'ia \texttt{Pillow} en su versi\'on \texttt{Pillow 6.2.1}. Se trata de un \textit{framework} no oficial sucesor de la libreria \texttt{PIL} (\textit{Python Imaging Library}) para el procesamiento de im\'agenes. \texttt{Pillow} tiene un mantenimiento activo y es una alternativa c\'omoda para los usuarios de \texttt{Python} \textbf{3}. Con la clase \texttt{Image} de esta librer\'ia se modela lo que es una imagen digital y permite mediante sus funciones obtener la representaci\'on matricial asociada. Esta clase Soporta el uso de la mayor\'ia de los formatos para imagenes de mapa de bits como son \texttt{.jpg}, \texttt{.jpeg}, \texttt{.png}, \texttt{.gif}, \textit{etc.} Tiene la opción de interpretar una imagen ya sea en \RGB o en escala de grises.

Para el trabajo con matrices, vectores y otros elementos y operaciones matem\'aticas se hizo uso de librer\'ia \texttt{NumPy} en su versi\'on \texttt{numpy 1.17.4}. Se trata de quizás la librer\'ia m\'as ampliamente usada del paquete científico de \texttt{Python}. La misma provee las herramientas para tratar con las matrices obtenidas mediante \texttt{Pillow} y adem\'as facilita realizar transformaciones y operaciones tanto elementales como avanzadas. Adem\'as de las matrices, se usa tambi\'en el m\'odulo \texttt{numpy.random} \'util para el trabajo con variables aleatorias. \texttt{Numpy} es r\'apida y eficiente, conocida por estar mayormente implementada en el lenguaje \texttt{C} lo que ayuda a evitar los procesos lentos y cargados en memoria t\'ipicos de \texttt{Python}.

Se hace uso tambi\'en de otra de las librer\'ia m\'as famosas del paquete científico, es el caso de \texttt{SciPy}. La versión usada es \texttt{scipy 1.3.2}. De esta se toma el m\'odulo \texttt{scipy.interpolate} el cual contiene, entre otras, una funci\'on para realizar la interpolaci\'on por \textit{splines} c\'ubicos.