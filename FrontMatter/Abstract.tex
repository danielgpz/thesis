\begin{eabstract}

In this work, a detailed study of image restoration is carried out, especially the one based on the smooth ordering of its patches (\SOP/). The implementation of the studied strategy was developed in \texttt{Python} and the work with the subimages was parallelized. The temporal and spatial cost of it was also analyzed in detail. Using the developed implementation, we experimented on a small number of images with the proposed strategy in two case studies, with known and unknown original image. The results were compared with two other known restoration strategies from the literature and in the first case a metric was applied. As a result of the experimentation on the two selected groups of images, the best and worst performing cases are described and improvement strategies and other applications of the \SOP/ are proposed.

\end{eabstract}

\begin{sabstract}

En este trabajo se realiza un estudio minucioso de la restauración de imágenes, y en especial la basada en el ordenamiento suave de sus parches (\SOP/). La implementación de la estrategia estudiada fue desarrollada en \texttt{Python} y en ella se paralelizó el trabajo con las subimágenes. También se analizó detalladamente el costo temporal y espacial de la misma. Usando la implementación desarrollada se experimentó en un número reducido de imágenes con la estrategia propuesta en dos casos de estudio, con imagen original conocida y desconocida. Los resultados se compararon con otras dos estrategias de restauración conocidas de la literatura y en el primer caso se aplicó una métrica. Como resultado de la experimentación sobre los dos grupos de imágenes seleccionados, se describen los casos en los que se obtienen mejores y peores resultados y se proponen estrategias de mejora y otras aplicaciones del \SOP/.

\end{sabstract}